\documentclass{article}

\usepackage{geometry}
\usepackage{makecell}
\usepackage{array}
\usepackage{multicol}
\usepackage[ngerman]{babel}
\usepackage[ngerman]{hyphsubst}
\usepackage{setspace}
\usepackage{changepage}
\usepackage{booktabs}
\usepackage{graphicx}
\usepackage{float}
\newcolumntype{?}{!{\vrule width 1pt}}
\renewcommand\theadalign{tl}
\setstretch{1.10}
\setlength{\parindent}{0pt}

\geometry{top=12mm, left=1cm, right=2cm}
\title{\vspace{-3cm}22W 520.003 Translationswissenschaftliches Proseminar 1}
\author{Andreas Hofer}

\begin{document}
	\section{Einheit 5}
	\subsection{Bibliografie}
	Auflistung aller Quellen, welche in der Arbeit, direkt oder indirekt, genannt werden. Alphabetisch geordnet. \\
	John Benjamins ist meist Amsterdam/Philadelphia
	\begin{itemize}
		\item{Monografie:}
		\begin{itemize}
			\item{Nachname, Vorname (Erscheinungsjahr; ev. Auflage durch hochgestellte Ziffer vor dem Erscheinungsjahr) \textit{Titel des Werkes} [kursiv]. Erscheinungsort: Verlag (Reihe, in der das Werk erschienen ist und Bandnummer).}
			\item{Hoffman, Lothar (1988) \textit{Vom Fachwort zum Fachtext. Beiträge zur Angewandten Linguistik.} Tübingen: Narr (Forum für Fachsprachenforschung 5).}
		\end{itemize}
		\item{Sammelband}
		\begin{itemize}
			\item{Nachname, Vorname (ed./eds.) (Erscheinungsjahr; ev. Auflage durch hochgestellte Ziffer vor dem Erscheinungsjahr) \textit{Titel des Werks} [kursiv]. Erscheinungsort: Verlag}
		\end{itemize}
		\item{Beitrag aus einem Sammelband}
		\begin{itemize}
			\item{Name, Vorname (Erscheinungsjahr) \glqq Titel des Aufsatzes\grqq [in Anführungszeichen], in: Nachname, Vorname (ed.) [bzw eds.] \textit{Titel des Werkes} [kursiv]. Erscheinungsort: Verlag (Reihe, in der das Werk erschienen ist und Bandnummer), Seitenzahlen.}
		\end{itemize}
		\item{Zeitschriften/Journal}
		\begin{itemize}
			\item{Name, Vorname (Erscheinungsjahr) \glqq Titel des Aufsatzes\grqq [in Anführungszeichen], in: \textit{Name der Zeitschrift} [kursiv] Jahrgang:Heftnummer, Seitenzahlen}
			\item{Turner, Graham H./Merrison, Andrew J. (2016) \glqq Doing 'understanding' in dialogue interpreting. Advancing theory and method\grqq , in: Interpreting 18:2, 137-171}
		\end{itemize}
		\item{Website/Internetartikel}
		\item{Wenn eine Website kein Datum hat kann man statt der Klammer [o.j.] schreiben.}
		\begin{itemize}
			\item{Name, Vorname (Jahr) \glqq Titel der Publikation\grqq , in: Angabe der Internetadresse (URL) [Datum der Recherche in eckigen Klammern]}
			\item{UNICEF-Österreich (1990) \glqq UN-Konvention über die Rechte des Kindes\grqq , in: https://unicef.at/fileadmin/media/Kinderrechte/crcger.pdf [22.10.2019], 1-20}
		\end{itemize}
	\end{itemize}

	\subsection{Fließtext}
	\item{Man muss innerhalb des Textes angeben, wenn man etwas zitiert. Quellenangabe vor dem Punkt.}
	\begin{itemize}
		\item{In der Bibliografie immer alle Autoren angeben}
		\item{(Turner/Merrison 2016:138)}
		\item{(Wadensjö 2007:8f.) [f. für folgend]}
		\item{(Wadensjö 2007:8ff.) [ff. für fortfolgende]}
		\item{(Snell-Hornby et al. 1997:275) (Mehr als 4 Autoren)}
		\item{(UNICEF-Österreich 2009:140) Bei Organisationen muss man nicht alle angeben}
		\item{(ibid.:140) gleiche aufeinanderfolgende Quellen}
		\item{(vgl. Wadensjö 2007) Verweist auf eine gesamte Publikation (oder mehrere) bei großen Gedankensträngen}
	\end{itemize}

	\subsection{Direkte Zitate}
	\begin{itemize}
		\item{Im Fließtext}
		\begin{itemize}
			\item{Kürzer als zwei Zeilen}
			\item{Durch Anführungszeichen gekennzeichnet}
		\end{itemize}
		\item{Blockzitat}
		\begin{itemize}
			\item{In eigenem Absatz}
			\item{1 pt. kleiner (11 Times New Roman/10 Arial)}
		\end{itemize}
	\end{itemize}
	\subsection{Indirekte Zitate}
	\begin{itemize}
		\item{Sinngemäße Zitate übernehmen Gedanken anderer AutorInnen und sind so zu gestalten, dass der ursprüngliche Sinn erhalten bleibt, ohne aber den Text wörtlich zu übernehmen}
	\end{itemize}

	\section{Anzeichen wissenschaftlicher Texte}
	\subsection{Was unterscheidet populärwissenschaftliche von wissenschaftlichen Texten?}
	

	

































\end{document}