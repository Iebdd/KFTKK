\documentclass{article}

\usepackage{geometry}
\usepackage{makecell}
\usepackage{array}
\usepackage{multicol}
\usepackage[ngerman=ngerman-x-latest]{hyphsubst}
\usepackage{setspace}
\usepackage{changepage}
\usepackage{booktabs}
\usepackage{graphicx}
\usepackage{float}
\newcolumntype{?}{!{\vrule width 1pt}}
\renewcommand\theadalign{tl}
\setstretch{1.10}
\setlength{\parindent}{0pt}

\geometry{top=12mm, left=1cm, right=2cm}
\title{\vspace{-3cm}22W Fachkommunikation- und Translation}
\author{Andreas Hofer}

\begin{document}
	\section{Einheit 2}
	\subsection{Terminology}
	Terminology is important for translators as terminological competence is translator competence (Austermühl 2010:3). If you don't have familiarity with the specific terminology, then you are unable to communicate effectively. All kinds of communities can have very specialised terminology and accuracy in LSP is especially important. For example an incorrect term can lead to fatal consequences in a medical context, making the accuracy of every translation important. In addition communities expect specialised terminology to be accepted as professionals in a field. The terminological competence is thus a key component of being a competent translator.
	\subsubsection{Defining Terminology}
	What is terminology even? Terminology is: \textit{The set of all terms, standardised or not, belonging to a specific domain}. Bowker also defines terminology as \textit{Any lexical item that poses a challenge and must be researched.}
	The ISO is providing an international list of internationally agreed definitions of key terminology within a specifc field.
	\subsubsection{Concept and Designation}
	A concept is \textit{A unit of knowledge created by a unique combination of characteristics.} Which is not very descriptive. For example a democracy is the idea of a political community in which the people have sovereign power and there is a popular vote. \\
	Another example are migrants and refugees, which are often used interchangeably but have key differences between them.
	For example these two terms share the feature that they are leaving their home country for a relatively long time but the motivation might be very different.
	\textbf{Concepts and Designations:} \\
	Concepts are defined by designations. For example happiness can be defined through several means, for example a smile or an emoji or a specific sentence. Similarly red can be designated by a string of letters, but also by HEX codes or nonverbally with the colour. \\
	For this reason when translating you must find the concept in a language and then find the correct designation in the target language. \\
	Often identifying the corresponding designation is not a straightforward process. A TL and SL may have similar concepts, but they are not necessarily identical. Political offices are often not similar due to similar systems used in differne countries. \\
	Occasionally a TL might not even have a designation for specific concept like the \textit{Speaker of the House of Commons} in the UK.
	\subsubsection{Translation Strategies}
	When you get a new client you first have to find information about the client. A good starting point for terminology would be the client's website. Ideally there is a company glossary which one can use but it often does not exist. \\
	A survey form 2012 showed that a large amount of translators at that time still used mono- and bilingual dictionaries. These two can also be combined in order to find candidates for translating a concept and then checking the specific definition. Online dictionaries are also often superior to physical dictionaries since they tend to be more up-to-date. But care should be taken when using a dictionary that the defined designation does in fact have the correct context. It also works within LSPs where LSP specific terms might not be too useful to an excessive amount of LSP words.
	\textbf{Wikipedia} is also a good starting point for one's research, as it gives a good overview and a good way to find additional resources. But Wikipedia should not be used as a final source as the content is not necessarily (But nevertheless often) correct.\\
	\textbf{Google Searches} can also prove an invaluable tool when trying to find information with wildcard characters and advanced search options. \\
	\subsection{Translation Problems nach Christiane Nord}
	Christiane Nord has defined four major translation problems, each of which describe different problems in translating:
	\begin{itemize}
		\item{Pragmatic Problems}
		\begin{itemize}
			\item{Intertextual references (to other documents, laws or directives)}
			\item{Names, places, peopl,e historical events}
		\end{itemize}
		\item{Intercultural Problems}
		\begin{itemize}
			\item{Convention based problems. For example centimetres and inches but also texttype conventions. Brackets are for example far less common in English than in German}
		\end{itemize}
		\item{Interlingual Problems}
		\begin{itemize}
			\item{Domain specific terms (Verbs and nouns and compound nouns)}
			\item{Structural differences in vocabulary, syntax or grammar}
		\end{itemize}
		\item{Text Specific Problems}
		\begin{itemize}
			\item{Problems arising from combinations of aspects in a particular text}
		\end{itemize}
	\end{itemize}
	\section{Corpora}
	\subsection{What is a Corpus?}
	A corpus (From Latin \textit{corpus, corpora} for body) is a collection of texts. An example could be the Hippocratic corpus, which is the collection of texts authored by Hippocrates. In Humanities though corpus has a more narrow definition: \\
	A corpus should be a large collection of texts adhering to a specific standard.
	\subsubsection{How large should the corpus be?}
	A corpus should be large enough to find secific patterns within a genre. One occurence would not be enough. Qualitative analyses usually include 20 to 25 texts. In a quantitative analysis many more should be included (100+) \\
	In this context qualitative means that the analysis goes deeper than merely counting the number of occurences, as would be usually done in quantitative analyses. \\
	\subsubsection{What criteria?}
	A corpus is not a random collection of texts but a carefully curated one. Based on the purpose of the corpus, the criteria can be different. A starting point could be the domain, genre and text type.
	\subsubsection{What is a corpus?}
	Bowker and Pearson state, that texts in a corpus must be authentic, reliable and representative. \\
	\begin{itemize}
		\item{Authentic:}
		\begin{itemize}
			\item{Genuine communication instead of fabricated examples. A medical exchange from a TV-Show should not be used in a medical corpus.}
		\end{itemize}
		\item{Reliable:}
		\begin{itemize}
			\item{Texts should be from reliable sources. Blogs are often not considered reliable a as anybody can pen a blog.}
		\end{itemize}
		\item{Representative:}
		\begin{itemize}
			\item{Texts have to fit the convention of the specific text you need to ensure that they are relevant to the task.}
		\end{itemize}
	\end{itemize}

	\subsection{Why use a specialised corpus?}
	\begin{itemize}
		\item{Because terminology research is very time consuming.}
		\begin{itemize}
			\item{A study has shown that veteran translators use 25\% of their time with corpus analysis, while inexperiences translators use up to 60\%.}
		\end{itemize}
		\item{You can find collocations for common words much quicker.}
		\begin{itemize}
			\item{Coming back to a medical context it is easier to figure out which preposition to use for gene encoding.}
		\end{itemize}
		\item{You can do a quantitative analysis.}
		\begin{itemize}
			\item{You can figure out which term is more often used. When a term collocates with two words but one is used radically more often, you are wont to use the more common one.}
		\end{itemize}
	\end{itemize}

	There are general and specialised corpora. General corpora tend to have many many more words than specialised ones but are also less accurate for things you might need. For example the British National Corpus (BNC) has 100 million entries, while the Medical Web Corpus \textit{only} has 33 but all of them are related to the medical field.\\
	

	

















	
\end{document}