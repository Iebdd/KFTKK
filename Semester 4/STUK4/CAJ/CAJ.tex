\documentclass{article}

\usepackage{geometry}
\usepackage{makecell}
\usepackage{array}
\usepackage{multicol}
\usepackage[british]{babel}
\usepackage{csquotes}
\usepackage{setspace}
\usepackage{changepage}
\usepackage{hyperref}
\usepackage{booktabs}
\usepackage{fancyhdr}
\usepackage{graphicx}
\usepackage{float}
%\usepackage[style=apa,backend=biber,sorting=ynt]{biblatex}
\graphicspath{ {./Images/} }
%\addbibresource{Literature.bib}
\newcolumntype{?}{!{\vrule width 1pt}}
\renewcommand\theadalign{tl}
\setstretch{1.50}

\pagestyle{fancy}
\fancyhead[L]{Andreas Hofer {\footnotesize (11705024)}}
\fancyhead[C]{STUK 4 CAJ}
\fancyhead[R]{\today}

%\geometry{top=12mm, left=1cm, right=2cm}
\title{\vspace{-3cm}22S 520.341 Englisch: Sprach-, Text- und Kulturkompetenz IV Gruppe 1 KS}
\author{Andreas Hofer}

\begin{document}
	\begingroup
	\section*{Influencer marketing is long overdue stricter regulation}
	\setlength{\parindent}{0pt}
	Within the last eight years influencer marketing has seen a sharp increase in market size, increasing from \$1.7B US-Dollars in 2016 to an estimate of \$16.4B US-Dollars in 2022. (Geyser 2022) This is not overly surprising given the importance of social networks and the amount of influencers and personalities that inhabit them. Yet despite this relative importance within certain demographics, it is still a mostly unregulated market. While both the USA and the EU do have legislation limiting the range of advertisements, there are few real restrictions on what can be advertised or to whom. (Ekşioğlu 2021) Thus, clear restrictions on the content as well as the target audience have been necessary for a long time and are very overdue at this point. \\ \\

	First, in order to understand influencer marketing one must understand its connection to parasocial relationships. Parasocial relationships are a type of social interaction where only one of the parties directly and actively interacts, giving the other party the impression that they know each other or are even friends. Despite this feeling of closeness, the other party often does not even know of their existence as they interact the same way with hundreds, or even hundreds of thousands of people. It has been shown that the formation of a parasocial relationship with one's audience can increase engagement and donations significantly. Patreon, a platform where people can support a content creator through a monthly patronage, encourages their users to share personal details and show emotion in order to foster this type of interaction. (Horton/Wohl 1956:216) Meanwhile content creators often do not have the emotional and financial well-being of their audience in mind, while still enjoying a significant amount of trust from their viewership. This imbalance makes it easy for content creators to exploit the goodwill of their audience by advertising questionable goods or services to them. \\ \\

	In addition, influencers also often assume a role model function, especially for consumer goods like clothes, jewellery or make up, enabling them to act as trend setters to their audience; a dynamic fashion companies have started to capitalise on. \textit{Shein}, a Chinese online clothing store, has become the world's largest fashion firm (Williams 2022) by catering to the Gen Z market, while using influencers as advertisers. By having influencers act as brand ambassadors towards their audience, they can influence the trends of a whole market segment. Yet it is not only the current trends that are being influenced, as hashtags on Twitter and Instagram titled \textit{'\#SheinHauls'} and \textit{'\#Outfit of the Hour'} try changing the purchasing habits of their customers by encouraging large bulk orders and buying an amount of clothes to enable changing your outfit once per hour. As these purchases are often reactions to current trends, with an influencer for example wearing a new type of sweater, they are just as quickly discarded again while following the next big trend, contributing to overconsumption, wastefulness and pollution. \\ \\

	Shein's cooperation with influencers also highlights critical flaws in the declaration of a paid advertisement, by cleverly avoiding the law. Influencers only rarely get paid directly in order to promote a certain product but instead are encouraged to sign up for a partnership program. In this program influencers are assigned an affiliate link, through which they receive a commission whenever a purchase is made using it. By not getting money from the company for making the video but only for purchases made as a result of the video, content creators are not beholden to marking their videos as ads, despite the prospect of referring their user base certainly influencing how positively they present the merchandise. Obfuscation of the nature of videos is prevalent, with another strategy being to increase the length of a video title and placing the disclaimer at the very end, where fewer people will notice it. (Monroe 2021) \\ \\

	Yet the true issue lies with unsatisfactory age controls on paid advertisements. Kids of any age can view content provided by influencers as few platforms check for age and platforms that do, like YouTube, only bar underage viewers from explicit content. (Fersko 2018) Children can, the moment they can interact with a smartphone or tablet, consume content from fashion influencers while not being able to differentiate between normal and paid content, contributing to self-image issues and normalising overconsumption. It can become even more of an issue with content creators being paid to gamble on their platforms. Several streamers on the streaming platform \textit{Twitch} are reportedly being paid several million dollars per month to gamble during their streams. They often publish disclaimers, telling people to not gamble as they will lose but one only has to look to cigarette packaging to know how well warnings like that work when faced with addiction. In addition, given these generous payouts to singular people, one might wonder how much money the gambling platform makes in return and how much of it comes from underage viewers. (D'Anastasio 2021) An informal survey conducted on the subreddit page of \textit{XQC}, the largest streamer on the platform who promotes gambling, averaging 60,000 viewers per stream, came to the conclusion that nearly one-third of his viewership is younger than 17 years and close to three-fourths are under the age of 22, suggesting, that on average 20,000 underage viewers are watching him promote gambling. \\ \\

	Influencer marketing, just like the internet as a whole, has entrenched itself firmly in people's lives and is unlikely to leave again in the near future. But despite it being an integral part of modern society, it has evaded effective legislation both nationally and internationally. We need to wrest control from private companies that decide if, when, where, how, and in what way we interact with the internet as the stakeholders in a public utility should not be profit oriented. How public internet figures sell wares or ideas to their audience may not be the most important issue in the grand scheme of internet regulation but it is an important one nonetheless. \\
	
	Type of English: British English \\
	Word Count: 994
	\newpage
	\endgroup
	\section*{Bibliography:}
	\begin{list}{}{\setlength{\leftmargin}{1cm}\setlength{\itemindent}{-1cm}}
	\item{D'Anastasio, Cecilia (2021) \glqq Twitch streamers rake in millions with a shady crypto gambling boom\grqq, in: https://arstechnica.com/gaming/2021/07/twitch-streamers-rake-in-millions-with-a-shady-crypto-gambling-boom/[14.06.2022]}
	\item{Ekşioğlu, Sıla (2021) \glqq Influencer Marketing Laws in Europe\grqq, in: https://inflownetwork.com/influencer-marketing-laws-in-europe/[14.06.2022]}
	\item{Fersko, Henry (2018) \glqq Is social media bad for teens' mental health\grqq, in: https://www.unicef.org/stories/\\social-media-bad-teens-mental-health[14.06.2022]}
	\item{Geyser, Werner (2022) \glqq The State of Influencer Marketing 2022: Benchmark Report\grqq, in: \\https://influencermarketinghub.com/influencer-marketing-benchmark-report/[14.06.2022]}
	\item{Horton, Donalds/Wohl, R. Richard (1956) \glqq Mass Communication and Para-Social Interaction\grqq, in: \textit{Psychiatry: Interpersonal and Biological Processes} 19:3 215-229}
	\item{Monroe, Rachel (2021) \glqq Ultra-Fast Fashion is Eating the World\grqq, in: \\ https://www.theatlantic.com/magazine/archive/2021/03/ultra-fast-fashion-is-eating-the-world\\/617794/[26.06.2022]}
	\item{Williams, Lara (2022) \glqq Rise of Shein Tests an Industry’s Go-Green Commitments\grqq, in: \\ https://www.bloomberg.com/opinion/articles/2022-04-10/shein-s-100-billion-valuation-is-fast-fashion-s-big-moment[26.06.2022]}
	\end{list}

	
\end{document}

%Influencer marketing and even the internet as a whole are nowadays ubiquitous yet still largely unregulated, not in small part due to its decentralised structure. But despite the lack of regulation, there are still powerful private entities which hold considerable sway. US-American technology companies like Google, Apple or Microsoft provide services used by the vast majority of internet users and also have a hand in the standardisation of its protocols. Yet while these companies are well-known, the registration and approval of a large percent of Top-Level Domains like \texttt{.com} or \texttt{.net} are also in the hands of private companies, albeit often with the US-Government reserving the right to deny applications. With the internet being such an international phenomenon and yet so firmly controlled by private, for-profit companies, it is no wonder that there are no or very few regulations spanning the globe or even only the western world, as that is bad for business. It would require a concerted effort by numerous nations to unify standards, not based on companies' interests but the peoples' and while that may seem daunting it can be achieved, little by little. One of these small steps would be not letting companies ignore local laws, solely because they are based elsewhere and regulate those within the borders. \\ \\