\documentclass{article}

\usepackage{geometry}
\usepackage{makecell}
\usepackage{array}
\usepackage{multicol}
\usepackage{setspace}
\usepackage{changepage}
\usepackage{booktabs}
\usepackage{graphicx}
\usepackage{float}
\usepackage{tikz}
\newcolumntype{?}{!{\vrule width 1pt}}
\renewcommand\theadalign{tl}
\setstretch{1.10}
\setlength{\parindent}{0pt}

%\geometry{top=12mm, left=1cm, right=2cm}
\title{\vspace{-3cm}22S 520.341 Englisch:Sprach-, Text- und Kulturkompetenz}
\author{Andreas Hofer}

\begin{document}
	\begin{tikzpicture}
	{\vspace{5cm}\draw[black] (5,6) rectangle (1,1);}
	\draw (2,0) ellipse (10pt and 5pt);
	\end{tikzpicture}
	\section*{Topic 1: Leaflet - Misrepresenting Data}
	A3 leaflet, where the long sides are folded into each other to make it as large as an A4 piece of paper. On the back side there is a newspaper article about climate change, which uses many of the tools used to falsify data or misrepresent it to serve a different purpose. On the flip side, from which the two arms open up, the newspaper article is printed a second time but with additional information pertaining to specific parts of the article, which outline the devices used. Common forms of misrepresenting data are:
	\begin{itemize}
		\item{Using the fact that scientists dislike to confirm or deny things for certain. If something has a 99\% chance, it can be argued that it has not been proven, even though any person who saw the real data would think so.}
		\item{Cherry picking: Only using the data points that support your point and not mentioning the others. Often only 10\% of the data chosen is actually used which can turn the argument from one side to the other.}
		\item{Messing with the way the data is represented. By reducing the distance of data points a small and likely insignificant drop or increase can be inflated. Similarily by switching the positions of the highest and lowest data point, data that shows a decrease will seem to increase to an unobservant audience.}
		\item{}
	\end{itemize}
	Info boxes
	\begin{itemize}
		\item{One could argue that a question is just asking questions.}
	\end{itemize}
	






















\end{document}