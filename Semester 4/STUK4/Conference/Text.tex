\documentclass{article}

\usepackage{geometry}
\usepackage{silence}
\WarningsOff[sectsty]% Suppress warnings related to package sectsty
\usepackage{makecell}
\usepackage{array}
\usepackage{multicol}
\usepackage{setspace}
\usepackage{changepage}
\usepackage{booktabs}
\usepackage{graphicx}
\usepackage{float}
\usepackage{sectsty}
\newcolumntype{?}{!{\vrule width 1pt}}
\allsectionsfont{\centering}
\renewcommand\theadalign{tl}
\setstretch{1.40}
\setlength{\parindent}{0pt}

%\geometry{top=12mm, left=1cm, right=2cm}
\title{\vspace{-3cm}21S 520.230 Deutsch: Mutter-/Bildungssprache: Textanalyse und Textproduktion Gruppe 3}
\author{Andreas Hofer}

\begin{document}
	\section*{Was climate change just a fad?} 
	\subsection*{\textit{The worst seems to behind us, so why not enjoy ourselves for once?}}
	\Large
	The doomsayers and climate terrorists are ever growing louder and continously disrupt our everyday lives. Little girls from Sweden scream of climate catastrophe. Meanwhile I was hard at work, to bring you my new book which uses data that proves that, for 5 years now, temperatures have actually been decreasing again. This is not overly surprising considering the substantial decrease in carbon emissions by western countries in both relative and absolute numbers which are historically the largest polluters. So when I, at long last, enjoyed the afternoon from my porch, I could already feel the difference, as the breeze underlined the perfect temperature outside. Why dream about the future, when it is already here?

	\normalsize
\setstretch{1}
	\newpage
	\section*{Boxes}
	\textbf{"Was climate change just a fad?" - Loaded question:} Relatively few people believe that and by asking this question in an ostensibly innocous manner, it is more likely that people start thinking about it. People who want to believe it will accept it without question. People who have their doubts might second-guess themselves. And people who shake their head at this brazen manipulation attempt are a lost cause anyway. \\

	\textbf{"Little girls from Sweden scream" - ad hominem:} Whatever Greta Thunberg might have to say, it does not make any difference, since she is a girl and on top of that far too young to have anything worthwhile to say. Just don't bother. Listen to me instead. \\

	\textbf{"to bring you my new book" - appeal to authority:} \textit{I} have written a book about this topic, thus my opinion is more important and reliable than other, less accomplished, people's\\

	\textbf{"data that proves" - appeal to emotion:} Never use the word 'suggest', as there is not enough certainty behind it, which is often the only thing on offer. During uncertain times, people crave nothing more than certainty and simple answers to difficult problems. \\

	\textbf{"temperatures have actually been decreasing again" - false cause:} 2016 was the hottest year on record but previous data has shown that extremes like that do happen, which have, until now, been beaten soon after. Claiming this to be the peak is at best naïve and at worst manipulative. \\

	\textbf{"considering the substantial decrease in carbon emission by western countries" - false cause:} While western countries have decreased their carbon emissions, it is AsianF countries which currently contribute the most to greenhouse emissions. (35\% in Europe and North America vs. 58\% in Asia) \\

	\textbf{"historically the largest polluters" - false cause/ambiguity:} It is true: The west has the highest cumulative CO\textsubscript{2} emissions with 51\% but they have also had more than 250 years to achieve that, which is currently largely irrelevant. Using 'historically' in the present tense can also carry the implication that they still are the largest polluters every year, which would give the argument a modicum merit. \\

	\textbf{"enjoyed the afternoon from my porch" - appeal to emotion/false equivalence:} This \textit{could} be you. You could be enjoying the afternoon, instead of being stressed out all the time. This directly implies that adopting this opinion will make it more likely to lead a similar lifestyle. \\

	\textbf{"I could already feel the difference" - anecdotal:} Personal experience is not an effective way to gauge a general trend but people might still consider it one. 'They felt like that too? It wasn't too warm here yesterday either. That proves it'

	\newpage

	\section*{Fallacies}
	\textbf{Loaded questions:} Loaded questions are questions with an assumption built into it already. The potential answer is only of minor importance. More important is the implication of the question itself, as it implants an idea in one's head which they might not have had before. \\

	\textbf{Ad Hominem:} Literally '\textit{to the person}' is an argument which targets a person's characteristics or attributes in an attempt to make them seem unfit to be making this kind of argument, instead of addressing the argument itself. A common \textit{ad hominem} exchange goes as follows: 'A makes a claim \textit{x}, B asserts that A holds a property that is unwelcome, and hence B concludes that argument \textit{x} is wrong.'\\

	\textbf{Appeal to Authority:} With an appeal to authority it is the position the person is holding which makes their claims true, not the arguments of the claim. In an appeal to authority one can also refernce themselves, claiming their position makes the claim true, while perhaps not even providing arguments.\\

	\textbf{Appeal to Emotion:} In an appeal to emotion, one tries to distract from the argument of the claim itself, instead trying to elicit strong emotions, which serve to distract from coming to a rational conclusion. Appealed emotions can, for example, be \textit{fear} or \textit{guilt} but also \textit{pride} or \textit{compassion}. This fallacy only applies if the emotion to be invoked is irrelevant to the argument.\\

	\textbf{False Cause:} A false cause is established, when a potential relationship between two things is interpreted as one causing the other. An example is the global decrease in piracy, which happened simultaneously as the increase in temperature, so one could erroneously claim that piracy and global warming are directly inversely proportional to each other.\\

	\textbf{Ambiguity:} A fallace of ambiguity is an intentionally unclear statement in order to convey a message, which, if challenged, can be deflected to actually be meaning something else. Common applications include literal interpretations of phrasemes like 'kicking somebody out' meaning to actually physically assaulting them.\\

	\textbf{Anecdote:} Anecdotes are personal testimonies and limited to personal perception. This can often be a much easier point to argue for as there is no continuum of experiences, which makes a clear statement difficult. For example: 'I have smoked 20 cigarettes per day for the last 30 years and I am fine. Thus smoking does not cause cancer.'\\
	
\end{document}

Ad Hominem | 183 - 65 - 64 | Teal
Loaded Question | 273 - 100 - 100 | Purple
Appeal to Authority | 258 - 65 - 64 | Red
Appeal to Emotion | 58 - 65 - 64 | Yellow
False Cause | 104 - 65 - 64 | Green
Ambiguity | 237 - 100 - 100 | Blue
Anecdote | 305 - 100 - 100 | Pink