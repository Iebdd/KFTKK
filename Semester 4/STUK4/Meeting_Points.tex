\documentclass{article}

\usepackage{geometry}
\usepackage{makecell}
\usepackage{array}
\usepackage{multicol}
\usepackage{setspace}
\usepackage{changepage}
\usepackage{booktabs}
\usepackage{graphicx}
\usepackage{float}
\newcolumntype{?}{!{\vrule width 1pt}}
\renewcommand\theadalign{tl}
\setstretch{1.10}
\setlength{\parindent}{0pt}

%\geometry{top=12mm, left=1cm, right=2cm}
\title{\vspace{-3cm}21S 520.230 Deutsch: Mutter-/Bildungssprache: Textanalyse und Textproduktion Gruppe 3}
\author{Andreas Hofer}

\begin{document}
	\section*{Brainstorming}
	\textbf{Pro:}
	\begin{itemize}
		\item{Teach them more professional programs and post-editing techniques to achieve higher quality translation}
		\begin{itemize}
			\item{MT is improving exponentially, as well as expanding to include more languages and specific fields of translation}
			\item{Students already use programs like DeepL to 'improve on their work' - we strive to teach them the right way to do it}
		\end{itemize}
		\item{It is the future}
		\begin{itemize}
			\item{Our department needs to kepp up with the development of new translation technologies}
			\item{It is upon us to prepare our students, so they don't have to seek it out themselves}
			\item{We need to prepare ourselves for future developments now, instead of when they become relevant}
		\end{itemize}
		\item{The English (or French) department can serve as a testing grounds for this new curriculum and can be expanded to other faculties if succesful}
		\begin{itemize}
			\item{If the ITAT offers a specialised module for this type of translation work we will attract more students}
			\item{Graduates better suited for the translation work of the future will also have a positive impact on the ITAT's reputation}
		\end{itemize}
		\item{Acknowledging Machine Translation in the curriculum enables us to offer a broader range of field-specific translation skills (Like law or medicine)}
		\begin{itemize}
			\item{Students don't need to acquire as much additional knowledge}
		\end{itemize}
	\end{itemize}
	\textbf{The proposal:}
	\begin{itemize}
		\item{The English (or French) department is convinced that the change in curricula will improve our students' experience, therefore we would like to offer relevant courses on a trial basis this coming semester.}
		\item{\textit{Trados}, an enterprise translation software offers courses on how to effectively implement machine translation into one's workflow. Familiarity with one of these tools would also prove beneficial in the job market.}
	\end{itemize}
	\textbf{Cons:} \\
	Potential counterarguments
	\begin{itemize}
		\item{We will just be teaching students how to cheat}
		\begin{itemize}
			\item{We need to teach them how not to plagiarise}
			\item{Ignoring this will not stop them from using MT}
			\item{We cannot control what students do with the skills we teach them regardless}
		\end{itemize}
		\item{We don't have the staff to teach these courses}
		\begin{itemize}
			\item{We would not need one lecturer per language. There already are several classes about general translation work which are held in German}
			\item{We could use Trados courses to have the lecturer acquire the skills needed to hold the lecture.}
		\end{itemize}
		\item{It is Expensive}
		\begin{itemize}
			\item{If it is succesful the cost of the programme will easily be offset by the increased competetiveness of our students. If we only start with a trial run the failure is even a relatively cheap one.}
		\end{itemize}
	\end{itemize}
	
\end{document}