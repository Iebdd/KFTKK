\documentclass{article}

\usepackage{geometry}
\usepackage{makecell}
\usepackage{array}
\usepackage{multicol}
\usepackage{setspace}
\usepackage{changepage}
\usepackage{booktabs}
\newcolumntype{?}{!{\vrule width 1pt}}
\renewcommand\theadalign{tl}
\setstretch{1.10}
\setlength{\parindent}{0pt}

\geometry{top=12mm, left=1cm, right=2cm}
\title{22W 520.044 Wissenschaftliches Schreiben}
\author{Andreas Hofer}

\begin{document}
	\section{Wissenschaftliches Schreiben}
	Zwei zu schreibende Texte

	\begin{itemize}
		\item{Zusammenfassung anhand von bereitgestellten Texten}
		\item{Wissenschaftlicher Text anhand von 3 oder mehr Texten}
	\end{itemize}

	\section{Steinhoff - Fachsprachlichkeit - 31.03.2022}
	Die Gegenstandsbindung ist ein wichtiges, eventuell das wichtigste Kriterium der Fachsprachlichkeit innerhalb des Forschungsbereiches der Fachsprachenforschung. \\
	Steinhoff nimmt innerhalb seines Textes auch auf Bühlers Organonmodell Bezug und nennt die Darstellungsfunktion als dessen zentralen Standpunkt. Dies geschieht weil das wesentliche eines fachsprachlichen Textes die Nomen sowie Nomenbildungen sind. Da Nomen etwas abbilden und Gegenstände beschreiben, verbindet das die Darstellungsfunktion und den Gegenstandsbezug der Fachsprachlichkeit. \\
	Ohne den Inhalt der zitierten Textteile zu kritisieren, weist er auch auf eine Lücke innerhalb dieser Definitionen hin.\\
	\subsection{Referieren von Texten}
	Bei der Referierung von Texten, falls man explizit klarstellen will, dass man auf die Arbeit einer Person beschreibt, kann man ein kurzes direktes Zitat einbauen, mit welchem diese Position klargestellt wird. \\
	Es ist Vorsicht geboten, welche Begriffe mit einem spezifischen Artikel versehen werden. \textit{Die Äquivalenz} ist beispielsweise mit Vorsicht zu verwenden, da es viele verschiedene Äquivalenzen, z.B. nach Koller, Vermeer etc gibt. Stattdessen ist es besser den \textit{Äquivalenzbegriff nach \[\]} zu verwenden.\\

	\subsection{Translation als zentrale Nebensache in einer globalisierten Welt – eine Einführung}
	Innerhalb der Publikationen gibt es mehrere verschieden Arten wie: Monografien, Sammelbänder sowie Publikationen in wissenschaftlichen Zeitschriften. Monografien bedeuten jedoch nicht, dass es nur von einer einzigen Person geschrieben worden ist, sondern, dass es einen Text über ein einzelnes Thema enthält. In einem Sammelband hingegen müssen verschieden Artikel innerhalb der Publikation nicht unbedingt etwas miteinander zu tun haben. Sammelbänder sind thematisch auch meist offener gehalten.\\
	Innerhalb eines Sammelbandes gibt es meistens eine Einleitung, in welcher die inhaltlichen Teile besprochen werden. \\
	Das besprochene Kapitel dient in dem Sammelband \textit{Berufsziel Übersetzen und Dolmetschen. Grundlagen, Ausbildung, Arbeitsfelder} um dieses Thema einzuleiten.

	\section{Wissenschaftliches Schreiben - 07.04.2022}

	Letzte Woche kommentierte jemand, dass bei den besprochenen Texten nahezu nur Kritik geäußert wurde und, dass der Eindruck enstünde, als ob der Text keine positiven Qualitäten hat. Zu diesem Thema unterstreicht Hr. Hebenstreit, dass man sich bei der Verbesserung eines Textes nur verbessern kann, wenn man die Fehler kritisch behandelt. \\
	\subsection{Exzerpte}
	Nur die ersten acht Seiten sind für die Aufgabenstellung relevant.
	\subsection{Zitieren}
	Es gibt mehrere Formen von Zitaten:
	\begin{itemize}
		\item{Wörtliche Zitate: Der Satz wird wörtwörtlich von dem übernommen, was jemand geschrieben hat.}
		\item{Sinngemäße Zitate: Man nimmt lediglich die Idee und formuliert seinen eigenen Satz der diesen beschreibt.}
	\end{itemize}
	Es gibt auch Sekundärzitate, in welchem man das Zitat, welches von einer anderen Person zitiert wird, selbst als Zitat verwendet. Solche Sekundärzitate sollten, falls möglich, vermieden werden, da es von großer Bedeutung ist die verwendeten Quellen verwendet zu haben. Sekundärzitate müssen als solche angegeben werden. \\
	Weiters kann man Zitate in mehreren Wegen erkenntlich machen:
	\begin{itemize}
		\item{Im Fließtext: Bei dem Zitieren einer Quelle schreibt man diesen am Ende des Satzes in eine Klammer. Z.B. (Hönig 1963:322)}
		\item{Als Fußnote: Quelle wird}
	\end{itemize}
	Zitate sollten grundsätzlich eine gewisse Zweckmäßigkeit besitzen. In der Argumentation innerhalb des Textes muss man das direkte Zitat verwenden und diesen nicht nur verwenden um den Text zu verlängern. \\
	Weiters sollte man, wenn man Informationen wiedergibt, sicherstellen, dass man wissenschaftliche Quellen verwendet und diese eine grundsätzliche Seriösität besitzen. \\
	Man sollte nur in Ausnahmefällen aus Einführungen oder Handbüchern und Nachschlagewerken zitieren. In solchen Publikationen wird meist nicht selbst Stellung genommen und stattdessen darüber geredet, was jemand anders gesagt hat. So sind diese meist nur Sekundärzitate. \\
	Bachelorarbeiten und Seminararbeiten sind auch keine zitablen Quellen. \\
	Verwendete Quellen müssen korrekt zitiert werden. Ein Zitat muss stets sinngemäß sein und darf nicht wörtlich übernommen werden. Wörter in der Satzfolge zu verändern ist kein sinngemäßes Zitieren. Quellen müssen stets in der Bibliographie angegeben werden und man sollte möglichst alle Quellen aus der Bibliographie verwenden. Ersteres ist jedoch bedeutend gravierender als Zweiteres. Nur allgemein bekannte Sachverhalte muss man nicht zitieren. \\
	\subsection{Form des Zitats:}
	Die Quellenangabe im Fließtext kommt grundsätzlich immer vor dem schließenden Satzzeichen. Es gibt Ausnahmen bei wörtlichen Zitaten.
	\begin{itemize}
		\item{Grundform: (Name Jahr:Seitenangabe) \textit{(Shreve 2002:256)}}
		\item{Zwei aufeinanderfolgende Seiten: \textit{(Shreve 2002:87f)} (folgende)}
		\item{Mehrere aufeinanderfolgende Seiten: \textit{(Shreve 2002:87ff)} (fortfolgend)}
		\item{Wenn man mehrere Zitate aus der gleichen Quelle nimmt: \textit{(ibid.:264)}}
		\item{Wenn man mehrere Werke des selben Jahres zitiert: \textit{(Vermeer 1986a:78) (Vermeer 1986b:32)}}
	\end{itemize}
	Multiple Autoren:
	\begin{itemize}
		\item{1 Autor: \textit{(Prun\v c  2012:74)}}
		\item{Mehrere Autoren: \textit{(Sirén/Hakkarainen 2002:74)}}
		\item{Mehr als 4 Autoren: \textit{(Kaguera et al. 2011:51)}}
	\end{itemize}

	Wenn man aus dem Internet zitiert gelten die gleichen Regeln. Falls keine Seitenangabe besteht, kann man eine erstellen. URLs von Websites nicht in den Fließtext oder der Fußnote einbauen, nur in die Bibliographie. \\
	Zitate aus zweiter Hand sollen \textbf{nur in Ausnahmefällen} geschehen. Ein Beispiel ist, wenn das Original schwer oder nicht mehr zugänglich ist. In der Regel gilt, dass man Sekundärzitate weniger verwenden sollte, je wichtiger das Zitat für die Arbeit ist. Falls man ein Sekundärzitat verwendet: \textit{(Huber 1932:34 zit. n. Mayer 1958:235)}

	\subsection{Sinngemäße Zitate}
	Sinngemäße Zitate sind nicht wörtlich zitiert und paraphrasieren das Werk einer anderen Person. Wenn das Zitat zu wörtlich ist, und man dieses nicht mit Anführungszeichen kennzeichnet, ist das eine Form des Plagiats. Man soll fremdsprachliche Quellen nicht einfach nur wörtlich übersetzen. \\
	\subsection{Bezug auf gesamte Publikationen}
	Wenn man gesamte Publikationen beschreibt wird die Seitenangabe ausgelassen: (Kade 1963) Eine Quellenangabe muss so genau wie möglich angegeben werden. Man kann nicht eine Reichweite des Zitats angeben z.B. (Kade 1963:5-87).\\
	Wörtliche Zitate können, falls sie kürzer als zwei Sätze sind, in Anführungszeichen im Fließtext eingebaut werden. Quellenangaben sollen unmittelbar nach dem Zitat eingefügt werden. \\
	Bei längeren wörtlichen Zitate, welche mehr als zwei Zeilen lang sind, muss man absetzen und einrücken. Die Schriftgröße soll ein pt kleiner sein als der Text. Die Quellenangabe befindet sich nach dem schließenden Satzzeichen. Es gibt keine Anführungszeichen. \\
	\subsection{Wörtlich Zitieren}
	Wenn man wörtliche Zitate ändern will, muss man dies markieren. 
	\begin{itemize}
		\item{Wenn man einen Teil des Satzes überspringen will, muss man [...] einsetzen, was jedoch keine drei Punkte sein dürfen, sondern ein spezifisches Satzzeichen sind [\dots].}
		\item{Wenn man Sachen hinzufügen will, kann man eine Erläuterung wieder in eckige Klammern setzen: "Die TLW [Translationswissenschaften]"}
		\item{Bei Explikationen, also wenn man implizite Information explizit machen will, muss man das auch in eckige Klammer setzen.}
	\end{itemize}

	Bei wörtlichen Zitaten geschehen manchmal auch im Original Fehler. Falls man solch eine Passage wörtlich zitiert sollte man diesen nicht korrigieren, sondern stattdessen [sic] direkt nach dem Fehler einzufügen um klarzustellen, dass man es nicht fehlerhaft kopiert hat. Alte Rechtschreibung sollte nicht korrigiert werden. \\


	Bei einem Sammelband werden die Herausgeber mit einem (eds) gekennzeichnet.

	\section{Wissenschaftlich formulieren - 28.04.2022}
	\textit{Kann man Wissen vorraussetzen?}
	Im Grunde schon. Wissenschaftliche Texte richten sich meist an Personen, welche bereits mit der Materie vertraut sind und grundlegende Fachbegriffe vorraussetzen kann. Selbst wenn man will, kann es jedoch passieren, dass man nicht immer bei 0 beginnen kann, da einfach die Zeit fehlt. \\
	\textbf{Bei PSI und PSII muss mehr erklärt werden, als es bei wissenschaftlichen Texten normal ist, da man vermitteln soll, dass man die Inhalte verstanden hat.}
	\begin{center}
	\_\_\_\_\_\_\_\_\_\_\_\_\_\_\_\_\_\_\_\_\_\_\_\_\_\_\_\_\_\_\_
	\end{center}
	Jedoch bedeutet, dass wissenschaftlich formulieren, nicht bedeutet, dass etwas kompliziert formuliert sein muss. Das Hauptaugenmerk sollte auf der Eindeutigkeit und Klarheit liegen. Dies wird vor allem durch:
	\begin{itemize}
		\item{Sachbezogenheit und Objektivität}
		\item{Präzision, Eindeutigkeit und Korrektheit}
		\item{Kürze und Prägnanz}
	\end{itemize}
	erreicht. \\
	\subsubsection{Sachbezogenheit und Objektivität}
	Laut Steinhoff zeichnet sich ein wissenschaftlicher Text Teils aus seiner Gegenstandsbindung aus. So haben wissenschaftliche Texte oft einen Fokus auf Nomen und nominale Strukturen und verwenden somit Strukturen wie beispielsweise "einer Frage nachgehen" oder "einen Aspekt anführen". \\
	Doch was heißt Sachbezoegenheit und Objektivität? \\
	\begin{itemize}
		\item{Fragestellungen werden durch Titel und Kapitelüberschriften kenntlich gemacht}
		\item{Ein Textabschnitt enthält genau das, was für das jeweilige Kapitel relevant ist.}
		\item{Erkenntnise müssen jederzeit nachprüfbar sein und sollen keine unbelegbaren Vermutungen oder Bewertungen sein}
	\end{itemize}
	\subsubsection{Präzision, Eindeutigkeit und Korrektheit}
	Sachliche Richtigkeit kann erreicht werden durch:
	\begin{itemize}
		\item{Fachbegriffe, wobei die zentralen Begriffe einer Arbeit definiert sein sollten}
		\item{Formulierungen sollten exakt und schnell erfassbar sein}
		\item{Korrektiere Zitierungen. Kenntlichmachung von Übernahme aus anderen Texten}
		\item{Eine vollständige Quellenangabe}
	\end{itemize}
	Fachwörter sind das wichtigste Mittel zur Erreichung höchstmöglicher Präzision, man sollte jedoch sicherstellen, dass man den richtigen Fachbegriff im richtigen Kontext verstehen und verwenden kann. Oft haben unterschiedliche Disziplinen verschiedene Auffassungen des selben Fachbegriffs. Das gleiche gilt auch für Fremdwörter, welche oft falsch verwendet werden. \\
	\subsubsection{Kürze und Prägnanz}
	Kürze bedeutet nicht so wenig Text wie möglich zu produzieren, sondern \textit{so ausführlich wie nötig, so prägnant wie möglich} zu sein. Es sollte möglich sein den Text schnell aufzufassen.\\
	\subsection{Weiteres}
	Angaben sollten so genau wie möglich sein. Es sollten keine Angaben wie "viele", "einige" oder "manche" gemacht werden, wenn diese nicht an einem anderen Ort zusätzlich erläutert werden. \\
	Auch sollte man es vermeiden Tautologien anzuwenden die zu einer Zirkularlogik führen. Ein Diensteleistungsunternehmen das "kundenorientiert" ist, sagt nicht mehr aus wie ein Satz ohne die Phrase, da man annimmt, dass ein Unternehmen kundenorientiert ist. \\
	Man sollte es vermeiden subjektive Aussagen zu machen. Wertungen, Empfindungen oder EInschätzungen sollten, falls der Gedankenweg nicht nachvollziehbar ist, vermieden werden. \\
	Worte sollten keine impliziten Wertungen enthalten. Implizite konnotationen in Wörtern die beschönigen, verharmlosen oder abwerten sollten vermieden werden.

	\section{Extra Einheit - 19.05.2022}

	Mit 19.05.2022 sind noch nicht alle Arbeiten verbessert. Er hat zur Entschuldigung einen Rharbarberkuchen mitgebracht. Die zwei Ursachen sind laut ihm:
	\begin{itemize}
		\item{Am Anfang der Lehrveranstaltung wurde gesagt, dass der erste Text eine Zusammenfassung sein soll, was jedoch im Endeffekt nicht die Aufgabenstellung des Textes war. Während in einer Zusammenfassung alle wichtigsten Punkte erwähnt werden sollten, gab die Aufgabenstellung die Aufgabe, die drei erwähnten Themen anhand des Ausgangstextes zu beschreiben. Deshalb sollte laut der Aufgabenstellung ein eigenständiger Text, anhand des Ausgangstextes erstellt werden. So benötigt er auch eine Einleitung sowie einen allgemeinen Schluss. Der Großteil der bisher gelesenen Texte versucht jedoch alle Themen aus dem Text innerhalb der 1,5 Seiten unterzubringen. Es gibt, und ich zitiere, "Aus jedem Dorf ein Hund".}
		\item{Zusätzlich ist er der Ansicht, dass es sehr oft scheint, als ob Abgaben Themen ansprechen, welche nicht wirklich verstanden wurden, wodurch Themen 1:1 übernommen werden, ohne dass dessen Sinn auch übertragen wird und dadurch oft nicht in den Text passt. Es scheint auch oft, als ob manche Sätze zwar grammatikalisch komplex formuliert sind, dabei jedoch die Sinnhaftigkeit des Satzes selbst oft nicht gegeben ist.}
	\end{itemize}
	Viele der Texte haben sehr viel Information über theoretische Konzepte wie Theorien von Personen von Kade bis Holz-Mänttäri, haben jedoch oft das Thema der Verbindung zwischen Theorie und Praxis nur nebensächlich erwähnt. Da es keine Zusammenfassung ist, sondern ein eigenständiger Text, dessen Information auf diesem Text basiert, kann man die Struktur des Textes auch anpassen wie man will und muss nicht zuerst von der Praxis sprechen, nur weil der Text das auch macht. \\
	Er schlägt vor, dass manche Personen den ersten Text überarbeiten, bzw. sogar neu schreiben, anstatt die zweite Abgabe und den zweiten Text zu behandeln. Im zweiten Text seien die Themen, welche in der Aufgabenstellung erwartet werden bei weitem nicht mehr so offen dargelegt und wenn man bereits bei dem ersten Text Schwierigkeiten hatte, dann hat man beim zweiten noch größere Probleme. \\
	Es ist wichtig, wenn man einen Text analysieren will, dass das Ausgangsmaterial anhand der eigenen Bedürfnisse angepasst wird. Nach einer Umfrage in der Klasse, gibt es einige Leute, welche Texte ausdrucken und diese dann anstreichen bzw. exzerpieren. Nur markieren ist jedoch oft nicht genug. Wenn man lediglich Textteile markiert und diese dann verwendet dann behandelt man den Text selbst nicht wirklich. Anscheinend hatte Umberto Eco in seinem Buch, aus welchem wir eh einen Abschnitt gesehen haben, erwähnt, dass man nicht sinnlos Seiten und Absätze kopieren soll, da man sein eigenes Verständnis des Textes erstellen soll, anstatt sich nur auf die Information anderer zu stützen. Ein guter Weg um Informationen selbst zu erschließen sei die Erstellung eigener Kategorien und Titel, welche von kurzen Titeln bis zu beschreibenden Sätzen gehen können. \\
	Ein Werkzeug, welches Hr. Hebenstreit verwendet ist \textit{MindManager}, welches auf der Universitätswebsite vorhanden sein soll. Jedoch kann man natürlich irgendein Werkzeug verwenden und die Verwendung einer mind map ist das ausschlaggebende. Eine mind map kann einem helfen zu visualisieren, wie Information kategorisiert wird und zusammenhängt. Noch besser soll die händische Zeichnung einer mind map sein, da man dann die mechanische Arbeit auch mit der mind map verbindet. Jedoch muss man dann, wenn man etwas verändern will oft eine neue zeichnen. \\
	


	

	














\end{document}