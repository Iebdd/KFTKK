\documentclass{article}

\usepackage{geometry}
\usepackage{makecell}
\usepackage{array}
\usepackage{multicol}
\usepackage{setspace}
\usepackage{changepage}
\usepackage{booktabs}
\newcolumntype{?}{!{\vrule width 1pt}}
\renewcommand\theadalign{tl}
\setstretch{1.10}
\setlength{\parindent}{0pt}

\geometry{top=12mm, left=1cm, right=2cm}
\title{22W 520.044 Wissenschaftliches Schreiben}
\author{Andreas Hofer}

\begin{document}
	\section{Wissenschaftliches Schreiben}
	Zwei zu schreibende Texte

	\begin{itemize}
		\item{Zusammenfassung anhand von bereitgestellten Texten}
		\item{Wissenschaftlicher Text anhand von 3 oder mehr Texten}
	\end{itemize}

	\section{Steinhoff - Fachsprachlichkeit - 31.03.2022}
	Die Gegenstandsbindung ist ein wichtiges, eventuell das wichtigste Kriterium der Fachsprachlichkeit innerhalb des Forschungsbereiches der Fachsprachenforschung. \\
	Steinhoff nimmt innerhalb seines Textes auch auf Bühlers Organonmodell Bezug und nennt die Darstellungsfunktion als dessen zentralen Standpunkt. Dies geschieht weil das wesentliche eines fachsprachlichen Textes die Nomen sowie Nomenbildungen sind. Da Nomen etwas abbilden und Gegenstände beschreiben, verbindet das die Darstellungsfunktion und den Gegenstandsbezug der Fachsprachlichkeit. \\
	Ohne den Inhalt der zitierten Textteile zu kritisieren, weist er auch auf eine Lücke innerhalb dieser Definitionen hin.\\
	\subsection{Referieren von Texten}
	Bei der Referierung von Texten, falls man explizit klarstellen will, dass man auf die Arbeit einer Person beschreibt, kann man ein kurzes direktes Zitat einbauen, mit welchem diese Position klargestellt wird. \\
	Es ist Vorsicht geboten, welche Begriffe mit einem spezifischen Artikel versehen werden. \textit{Die Äquivalenz} ist beispielsweise mit Vorsicht zu verwenden, da es viele verschiedene Äquivalenzen, z.B. nach Koller, Vermeer etc gibt. Stattdessen ist es besser den \textit{Äquivalenzbegriff nach \[\]} zu verwenden.\\

	\subsection{Translation als zentrale Nebensache in einer globalisierten Welt – eine Einführung}
	Innerhalb der Publikationen gibt es mehrere verschieden Arten wie: Monografien, Sammelbänder sowie Publikationen in wissenschaftlichen Zeitschriften. Monografien bedeuten jedoch nicht, dass es nur von einer einzigen Person geschrieben worden ist, sondern, dass es einen Text über ein einzelnes Thema enthält. In einem Sammelband hingegen müssen verschieden Artikel innerhalb der Publikation nicht unbedingt etwas miteinander zu tun haben. Sammelbänder sind thematisch auch meist offener gehalten.\\
	Innerhalb eines Sammelbandes gibt es meistens eine Einleitung, in welcher die inhaltlichen Teile besprochen werden. \\
	Das besprochene Kapitel dient in dem Sammelband \textit{Berufsziel Übersetzen und Dolmetschen. Grundlagen, Ausbildung, Arbeitsfelder} um dieses Thema einzuleiten.

	














\end{document}