\documentclass{article}

\usepackage{geometry}
\usepackage{makecell}
\usepackage{array}
\usepackage{multicol}
\usepackage{accents}
\usepackage{setspace}
\usepackage{changepage}
\usepackage{booktabs}
\usepackage{graphicx}
\usepackage{float}
\usepackage{fancyhdr}
\usepackage[ngerman]{babel}
\newcolumntype{?}{!{\vrule width 1pt}}
\renewcommand\theadalign{tl}
\setstretch{1.30}
\setlength{\parindent}{0pt}
\setlength{\headheight}{22.76004pt}

\pagestyle{fancy}
\fancyhead[L]{\today}
\fancyhead[C]{22S Wissenschaftliches Schreiben}
\fancyhead[R]{Andreas Hofer{\scriptsize(11705024)}}

%\geometry{top=12mm, left=1cm, right=2cm}
\title{\vspace{0cm}22S 520.044 Allgemein: Wissenschaftliches Schreiben KS \\ Textproduktion 1}
\author{Andreas Hofer}

\begin{document}
	\section*{Textproduktion 1}
	In dem einführenden Kapitel \textit{ Translation als zentrale Nebensache in einer globalisierten Welt - eine Einführung} ihres Sammelbandes von Mira Kadrić und Klaus Kaindl, behandeln die Autoren die grundlegende Bedeutung des Fachbereichs der Übersetzung und des Dolmetschens in einer globalen Welt und gehen auf vergangene Entwicklungen ein, welche zur heutigen Ausgangssituation geführt haben. Sie stellen fest, dass Übersetzungen sowie Dolmetschen ein allgegenwärtiges Phänomen sind, welches von Handbüchern, über Nachrichten und Interviews bis zur elementaren Funktionsweise der Europäischen Union reichen, da der Austausch so viele unterschiedlicher Länder sonst nie möglich wäre (Kadrić/Kaindl 2016:1f). \\

	Nach Ansicht von Kadrić und Kaindl ist es auf die Festlegung konkreter Berufsbilder zurückzuführen, dass Studiengänge im Bereich der Translation oft Praxis gegenüber der Theorie bevorzugen und bemerken, dass die lange andauernde angespannte Beziehung dieser Richtungen historisch bedingt ist. Eine Entspannung führen sie zum Teil auf die kürzliche Erkenntnis zurück, dass wissenschaftliche Theorien einen Vorteil gegenüber rein empirischer Praktiken bieten können. Gleichzeitig weisen die Autoren auch darauf hin, dass die Translationswissenschaft sich über deren Geschichte kontinuierlich der Praxis angenähert und somit ebenfalls zu einer verträglicheren Haltung geführt hat. Die frühen Vertreter der Tätigkeit der Übersetzung hätten hierbei zur Ableitung der Grundsätze stets auf ihre eigene praktische Erfahrung zurückgegriffen; Ein Trend, welcher sich bis zum Ende des zweiten Weltkrieges fortgeführt habe (Kadrić/Kaindl 2016:3f). \\

	Somit begann man laut Kadrić/Kaindl erst mit der Formalisierung der Übersetzungswissenschaft, als nicht die Sprachverwendung selbst, sondern der Regelapparat einer Sprache zur maschinellen Übersetzung benötigt wurde. Gleichzeitig bemerken sie jedoch, dass der menschliche Aspekt nur in Betracht gezogen wurde, als ersichtlich geworden war, dass zur maschinellen Übersetzung mehr als die Regeln einer Sprache nötig sind. Ein laut Kadrić und Kaindl von Prun\v c als \glqq Paradigmenwechsel\grqq \:bezeichnete Entwicklung der Translationwissenschaft war schließlich die \glqq Neuorientierung\grqq \:von Mary Snell-Hornby, dessen funktionale Sichtweise auf Translation sowohl kulturelle, als auch sprachliche Aspekte bei der Translation miteinbezogen. So habe sich, zusätzlich zur Verschiebung des Fokus vom Ausgangstext auf den Zweck des Zieltextes, auch die Stellung des Translators oder der Translatorin verändert, da diese/r nun eine entscheidendere Rolle in der Erarbeitung der Theorien einnahm (Kadrić/Kaindl 2016:4f). \\ \\

	Nach dieser Entwicklung begann man nach Kadrić/Kaindl anstatt Sprachen, Texte oder Kulturen die Person selbst in den Mittelpunkt zu stellen und zu analysieren welche Einstellungen und Positionen diese einnehmen, sowie in welcher Weise Emotionen oder Einstellungen den Prozess beinflussen. Ebenfalls habe sich gleichzeitig ein separater Forschungszweig zur soziologischen Ansicht entwickelt, welcher Fragen über den Status eines Translators oder einer Translatorin, der Berufsorganisation oder den Arbeitsbedingungen stellt. Durch diesen vermehrten Fokus habe Chesterman laut Kadrić und Kaindl diese Entwicklung als \glqq Translator Studies\grqq \:bezeichnet (Kadrić/Kaindl 2016:5). \\

	Im Kontext der translatorischen Kompetenz, wurde diese nach Kadrić/Kaindl lange im besten Fall autodidaktisch erworben, wobei bis in das 20. Jahrhundert keine dezidierten Ausbildungsstätten existierten. (Kadrić/Kaindl 2016:5). Kadrić und Kaindl bemerken, dass zusätzlich zur Erlangung einer Gesamtkompetenz, Studierende ebenfalls eine Rolle in der Gesellschaft besitzen und sich dieser auch bewusst werden sollten. Ihrer Ansicht nach sind qualitativ hochwertige Translationen integral für Rechtsstaat, Sozialstaat und Demokratie wie beispielsweise Dolmetschleistungen in Krankenhäusern, was auch mit erhöhten gesellschaftlichen Erwartungen und einer Verantwortung einhergeht. Die Qualität eines Translats oder einer Dolmetschung könne über den Ausgang eines Straf- oder Asylverfahrens entscheiden oder zu Fehldiagnosen mit weitreichenden Folgen führen. Aus diesem Grund sehen sie es als äußerst wichtig an, bereits während des Studiums selbstständiges Handeln zu lernen und selbstverantwortlich zu agieren. (ibid:7).

	\newpage
	\section*{Literaturverzeichnis}
	Kadrić, Mira/Kaindl, Klaus (2016) \textit{„Translation als zentrale Nebensache in einer globalisierten Welt – eine Einführung“}, in: Kadrić, Mira/Kaindl, Klaus (eds.) \textit{Berufsziel Übersetzen und Dolmetschen: Grundlagen, Ausbildung, Arbeitsfelder}. Tübingen: A. Francke Verlag, 1-8. 
\end{document}