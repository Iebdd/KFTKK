\documentclass{article}

\usepackage{geometry}
\usepackage{makecell}
\usepackage{array}
\usepackage{multicol}
\usepackage[ngerman]{babel}
\usepackage[ngerman=ngerman-x-latest]{hyphsubst}
\usepackage{setspace}
\usepackage{csquotes}
\usepackage{changepage}
\usepackage{booktabs}
\usepackage{graphicx}
\usepackage{float}
\newcolumntype{?}{!{\vrule width 1pt}}
\renewcommand\theadalign{tl}
\setstretch{1.10}
\setlength{\parindent}{0pt}

\geometry{top=12mm, left=1cm, right=2cm}
\title{\vspace{-3cm}21S 520.230 Deutsch: Mutter-/Bildungssprache: Textanalyse und Textproduktion Gruppe 3}
\author{Andreas Hofer}

\begin{document}
	\section*{Tätigkeitsbereiche Selbstständiger TranslatorInnen \\ {\footnotesize Sabine Steinlechner}}
	{\large Gibt es heute noch den/die \glqq klassische/n\grqq \: ÜbersetzerIn bzw. DolmetscherIn? Oder haben sich die Anforderungen und somit die Tätigkeitsbereiche von TranslatorInnen geändert? Wie passen sich selbstständige ÜbersetzerInnen und DolmetscherInnen an diese Situation an? - Eine Analyse des österreichischen Marktes im Rahmen einer Masterarbeit hat sich mit genau diesen Fragen beschäftigt.} \\ \\
	\textbf{Seite 7} \\
	Ich arbeite seit einigen Jahren als selbstständige Translatorin in Österreich und habe dabei festgestellt, dass der Beruf der Translatorin viel mehr als \glqq nur\grqq \: übersetzen und dolmetschen beinhaltet. Meine persönlichen Erfahrungen haben gezeigt, dass die Globalisierung, internationale Geschäftsbeziehungen von AuftraggeberInnen und der technologische Fortschritt die Aufgabenbereiche von ÜbersetzerInnen und DolmetscherInnen stark verändert haben und auch weiterhin verändern werden. Aus diesem Grund wollte ich im Rahmen meiner Masterarbeit herausfinden, welche Tätigkeiten selbstständige TranslatorInnen in Österreich heutzutage ausüben, ob sie ausschließlich als ÜBersetzerInnen oder DolmetscherInnen arbeiten oder die Tätigkeiten kombinieren. Außerdem sollte die Frage beantwortet werden, in welchen \textit{neuen} und/oder \textit{verwandten} Berufsfeldern, wie z.B. Technischer Dokumentation und Redaktion, Lektorat und Revision, Terminologiemanagement, Lehre usw., sie noch tätig sind und wie \textit{eng verwandt} sie diese mit ihren Kerntätigkeiten sehen. \\ \\
	\subsection*{Spiegeln Leitfäden und Ratgeber von Berufsverbänden die Praxis in Österreich wider?}
	Zum Beatnworten dieser Fragen könnte man einen Blick in diverse Leitfäden für AbsolventInnen der TLW, Informationen für BerufseinsteigerInnen oder Ratgeber von Berufsverbänden werfen, in denen eine Vielzahl von Berufsbildern und Tätigkeitsbereichen für TranslatorInnen beschrieben wird. Aber spiegeln diese Quellen die gegenwärtige Praxis in Österreich tatsächlich wider?
	\subsection*{Analyse der Fachlehrbücher und Leitfäden}
	Die durchgeführte Analyse der Fachlehrbücher und Leitfäden hat gezeigt, dass die Tätigkeitsbereiche für selbstständige TranslatorInnen nicht ausschließlich auf Übersetzen und Dolmetschen begrenzt sind. Dennoch ist der Anteil an Informationen zu den \textit{traditionellen} Berufsfeldern, die in \textbf{Übersetzen} (Fachübersetzen, Urkundenübersetzen und allgemeines Übersetzen) und \textbf{Dolmetschen} (Konferenzdolmetschen, Gerichtsdolmetschen, Community Interpreting, Verhandlungsdolmetschen, diplomatisches Dolmetschen sowie Messe- und Begleitdolmetschen) unterteilt werden, am größten. Auffallend ist dabei, dass die einzelnen Tätigkeitsbereiche meist getrennt voneinander beschrieben werden und dadurch der Eindruck ensteht, dass ÜbersetzerInnen und DolmetscherInnen jeweils nur einer einzigen Tätigkeit nachgehen. Die Ausnahme bilden dabei Gerichtsdolmetschen und Urkundenübersetzen. In diesem Zusammenhang wird stets darauf verwiesen, dass TranslatorInnen, die vereidigt sind, beide Tätigkeiten ausüben. Eine weitere Besonderheit ist, dass nur in diesem Bereich das \textit{Vom-Blatt-Dolmetschen} bzw. \textit{-Übersetzen} erwähnt wird. \\
	\textbf{Seite 8} \\
	Den \textit{traditionellen} Berufsfeldern stehen die \textit{verwandten} Tätigkeitsbereiche, wie Terminologiemanagement, Lektorat und Revision, Texten, Sprachunterricht in der Erwachsenenebildung, Lehre im Bereich Fremdsprachen sowie TLW und Projektmanagement, gegenüber. Dabei waren lediglich in den Leitfäden über allgemeine Sprachberufe zahlreiche Informationen dazu zu finden. \\ \\
	Bei der Beschreibung der \textit{neuen} Berufsfelder fällt auf, dass v.a. der technologische Fortschritt eine wichtige Rolle spielt, denn ohne moderne Tehcnik wäre keines der genannten Tätigkeitsfelder, wie technische Dokumentation bzw. Redaktion, Lokalisierung, Mediendolemtschen und Medienübersetzen, möglich. In den Ratgebern der Berufsverbände sind dabei kaum Informationen zu diesen neuen Tätigkeitsfeldern zu finden - der Fokus liegt hier besondern auf den \textit{tradtionellen translatorischen} Berufsfeldern. Dies ist einerseits nachvollziehbar, da eine Berufsverband in erster Lineie die \glqq Kerntätigkeiten\grqq \: von TranslatorInnen vertritt. Es wäre jedoch auch Aufgabe eines Berufsverbandes, auf neue Entwicklungen zu reagieren und Verbandsmitgliedern bzw. junge TranslatorInnen, die versuchen am freien Markt Fuß zu fassen, entsprechend zu informieren.
	\subsection*{Erhebung in Österreich}
	Im Rahmen einer Online-Umfrage wurden schließlich selbstständige TranslatorInnen (Universitas-Mitglieder) kontaktiert und gebeten, über ihre beruflichen Tätigkeiten Auskunft zu geben, um eine Bestandsaufnahme in Österreich zu erhalten. Diesem Aufruf sind insgesamt 69 TranslatorInnen gefolgt. Und ich möchte mich an dieser Stelle auch bei allen TranslatorInnen bedanken, die sich dafür Zeit genommen haben. \\ \\
	Die Erhebung zeigte in erster Linie, dass selbstständige TranslatorInnen in Österreich nicht ausschließlich in einem einzigen Tätigkeitsbereich arbeiten. Der Großteil ist sowohl als ÜbersetzerIn als auch DolmetscherIn tätig und übt zusätzlich \textit{verwandte} Tätigkeiten, wie Korrekturlesen von Übersetzungen, Lektorat, Sprachlehre, Texten usw., aus. Außerdem konnte festgestellt werden, dass fast die Hälfte aller Befragten u.a. in einem Angestelltenverhältnis arbeitet - wobei dies am häufigsten bei junden Selbstständigen der Fall ist, aber auch bei sehr erfahrenen TranslatorInnen, die seit mehr als 20 Jahren selbstständig sind. \\ \\
	Die Umfrage hat weiters gezeigt, dass die meisten TranslatorInnen Übersetzen und/oder Dolmetschen als Haupttätigkeit angaben, wobei alle DolmetscherInnen auch als ÜbersetzerInnen tätig sind. \textit{Verwandte} Tätigkeitsfelder werden entweder regelmäßig ausgeübt oder als Nebentätigkeit angeführt (siehe Abb. S.8). Im Zusammenhand mit Übersetzen gaben die meisten Befragten an, v.a. allgemeine Texte zu bearbeiten, aber auch in den Bereichen Recht, Technik sowie Marketing und Werbung zu übersetzen. Beim Dolmetschen werden dabei Konferenz-, Begleit- und Verhandlungsdolmetschen als die drei häufigsten Arbeitsschwerpunkte genannt. \\
	Die Ergebnisse lassen den Schluss zu, dass selbstständige TranslatorInnen eher mehrere unterschiedliche Tätigkeitsbereiche kombinieren, um flexibler zu sein, und um besser auf Anfragen von verschiedenen AuftraggeberInnen reagieren zu können. Diese Vermutung lässt sich damit begründen,d ass die meisten TranslatorInnen angaben, von einem Großteil ihrer AuftraggeberInnen sowohl Übersetzungs- als auch Dolmetschaufträge zu erhalten. \\ \\
	In Bezug auf die Frage, wie \textit{eng} bzw. \textit{weit verwandt} bestimmte Tätigkeiten mit dem Übersetzen bzw. Dolmetschen sind, konnte Folgendes festgestellt werden: (sieie Abb. S.9) Korrekturlesen von Übersetzungen, Lehrtätigkeiten im Rahmen \\
	\textbf{Seite 9} \\
	der Studien Übersetzen und Dolmetschen, Beratung für interkulturelle Kommunikation und die Tätigkeit von ExpertInnen für Übersetzungs- bzw. Kommunikationstechnologie (Language Technologist) wurden als \textit{(end) verwandt} eingestuft. Sprachlehre in der Erwachsenenebildung und in Schulen, Texten, Telefonüberwachung und Technische Dokumentation gehören hingegen zu den Bereichen, die von zahlreichen Befragten als \textit{wenig bis nicht verwandt} bezeichnet wurden. Dieses Ergebnis ist besonders in Bezug auf die Technische Dokumentation und die Telefonüberwachung verwunderlich, da bei diesen Tätigkeiten sowohl Fremdsprachenkenntnis als auch translationswissenschaftliches Wissen Vorraussetzung für eine erfolgreiche Arbeit sind.
	\subsection*{Vergleich: Fachliteratur \& Praxis}
	Verglichen mit den Ergebnissen der Analyse der Fachlehrbücher und Leitfäden konnten besonders bei den Angaben in Bezug auf \textit{traditionelle} Tätigkeitsbereiche die größten Unterschiede festgestellt werden: Diplomatisches Dolmetschen und Konferenzübersetzen werden bei der Onlineumfrage von keiner/m der Befragten als mögliche Tätigkeitsbereiche genannt, vor allem Letzterem wird jedoch in den Fachlehrbüchern relativ viel Aufmerksamkeit gewidmet. Auch das Urkundenübersetzen wurde bei der Onlineumfrage, im Gegensatz zu den Informationen in den Letfäden, nicht als separates Aufgabengebiet angegeben. Einen wesentlichen Unterschied gibt es beim Begeltidolmetschen: Dies wird in den analysierten Quellen zwar erwähnt, kommt in der Praxis jedoch wesentlich häufiger vor als aus den Fachlehrbüchern hervorgeht. Ähnlich verhält es sich mit den Angaben zu Übersetzen von allgemeinen Texten. Diese werden in der Fachliteratur nur bei Gouadec (2010) angeführt, die Angaben der selbstständigen TranslatorInnen zeigten jedoch, dass es sich dabei um die am häufigsten genannte Tätigkeit handelt. \\ \\
	Im Bezug auf \textit{verwandte} und \textit{neue} Tätigkeiten gibt es weniger Unterschiede, wobei die größten Abweichungen bei der Darstellung der Lehre im Rahmen der Studien Übersetzen und Dolmetschen festgestellt werden konnten. Die (translationswissenschaftliche) Lehre wird in den Fachlehrbüchern kaum oder gar nicht erwähnt, die Umfrage hat jedoch gezeigt, dass dieser Tätigkeitsbereich einen zentralen Aufgabenbereich für selbstständige TranslatorInnen darstellt. \\
	In diesem Zusammenhang sei darauf hingewiesen, dass die umfangreichsten Informationen zum Übersetzen bei Gouadec (2010) zu finden sind und dbaie sämtliche Tätigkeitsbereiche von ÜbersetzerInnen abgedeckt werden. Es fehlt jedoch eine vergleichbare Literatur zum Dolmetschen, die ebenso wichtig wäre.
	\subsection*{Schlussfolgerung}
	Zusammenfassend kann in Bezug auf die am Anfang gestellten Fragen Folgendes festgestellt werden: TranslatorInnen arbeiten meist als DolmetscherInnen UND ÜbersetzerInnen, wobei alle DolmetscherInnen auch übersetzen. Außerdem sind sie auch in \textit{nicht-translatorischen} Aufgabenbereichen, wie Lektorat, Lehre im Rahmen der Studien Übersetzen und Dolmetschen, Terminologiemanagement usw., tätig, unabhängig davon, wie lange sie bereits selbstständig sind. Des Weiteren konnte herausgefunden werden, dass es kaum eine Rolle spielt, wie \textit{eng} bzw. \textit{weit} verwandt die Tätigkeitsbereiche mit dem Übersetzen bzw. Dolmetschen sind, auch wenn tendenziell \textit{eher verwandte} Tätigkeiten mit dem Übersetzen und Dolmetschen kombiniert werden. \\ \\
	Abschließend konnte daher kein/e prototypische/r ÜbersetzerIn bzw. DolmetscherIn gezeichnet werden, da die Angaben der selbstständigen TranslatorInnen zu ihren Tätigkeitsbereichen zu unterschiedlich waren. Die Ergebnisse zeigen jedoch, dass Flexibilität und Offenheit gegenüber \textit{verwandten} bzw. \textit{neuen} Aufgabenbereichen üfr selbstständige TranslatorInnen unverzichtbar sind.

	
\end{document}