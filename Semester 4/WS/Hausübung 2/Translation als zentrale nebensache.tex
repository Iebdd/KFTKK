\documentclass{article}

\usepackage{geometry}
\usepackage{makecell}
\usepackage{array}
\usepackage{multicol}
\usepackage[ngerman]{babel}
\usepackage[ngerman=ngerman-x-latest]{hyphsubst}
\usepackage{setspace}
\usepackage{changepage}
\usepackage{booktabs}
\usepackage{graphicx}
\usepackage{float}
\newcolumntype{?}{!{\vrule width 1pt}}
\renewcommand\theadalign{tl}
\setstretch{1.10}
\setlength{\parindent}{0pt}

\geometry{top=12mm, left=1cm, right=2cm}
\title{\vspace{-3cm}21S 520.230 Deutsch: Mutter-/Bildungssprache: Textanalyse und Textproduktion Gruppe 3}
\author{Andreas Hofer}

\begin{document}
	\section*{Translation als zentrale Nebensache in einer globalisierten Welt - eine Einführung \\ {\footnotesize Mira Kadrić \& Klaus Kaindl}}
	Übersetzungen und Dolmetschungen begegnen und täglich, sie sind allgegenwärtige Phänomene und ohne sie sähe unser Alltag ganz anders aus. Wer zur Installierung der Software auf dem neuen Computer ein Handbuch zu Rate ziehtm eine Blu-Ray-Disc mit Untertiteln abspielt, den neuesten Roman von Michel Houellebecq auf Deutsch liest, in den Sportnachrichten im Fernsehen ein Interview mit dem brasilianischen Fussballer Ronaldinho verfolgt und dabei über dem Originalton seine Antworten auf Deutsch hört, konsumiert Übersetzungen und Dolmetschungen. Man kann ohne Übertreibung sagne, dass Translation für den Informations- und Bildungsfluss der Welt unverzichtbar ist und in allen Lebenssituationen eine wichtige Rolle spielt: im öffentlichen Raum, wenn es z.B. um faire Behördenverfahren oder einen guten Zugang zur Gesundheitsversorgung geht, ebenso wie im privaten Bereich, wenn wir ein übersetztes Buch lesen, eine Gebrauchsanweisung konsultieren oder eine amerikanische Fernsehserie in Synchronfassung ansehen. Ohne Translation würden internationale Organisationen wie die Europäische Union oder die Vereinten Nationen ebenso wenig funktionieren wie Wirtschaftsunternehmen, die ihre Produkte global vermarkten. Der Bedarf an Translation, an Übersetzungen und Dolmetschungen ist enorm und bildet ein wirtschaftlich relevantes Marktsegment. Unser Zeitalter der Globalisierung ist somit zweifellos auch das Zeitalter der Translation. Migrationsströme, immer engere wirtschaftliche Verflechtungen und Bevölkerungswachstum werden den Bedarf an Translation in Zukunft sogar noch ansteigen lassen. \\ \\
	Die angeführten Beispiele machen jedoch auch deutlich, dass viele Übersetzungen und Dolmetschungen gar nicht bewusst wahrgenommen werden. Neben der Tatsache, dass Translation eine Selbstverständlichkeit zu sein scheint, hat dies auch mit der Unsichtbarkeit \\
	\textbf{Seite 2} \\
	jener zu tun, die diese Aufgabe erledigen. Selbst dort, wo TranslatorInnen im wahrsten Sinne des Wortes in Erscheinung treten, etwa bei einem Gipfeltreffen von StaatspräsidentInnen, fehlen sie oft auf den Aufnahmen im Fernsehen und dne Fotos in Zeitungen. \\ \\
	Aber gerade weil TranslatorInnen oft nicht in Erscheinung treten, sondern im Hintergrund agieren, herrschen meist unklare Vorstellungen über ihre Aufgabe und die mit dieser Tätigkeit verbundenen Anforderungen. Und auch die Vorstellung über die verschiedenen Arbeitsbereiche und EInsatzgebiete sind oft diffus. Der vorliegende Band unternimmt den Versuch, einen Überblick über die vielen translatorischen Tätigkeitsfelder und die dafür nötigen Kompetenzen zu geben. Als AutorInnen wurden daüfr maßgebliche VertreterInnen der Wissenschaft und erfahrenen PraktikerInnen, also ÜbersetzerInnen und DolmetscherInnen aus verschiedenen Tätigkeitsbereichen gewonnen. Die einzelnen Beiträge wenden sich zum einen an Studierende translationswissenschaftlicher und interkultureller Fächer bzw. Interessierte an einer solchen Ausbildung. Ihnen will dieser Band eine Orientierung bieten, welche Vorraussetzungen sie für die translatorische Ausbildung mitbringen sollten, welche Fertigkeiten und Techniken das Studium vermittelt und welche Berufsfelder und Einsatzgebiete ihnen nach abgeschlossenem Studium offenstehen und was sie dort erwartet. Zum anderen sind Lehrende an translatorischen Ausbildungsstätten angesprochen. Sie sollen einen Überblick über eine sich rasch wandelnde translatorische Berufspraxis erhalten sowie Unterstützung bei der Feststellung, welche Kompetenzen künftigen TranslatorInnen jedenfalls vermittelt werden sollten. Diesen beiden Hauptzielgruppen und allen an Translation Interessierten wollen wir einen Überblick übder das translatorische Tätigkeitspanorama mit all seinen theoretischen und pragmatischen Facetten geben (von der Ausgestaltung der Auftragsverhältnisse, den Erwartungen der Auftraggebenden bis hin zum Einsatz von neuen Medien und Techniken). \\ \\
	Wir gehen dabei von einem breiten Verständnis von Übersetzen und Dolmetschen aus. Wenn von beiden Tätigkeitsbereichen die Rede ist, so sprechen wir von Translation, ein Oberbegriff der von Otto Kade (1968) geprägt wurde. Unser Verständnis von Translation orientiert sich dabei am Ansatz von Justa Holz-Mänttäri (1984) und meint mündliche und schriftliche Textproduktion für fremden Bedarf über Sprach- und Kulturgrenzen hinweg. Wie vielfältig und unterschiedlich die verschiedenen Formen der translatorischen Textproduktion sein können, soll nicht zuletzt dieser Band zeigen. \\
	\textbf{Seite 3} \\

	\subsection*{Berufspraxis und Wissenschaft}
	Im Gegensatz zu vielen anderen Studiengängen ist das Studium im Bereich Translation auf konkrete Berufsbilder ausgerichtet. Vielleicht sind gerade deshalb auch die Erwartungen der Studierenden in diesem Bereich weniger auf Wissenschaft und Forschung als auf den Erwerb von Fertigkeiten und konkreten Handlungspraktiken ausgerichtet. Professionelle Translation ist jedoch keine angeborenen Fähigkeite, die durch praktisches Üben lediglich verfeinert werden muss, sondern eine komplexe kulturelle Praxis, deren Ausübung von zahlreichen Faktoren abhängt, die sowohl sprachlicher, kultureller, kognitiver als auch soziologischer, ideologischer und ethischer Natur sein können. Theoretische Reflexion und wissenschaftliche Auseinandersetzung sind daher integrative Bestandteile einer professionellen Translationspraxis, wie vielfach betont wird (u.a. Wilss 1983; Reiß 1995; Chesterman/Wagner 2002; Kaindl 2005; Kadrić 2011). \\
	Dass das Verhältnis zwischen Translationstheorie und -praxis lange Zeit angespannt war, ist Teil der Geschichte dieses Faches. Der Annäherungsprozess zwischen beiden ist zwar noch im Gange, es ist jedoch unübersehbar, dass das gegenseitige Misstrauen zum großen Teil überwunden ist. Dies hat zum einen mit veränderten Erwartungen der PraktikerInnen an die Translationstheorie zu tun. PraktikerInnen erkennen inzwischen vielfach den Vorteil von wissenschaftlichen Theorien an, die nicht allein auf subjektiven Erfahrungswerten, sondern auch methodisch gewonnenem Wissen basieren. Durch dieses erhalten sie einen Erklärungs- und Argumentationsrahmen, der ihnen hilt, praktische Entscheidungen bewusste, systematischer und widerspruchsfrei zu treffen. \\
	Doch auch die Translationswissenschaft hat sich im Laufe ihrer Geschichte der Praxis angenähert und so zu einer Entspannung beigetragen. Die ursprüngliche Distanz der Theorie zur gelebten Berufspraxis ist dabei auf die Anfänge der Disziplin und die damit verbundenen Erkenntnisinteressen zurückzuführen. \\
	Reflexionen über die Tätigkeit und wie man sie ausüben sollte, gibt es seit Jahrtausenden. Von Cicero, Hieronymus (der im Übrigen als Schutzpatron der ÜbersetzerInnen gilt) \\
	\textbf{Seite 4} \\

	über Luther bis hin zu Goethe, Schleiermacher und Benjamin spannt sich der Bogen von Schriftstellern, Theologen und Philosophen, die sich mit der Frage der Übersetzung auseinander gesetzt haben. Die Basis bildete dabei immer die eigene praktische Erfahrung, aus der jeweils Grundsätze und leitlinien abgeleitet wurden. Einen wissenschaftliche Beschäftigung im eigentlichen Sinne entwickelte sich allerdings erst nach dem 2. Weltkrieg. Die Übersetzungswissenschaft wurde dbaie eigentlich als eine Hilfswissenschaft für die Entwicklung der Maschinellen Übersetzung \glqq erfunden\grqq. Dementsprechen konzentrieren sich die ersten Theorien auf das Sprachsystem, auf die von Sassure als \textit{langue} bezeichnete Ebene der Sprache. Nicht die konkrete Sprachverwendung, sondern der abstrakte Regelapparat der Sprache stand im Mittelpunkt. Der Faktor Mensch, also jenen Personen, die Übersetzungen und Dolmetschungen anfertigen, blieb weitgehend ausgeklammert. \\
	Nachdem die Programme zur Maschinellen Übersetzung entgegen der ursprünglichen Erwartungen nicht so einfach und rasch zu entwickeln waren, wandte man sich ab Ende der 1960er Jahre verstärkt dem Text als Übersetzungsgrundlage zu. Die textlinguistischen Ansätze, wie sie etwa von Katharina Reiß (1971) vertreten wurden, rückten die Pragmatik, also den Verwendungszusammenhang von Sprache, in den Mittelpunkt. Die neue Ausrichtung führte dazu, dass man sich nicht mehr auf Strukturen konzentrierte, sondern die kommunikativen Zusammenhänge als wesentlich für die Übersetzung erachtete. Eine entscheidende Ausweitung erfuhr dieser Ansatz durch die sogenannte \glqq Neuorientierung\grqq \: (Snell-Hornby 1986) in den 1980er Jahren. Sprachverwendung wurde nunmehr in ihrer kulturellen Einbettung betrachtet, die Übersetzung war licht länger lediglich Sprach-, sondern Kulturtransfer (vgl. Vermeer 1986). Diese sogenannte funktionale Sichtweise auf Translation brachte auch einen, wie Prun\v c es nennt, \glqq Paradigmenwechsel\grqq \: mit sich. War bisher der Ausgangstext der Maßstab für die Translation, wurde nun der Zweck, den der Zeiltext erfüllen sollte, zum entscheidenden Kriterium für das translatorische Handeln. Er führte auch zu einer entscheidenden Aufwertung der TranslatorInnen, die nunmehr in ihrer Rolle als ExpertInnen einen wichtigen Platz in der Theorienbildung einnahmen (vgl. Holz-Mänttäri 1984). \\
	\textbf{Seite 5} \\

	Damit war ein Anstoß gegeben, nicht länger lediglich Sprachen, Texte oder Kulturen in der translationswissenschaftlichen Betrachung in den Mittelpunkt zu stellen, sondern die Person des/der Translator/In selbst. Chesterman spricht folgerichtig auch von den sogenannten \glqq Translator Studies\grqq \: , die mit den 1990er Jahren zunehmen an Bedeutung gewannen. TranslatorInnen wurden nun verstärkt aus unterschiedlichen Perspektiven analysiert (vgl. Chesterman 2009:19f.): Einerseits begann man sich für die Rolle von TranslatorInnen in der Geschichte zu interessieren, ihre Einstellungen, ihre ethischen Positionen und ihren Beitrag zur Entwicklung einer Kultur; andererseits rückten die kgnitiven Prozesse, die Wege der Entscheidungsfindung, der Einfluss von Emotionen und Einstellungen einer Person im Zuge des Translationsprozesses in den Fokus der Forschung. Und schließlich bekamen auch soziologische Fragen mehr Gewicht, wie etwa der Status von TranslatorInnen, Berufsorganisationen, Arbeitsbedingungen, Akkreditierungssysteme etc. In diesem Zusammenhang ist auch die Darstellung von TranslatorInnen in literarischen Werken und Filmen zu erwähnen, die sich zu einem neuen Forschungszweig entwickelt hat, in dem u.a. untersucht wird, welchen Einfluss die fiktionale Darstellung von ÜbersetzerInnen und DolmetscherInnen auf die Wahrnehmung in der Öffentlichkeit hat (vgl. Kaindl/Kurz 2010). \\
	Die Translationswissenschaft hat also in ihrer relativ kurzen Geschichte einen weiten Weg zurückgelegt, indem sie sich von einer abstrakten, strukturfixierten Sichtweise auf Translation verabschiedet und zu einer Disziplin entwickelt hat, die das Phänomen ganzheitlich, vor allem auch unter Einbeziehung der handelnden TranslatorInnen und ihrer Berufsrealität, erforschen und erklären will.
	\subsection*{Berufspraxis und universitäre Ausbildung}
	Jahunderteland erfolgte der Erwerb translatorischer Kompetenz intuitiv und autodidaktisch. Abgesehen von einigen historischen Beispielen - zumeist für Dolmetscher im diplomatischen Dienst, wie etwa die sogenannten \glqq Sprachknaben\grqq \: zur zeit Maria Theresias - war die Ausbildung im Bereich Übersetzen und Dolmetschen nicht institutionalisiert. Erst im 20. Jahrhundert - das auch das das Jahrhundert der Übersetzung bezeichnet wird, da aufgrund verschiedener \\
	\textbf{Seite 6} \\

	politischer und wirtschaftlicher Notwendigkeiten ein immer höherer Bedarf an Translationsleistungen entstand - werden universitäre Ausbildungsstätten gegründet. Die Entwicklung der einzelnen Studienrichtungen im Bereich Translation in EUropa könnte man dabei zusammenfassend wie folgt beschreiben: die 1930er bis 1950er Jahre als Gründungsphase der ersten Institute, die 1960er Jahre als Stabilisierungsversuch im Bemühen, Übersetzungs- und Dolmetschfertigkeiten \glqq praktisch\grqq \: zu vermitteln. Die 1970er Jahre brachten eine Reorganisation und neue Studienpläne, während die echte Emanzipierung des Faches erst Ende der 1980er Jahre begann (\glqq Neuorientierung\grqq). Die 1990er Jahre können als Konsolidierungsphase des Faches bezeichnet werden: translationswissenschaftliche Theorien und Anwendungsmodelle mit interdisziplinärer Ausrichtung finden Verbreitung (vgl Kadrić 2011:17f.). Insgesamt ist weltweit ein rasanter Anstieg an translatorischen Ausbildungsstätten festzustellen, die sich allein von den 1970er bis in die 1990er Jahre verfünffacht haben (vgl. Caminade/Pym 1998). \\
	Die Entwicklungen der letzten beiden Jahrzehnte führten nicht nur quantitativ, sondern auch qualitativ zu wesentlichen Änderungen - insbesondere durch die Einführung der neuen Studienarchitektur in europäischen Ländern. Die in den Leitbildern und Curricula der Hochsculen definierten Ziele und die sich daraus ergebende Einforderung einer hohen Qualität der forschungsgeleiteten Lehre sollen idealerweise dazu führen, dass Forschung und Lehre miteinander kommunizieren, dass die an den Universitäten tätigen Lehrenden und Forschenden die Auseinandersetzung um Zielsetzungen und Definitionskriterien gemeinsam mitgestalten und dass Forschung und Lehre in ihrer Bedeutung in einem komplementären Verhältnis stehen. \\
	Die Curricula der einschlägigen Ausbildungsstätten in Europa weisen bei all ihrer Verschiedenheit unverkennbare gemeinsame Züge auf. Trotz länderspezifisch abgestimmter Konzeptionen der Studien ist allen universitären Translationsausbildungen die Vorbereitung auf die berufliche Wirklichkeit gemeinsam. Im Wesentlichen stimmen auch die fachlichen Lehrinhalte dahingehend überein, dass das Studium die Vermittlung von Sprach-, Kultur- und Sachkompetenz, Recherchierkompetenz, das Trainieren von Technik, das Erlernen von Translationsstrategien usw. umfasst. \\
	\textbf{Seite 7} \\

	Darüber hinaus sollen universitäre Lehre und Studium im Rahmen der Erlangung einer Gesamtkompetenz den Studierenden ermöglichen, über ihre Rolle in der Gesellschaft und ihre Verantwortung sich selbst und der Gesellschaft gegenüber nachzudenken. Dies ist umso wichtiger, als die Bedeutung hochwertiger Translationsleistungen für Demokratie, Rechtsstaat und Sozialstaat zunehmen erkannt wird. Als Beispiel sei hier das Dolmetschen im öffentlichen Raum genannt. Zentrale Gesundheitsbehörden und Krankenanstalten unternehmen in Europa in jüngster Zeit Anstrengungen, um eine flächendeckende Versorgung der Krankenhäuser mit qualifizierten Dolmetschleistungen sicherzustellen. Ähnliches gilt für das Polizei- und Gerichtsdolmetschen. Seit der Jahrtausendwende hat sich die Europäische Kommission bemüht, mittels interdisziplinär besetzter ExpertInnenforen den Standard der Dolmetschung vor den Strafgerichten der Union zu verbessern. Auch ÜbersetzerInnen arbeiten längst nicht mehr - wie es einem alten Klischee entspricht - im stillen Kämmerlein, sondern interagieren und kooperieren mit anderen ExpertInnen. TranslatorInnen benötigen in der heutigen Zeit somit ein feines soziales und kommunikatives Sensorium. Die Translationsqualität bestimmt etwa den Ausgang eines Asyl- oder Strafverfahrens und damit den weiteren Lebensweg von Menschen mit. Ähnliches gilt für die Kommunikationsmittlung im Gesundheitswesen. Fehler bei der Dolmetschung können hier zu fehldiagnosen und dramatischen Folgen führen. \\
	Die steigenden gesellschaftlichen Erwartungen und damit verbunden die gesellschaftliche Verantwortung sowie die Anforderungen des Arbeitsmarktes müssen auch in der Translationsausbildung ihren Niederschlag finden. Zwar ist es äußerst wichtig, fachliche Kenntnisse, Fähigkeiten und Fertigkeiten zu erwerben, doch dies allein genügt immer weniger. Veilfältige Einsatzmöglichkeit auf dem Arbeitsmarkt und eine anzustrebende gesellschaftlich-politische Bildung machen es nötig, neben Fach- und Medienkompetenz auch sozial-kommunikative und affektiv-ethische Studienkomponenten in die Ausbildung zu integrieren. Das bedeutet, dass man bereits während des Studiums selbstständiges Handeln lernt, interaktiv und selbstverantwortlich agiert ung Probleme selbstständig löst. \\
	Die zeitgenössische Translationsdidaktik definiert daher das Lernen als Prozess und selbstständiges, reflexives Handeln. Es beinhaltet Strategien, die die bewusste Wahrnehmung und Reflexion der \\
	\textbf{Seite 8} \\
	fachlichen metafachlichen Fähigkeiten fördern, aber auch gesellschaftliche Zusammenhänge hinterfragen und berücksichtigen (vgl. u.a. Arrojo 1996; Kiraly 2000; Chesterman/Wagner 2002; Cronin 2005; Gile 2009; Kadrić 2011; Kearns 2013; Pöchhacker 2013). \\
	Vor einem solchen Hintergrund muss auch die häufig erhobenen Forderung nach \glqq Praxisnähe\grqq \: des Unterrichts neu definiert werden. Jahrzehntelang erschöpfte sich die Translationslehre \glqq in der möglichst interessanten Wiedergabe der Tipps und Tricks erfahrener Praktiker\grqq \: (Prun\v c 2004:11) Praxisrelevanz war dabei ein rein subjektiver Erfahrungswert, der jedoch zu einer umfassenden, professionellen und fundierten Kompetenzvermittlung nicht ausreicht. Dafür ist eine entsprechende wissenschaftliche Fundierung und pädagogische (Vor)Bildung der Lehrenden notwendig. In diesem Zusammenhang wäre eine Diskussion um die Ausbildung von Lehrenden äußerst wichtig (vgl. Kiraly 2000; Englung Dimitrova 2002; Kelly 2005, 2008 und 2010; Kadrić 2011; Pym 2011). \\
	Wenn wir daher die Aufgabe des Translationsunterrichts in der Vermittlung einer Gesamtkompetenz sehen, so ist damit eine ganzheitliche integrative Fähigkeit gemeint, die in die miteinander vernetzten Komponenten der Fach-, Methoden-, Sozial und Individualkompetenz gegliedert werden kann. Die Fachkompetenz als Teil der materiellen Bildung fragt nach dem Wissen, das die Studierenden ansammeln; die Methodenkompetenz meint Fähigkeiten und Fertigkeiten im Umgang mit Wissen; die Sozialkompetenz vermittelt durch die Wahl von Sozial- und Aktionsformen im Unterricht inbesondere die Team- und Kooperationsfähigkeit; und schließlich die Individualkompetenz, die darauf wert legt, dass Themen nicht nur sachadäquat. sondern vor allem auch interaktions- und kommunikationsadäquat behandelt werden. \\
	Ein solcher Translationsunterricht, der den Menschen in das Zentrum seiner didaktischen Ziele rückt, ist letztlich entscheidend dafür, dass zukünftige TranslatorInnen nicht nur die sachliche und fachliche Kompetenz, sondern auch das Selbstvertrauen und das Selbstbewusstsein erwerben, um ihre Tätigkeit im Dienste der Gesellschaft erfolgreich ausüben zu können.
	























\end{document}