\documentclass{article}

\usepackage{geometry}
\usepackage{makecell}
\usepackage{array}
\usepackage{multicol}
\usepackage{setspace}
\usepackage{changepage}
\usepackage{booktabs}
\newcolumntype{?}{!{\vrule width 1pt}}
\renewcommand\theadalign{tl}
\setstretch{1.10}
\setlength{\parindent}{0pt}

\geometry{top=12mm, left=1cm, right=2cm}
\title{22S GSD.02100UB Grundprobleme der Neuen Geschichte}
\author{Andreas Hofer}

\begin{document}
	\section{Einleitung - 10.03.2022}
	\textbf{Europa ist ein Konstrukt.} Während der Eurasische Kontinent eindeutig definiert sind, ist im Osten nicht genau klar, wie weit es sich erstreckt und wann Asien beginnt. Historisch gesehen wird die Geschichte durch Imperien definiert, gleichzeitig zeigte sich das Ende eines Imperiums noch nie ohne Krieg. Man nahm an, dass der größtenteils friedliche Zerfall der Sowjetunion diesen Trend nicht unterstützte, jedoch wird es nun mit der russischen Invasion der Ukraine klar, dass Russland als Imperiale Macht nicht mit dem Zerfall der Sowjetunion endete. \\
	

























	
\end{document}