\documentclass{article}

\usepackage{geometry}
\usepackage{makecell}
\usepackage{array}
\usepackage{multicol}
\usepackage{setspace}
\usepackage{changepage}
\usepackage{booktabs}
\usepackage{graphicx}
\usepackage{pdfpages}
\usepackage{float}
\usepackage{wrapfig}
\newcolumntype{?}{!{\vrule width 1pt}}
\renewcommand\theadalign{tl}
\setstretch{1.10}
\setlength{\parindent}{0pt}
\graphicspath{{./Images/}}

\geometry{top=12mm, left=1cm, right=2cm}
\title{22S 520.042 Einführung in die Translationswissenschaften}
\author{Andreas Hofer}

\begin{document}
	\section*{Organisatorisches}
	\begin{itemize}
		\item{Kein synchroner Livestream aber Aufzeichnungen  (Aus 20/21 aber mit dem selben Inhalt).}
	\end{itemize}
	\subsubsection*{Drei Prüfungstermine:}
	\begin{itemize}
		\item{27.6.2022}
		\item{18.07.2022}
		\item{19.9.2022}
	\end{itemize}
	"Perception-Prüfung" mit PCs welche automatisch Fragen zur Verfügung stellt. Ebenfalls Zuordnung von Konzepten und Lückentexte, bei dem man die genannten Konzepte in die Texte einbauen muss. \\ \\
	Vorlesung behandelt die Einführung neuer translatorischen Methoden, welche die alten jedoch nicht ersetzt, sondern lediglich ergänzt haben. \\
	Fokus auf die "Verschiedenen Entwicklungslinien der Translationswissenschaften sowie deren Fragen" und die Frage wie sie zusammenhängen. \\
	Grundlegende Fragen dieser Vorlesung:
	\begin{itemize}
		\item{Anfänge bei der Frage, was macht eine gute Übersetzung aus? Angefangen mit Martin Luther.}
		\item{Was ist der Prozess des Übersetzens sodass eine gute Übersetzung entstehen kann.}
		\item{Was sind die Merkmale einer guten Übersetzung?}
		\item{Wie macht sich das literarische Übersetzen bemerkbar und wie verändern diese Kulturen?}
		\item{Was ist die Rolle der Übersetzung und welche Rolle und welchen Status haben die Personen die diese entwickelten?}
	\end{itemize}

	\newpage
	\section{Translation ist überall - 28.02.2022}
	Die Behauptung steht: Translation ist ein allgegenwärtiges sprachliches und kulturelles Phänomen. Doch wieso? \\
	Die CIUTI, eine Organisation zur Verbindung von Translatorischen Instituten wie das ITAT, hat ein "Policy Statement" ausgegeben und einige Punkte genannt:
	\begin{itemize}
		\item{Zur \textbf{Kommunikation} selbst: Nicht jeder spricht Englisch oder will Englisch sprechen und aus diesem Grund sind selbst in Zeiten wo Englisch als Lingua Franca agiert, andere Sprachen noch relevant.}
		\item{Zur \textbf{Qualität der Kommunikation}: Selbst wenn man Englisch spricht, beherrscht man es wahrscheinlich nicht im gleichen Ausmaß wie seine Muttersprache. Aus diesem Grund kann die Qualität der Interaktion zwischen Kulturen besser geschehen, wenn es keinen "Filter" in Form von Englisch gibt.}
		\item{Zur \textbf{Vorbeugung von Monolingualismus}: Viele sehen Englisch als einzige lernenswerte Sprache, da dadurch ein gemeinsamer Nenner geschaffen wird, jedoch sind alle Sprachen eine Bereicherung und diese statt dem Englischen zu lernen führt zu einer Diversifikation der Kulturen, anstatt einer Singularität um das Englische.}
		\item{Zur Durchsetzung der \textbf{Rechte aller Kulturen}: Jeder hat das Recht Medien und andere Sachen in seiner eigenen Sprache konsumieren zu können anstatt stets zu einer Zwischensprache ausweichen zu müssen.}
	\end{itemize}

	Es gibt viele, eventuell unsichtbare, Arten des Übersetzens:
	\begin{itemize}
		\item{Übersetzen von Anleitungen und Packungsbeilagen}
		\item{Lokalisierung von Nachrichten. Nicht unbedingt den Text selbst sondern dessen Quellen und auch der Methode des Informationssammelns}
		\item{Auch Lokalisierungen von Interfaces: Zum Beispiel Websites aber auch die Benutzeroberfläche von Programmen.}
		\item{Dolmetschen für Migranten bei Arztbesuchen. Meist sind spezialisierte Begriffe nur schwer zu übersetzen und eine Genauigkeit der Übersetzung ist vorausgesetzt.}
		\item{Übersetzung wissenschaftlicher Texte. Während eine 1:1 Übersetzung des Textes eventuell angemessen wäre, müssen gleichzeitig die kulturellen Eigenheiten betrachtet werden. So kann ein Titel im Englischen ein radikal anderer als in der deutschen Übersetzung sein.}
		\item{Übersetzung von Filmen und Übertragungen. Während im DACH Bereich sowie auch in Südeuropa Synchronisierungen die Norm sind, ist es in Skandinavien aber auch Belgien und den Niederlanden verbeitet, dass Filme aber auch Übertragungen fremdsprachiger, unter anderem, Politiker mit Untertitel versehen werden, während stets die Originalstimme als Ton beibehalten wird.}
	\end{itemize}
	\includepdf[landscape=true]{Skalen_der_Übersetzung.png}
	\newpage
	\section{Das Heilige Wort - 07.03.2022}
	Bibelübersetzungen haben seit Jahrhunderten Tradition: Der Wille die Heilige Schrift in andere Sprachen zu übersetzen. Bei religiösen Übersetzungen ist stets größte Vorsicht geboten, da Interpretationen von gewissen Wörtern weise gewählt werden müssen. Mikael Agricola, ein Schüler Luthers, wurde beuaftragt die Bibel in das Finnische zu übersetzen und verwendete um dies zu erreichen sechs verschiedene Versionen der Bibel. \\
	Einen ähnlichen Status genießen heutzutage große Werke der Fiktion, welche in etliche Sprachen übersetzt werden. In diesen Übersetzungen von, beispielsweise Herr der Ringe oder Harry Potter, wird oft, wie bei der Bibel, der genaue Wortlaut zwischen Übersetzungen verglichen und kritisch diskutiert. \\
	Man kann Übersetzen als "Vergleichshandlung" betrachten, wo gewisse Gegebenheiten existieren müssen um es realisieren zu können. Eventuelle Fragen sind:
	\begin{itemize}
		\item{Wie soll man gewisse Sachen ausdrücken?}
		\begin{itemize}
			\item{Beispielsweise idiomatische Redewendungen, welche in anderen Sprachen oft nicht 1:1 vergleichbar sind}
		\end{itemize}
		\item{Existiert das Phänomen in der Zielkultur?}
		\begin{itemize}
			\item{}
		\end{itemize}
		\item{Kennt meine Leserschaft dieses Phänomen?}
		\begin{itemize}
			\item{Kulturspezifische Bezeichnungen für vermeintlich sehr generelle Sachen, wie zum Beispiel "Die Mauer", welche in Deutschland immer den Eisernen Vorhang bezeichnet, in einer anderen Sprache aber eventuell keine Konnotation zu etwas kpnkretem.}
		\end{itemize}
		\item{Ist die Übersetzung getreu der Vorraussetzungen?}
		\begin{itemize}
			\item{Erfüllt sie die spezifischen Vorraussetzungen die dieses Übersetzungswerk benötigt? Dies kann zum Beispiel eine Vorraussetzung sein, dass es in 60 Minuten schon benötigt wird.}
		\end{itemize}
	\end{itemize}
	\subsection{Vergleich als Methode der Übersetzung}
	Translationswissenschaft kann in zwei grobe Hauptorientierungen geteilt werden:
	\begin{itemize}
		\item{Prospektive Ansätze}
		\begin{itemize}
			\item{Beachten der Textvorraussetzungen eines spezifischen Textes. Zum Beispiel Äquivalenz}
		\end{itemize}
		\item{Retrospektive Ansätze}
		\begin{itemize}
			\item{Eher rückblickendes Ansehen bereits getätigter Übersetzungen. Dadurch soll herausgefunden werden, wie sie entstanden sind und was zu ihnen geführt hat und wie diese später zur Bildung der Theorie genutzt werden können.}
		\end{itemize}
	\end{itemize}
	Die beiden Orientierungen sehen jeweils nach vorne oder nach hinten. Eine denkt über die Übersetzungen der Zukunft und wie diese getätigt werden kann, während die andere die Übersetzungen der Vergangenheit ansieht und anhand dieser Aufschlüsse über die bessere Anwendung für zukünftige Übersetzungen gibt. \\
	Kritiken an der Übersetzungswissenschaft rührt meist von der Unsichtbarkeit des Übersetzers her. Eine Person, welche eine Übersetzung liest nimmt in der idealen Übersetzung den Übersetzer gar nicht wahr und dieser verschwindet hinter dem Autor. \\
	\subsection{Was ist Gleichheit?}
	Wenn wir von Übersetzung sprechen, wird oft angenommen, dass etwas verglichen wird, was stets zur Frage der Gleichheit führt. Dabei stellt sich oft die Frage: Was ist Gleichheit eigentlich? Wann sind zwei Texte "gleich"? \\
	Meist stellen sich bei Gleichheit die Frage verschiedener Gleichheiten:
	\begin{itemize}
		\item{Inhalt oder Form}
		\begin{itemize}
			\item{Soll der Inhalt des Textes oder die Form des Textes übereingestimmt werden? }
		\end{itemize}
		\item{Untreue oder Treue}
		\begin{itemize}
			\item{Wem hat der Übersetzer bei einer Übersetzung die Treue zu halten? Dem Leser, damit dieser es eher versteht, oder dem Autor, damit dessen Vision so getreu wie möglich wiedergegeben werden?}
		\end{itemize}
		\item{Verfremdung oder Entfremdung}
		\item{Sinntreue oder Formtreue}
		\item{Wort oder Sinn}
		\begin{itemize}
			\item{Hieronymus schrieb, dass er nicht Wörter durch andere Wörter ausdrückt, sondern einen Sinn durch einen anderen, selbst wenn diese in der Zielsprache weniger verständlich sind. Anders sah dies Martin Luther mit seinem bekannten Zitat "Dem Volk aufs Maul schauen"}
			\item{Luther erhielt schärfste Kritik für diese Ansicht. Als er das Wort "allein" in einem Satz einbringt, ohne dass es im Originaltext vorhanden war, wurde ihm nachgesagt, dass er Gott Wörter in den Mund legen wolle. Später veröffentlichte er einen offenen Brief über das Übersetzen und kritisiert die Priester, welche ihn kritisiert hatten, da diese selbst auch keinen Versuch wagen die Bibel zu übersetzen.}
		\end{itemize}
	\end{itemize}
	Zusammengefasst, sehen viele Personen bei der Bibelübersetzung eine möglichst Sinngetreue Übersetzung als unabdingbar. Luther hingegen, fand, dass ein Buch, welches vom gemeinen Volk nicht verstanden werden kann, keinen Sinn macht. \\
	\subsection{Friedrich Schleiermacher}
	Friedrich Schleiermacher, geboren in[[GEBURT HINZUFÜGEN]], prägte zwei langlebige Unterscheidungen:
	\begin{itemize}
		\item{Das Literarische Übersetzen}
		\begin{itemize}
		\item{Die Paraphrase}
		\item{Die Nachbildung}
		\end{itemize}
		\item{Schriftliches und mündliches Fachübersetzen}
	\end{itemize}
	Schleiermacher behandelt gibt es bei Übersetzungen nur zwei Ausgänge: Entweder der Autor wird größtenteils in Ruhe gelassen und der Übersetzer bringt den Leser näher an den Autor, oder der Leser wird in Ruhe gelassen und der Autor wird näher an den Leser gebracht. Für ihn war das Übersetzen grundsätzlich "ein Vorgang des Verstehens und des Zum-Verstehen-Bringens". \\
	Gleichzeitig war er der Ansicht, dass unterschiedliche Textgattungen auch andere Methoden der Übersetzung erfordern. \\
	Ein ähnliches Konzept brachte Goethe mit \textbf{Goethe's Maxim} hervor: \textit{Es gibt zwei Übersetzungsmaximen: Die eine verlangt, dass der Autor einer fremden Nation zu uns herüber gebracht werde, sodass wir ihn als unsrigen ansehen können, während die andere erwartet, das}

	\section{Sprachrelativismus vs Zeichentheorie - 14.03.2022}
	Da wir letztes Mal die Väter des Übersetzens und Dolmetschens kennengelernt haben, gibt es noch weitere, welche dieses Werk fortgeführt haben. Bei der Übersetzung spricht man auch oft von einem Seiltanz bzw. der Möglichkeit oder Unmöglichkeit und dem Kompromiss. \textbf{Wilhelm von Humboldt} war einer der Ersten, welcher sich über die grundsätzliche Machbarkeit einer Übersetzung Gedanken gemacht hat. Ebenfalls machte er sich Überlegungen über das Ausmaß, in welchem die Muttersprache das Denken beeinflusst. \\
	Er führte dies darauf zurück, dass man stets innerhalb einer Kultur eine Sprache lernt und dadurch das Wirklichkeitsverständnis geprägt wird. Dadurch leitet er schließlich die (Un-)Möglichkeit des Übersetztens ab. Er spricht auch davon, dass man nie möglichst nahe am Original bleiben kann, während man es auch dem Leser so verständlich wie möglich macht. \\
	Dies ist nicht nur der Unterschied zwischen Wörtern wie, zum Beispiel "Stuhl", sondern auch das Verständnis eines solchen Gegenstandes. \\
	Ein weiterer großer Name in der Sprachinhaltsforschung ist \textbf{J. Lei Weisgerber}, welcher über Eigentümlichkeiten zwischen verschiedenen Sprachen existieren. So beobachtete er, dass Verwandschaftsbezeichnungen oder Naturerscheinungen in manchen Sprachen viel genauer beschrieben werden. Weiters beschrieb er die Schwierigkeit der Wiedergab einiger Wörter in manchen Sprachen, wie zum Beispiel "Weltschmerz" oder "Sehnsucht" aber auch andere Wörter wie "gentleman" oder das finnische "keli". \\
	Gleichzeitig gibt es sehr generische Wörter, wie zum Beispiel das Wort "Wald" oder "Kaffee". Diese Wörter beschreiben zwar das selbe Konzept, haben aber oft unterschiedliche spezifische Auffassungen. Ein Kaffee in Italien ist wahrscheinlich ein Espresso, während es in Österreich viel schwächer ist. \\
	Diese Überlegungen wurden in der \textbf{Sapir-Whorf-Hypothese} formalisiert und beschreibt die Überlegung ob Sprache unser Denken beeinflusst. Ebenfalls, ob Eindrücke und Erfahrungen in der Umwelt oft unterschiedlich ausgedrückt werden, da Sprache die Konzepte relativiert. \\
	\begin{quote}
	\textit{Es besagt, grob gesprochen, folgendes: Menschen, die Sprachen mit sehr verschiedenen Grammatiken benützen, werden durch diese Grammatiken zu typisch verschiedenen Beobachtungen und verschiedenen Bewertungen äußerlich ähnlicher Beobachtungen geführt. Sie sind daher als Beobachter einander nicht äquivalent, sondern gelangen zu irgendwie verschiedenen Ansichten von der Welt. - Benjamin Lee Whorf 1963:20}
	\end{quote}

	\subsection{Zeichentheorie}
	Die Zeichentheorie war in dem Zusammenhang die wahrscheinlich einflussreichste Theorie und hat die Übersetzungswissenschaft umgehend geprägt. Es teilt die Sprache in drei Komponenten auf:
	\begin{itemize}
		\item{Langue}
		\begin{itemize}
			\item{Das Sprachinventar, das Regelsystem und deren Verknüpfungen}
		\end{itemize}
		\item{Parole}
		\begin{itemize}
			\item{Die Rede bzw. die tatsächlichen Äußerungen}
		\end{itemize}
		\item{Zeichen}
		\begin{itemize}
			\item{Wörter der Sprache, welche konkret aus Ausdruck und Inhalt bestehen.}
		\end{itemize}
	\end{itemize}

	Diese Entwicklungen habne dazu geführt, dass in der Übersetzungswissenschaft zu dem Wunsch geführt hat, Modelle zu entwickeln die offensichtlich wissenschaftlich sind. Man versuchte Übersetzen als Modell zu modellieren, sodass möglichst viele Tätigkeitsbereiche beschrieben werden würden. Stattdessen wurden kreativerer Modelle eher außenvor gelassen. Im Vordergrund von alledem stand die maschinelle Übersetzung. Es wurde auch durch den Kalten Krieg angefacht, indem versucht wurde möglichst effizient Informationen über den Gegner zu erhalten. \\
	Eines der einflussreichsten Modelle war das \textbf{Kommunikationsmodell von Shannon-Weaver}. In dem Modell wird die Kommunikation in Formalitäten umgewandelt und über gewisse Kanäle gesendet wird. Man nennt es auch oft "The Mother of all Models" da es sehr oft als Basis verwendet wurde. \\
	Später wurde Shannon-Weavers Modell in der Leipziger Schule von Otto Kade angepasst und in das \textbf{Dreiphasige Kommunikationsmodell} abgewandelt. Diese Modelle führten größtenteils dazu, dass man Übersetzung als etwas wissenschaftliches etablieren wollte. So sind auch viele recht triviale Begriffe durch andere ersetzte worden. Zum Beispiel nicht der Text sondern der \textit{Kode} und nicht das Übersetzen, sondern der \textit{Kodewechsel}. Die gesamte Terminologie spiegelt den Wunsch der Wissenschaftlichkeit wider und erschwert es oft die Theorie zu verstehen. \\

	\textbf{Werner Koller} erarbeitete 1979 die fünf Typen der potenziellen Äquivalenz:
	\begin{itemize}
		\item{Eins zu Eins Entsprechung}
		\begin{itemize}
			\item{Wenn es eine direkte Äquivalenz zwischen zwei Wörtern gibt: Großmutter -> Grandmother}
		\end{itemize}
		\item{Eins zu Viele Entsprechung}
		\begin{itemize}
			\item{Wenn es in der Ausgrangssprache nur ein Wört gibt, in der Zielsprache aber mehrere: Großmutter -> Farmor, Mormor}
		\end{itemize}
		\item{Viele zu Eins Entsprechung}
		\begin{itemize}
			\item{Wenn es in der Ausgrangssprache mehrere Wörter gibt, in der Zielsprache diese aber auf ein Wort zusammenlaufen}
		\end{itemize}
		\item{Eins zu Null Entsprechung}
		\begin{itemize}
			\item{Wenn es in der Zielsprache kein übersetzbares Wort gibt.}
		\end{itemize}
		\item{Eins zu Teil Entsprechung}
		\begin{itemize}
			\item{Wenn es zwar ein ähnliches, aber nicht äquivalentes Wort gibt}
		\end{itemize}
	\end{itemize}
	Vinay und Darbelnet kreierten auch die Sprachpaartheorie, in welcher gewisse Wörter und Redewendungen als Index direkt angegeben werden. Dabei wird zwischen Wörtlichen/Substituierenden und Nichtwörtlichen/Paraphrasierenden unterschieden. \\
	\begin{itemize}
		\item{Wörtlich}
		\begin{itemize}
			\item{Direktentlehnung: Inhaltliche unveränderte Wörter, werden direkt übersetzt -> Know-how, Small-Talk}
			\item{Lehnübersetzung: Zusammengesetzte Wörter werden in ihre Einzelteile getrennt und jedes einzeln übersetzt, bevor es wieder zusammengefügt wird -> Premierminister - pääministeri (Premier-Minister)}
			\item{Wortgetreue Übersetzung: Formale und genaue Übersetzungen: Today is a good day -> Heute ist ein guter Tag}
		\end{itemize}
		\item{Nichtwörtlich}
		\begin{itemize}
			\item{Transposition: Sinngetreue jedoch nicht äquivalente Übersetzung -> Er sagte, dass}
			\item{Modulation: Der Blickwinkel wird geändert -> Der Film war gut - Der Film war nicht schlecht}
			\item{Kulturelle Entsprechung: Redewendungen werden geändert um kulturelle besser zu entsprechen}
			\item{Adaptation: }
		\end{itemize}
	\end{itemize}
	\subsection{J. C. Catford - Translation Shifts}
	Catford war der Ansicht, dass eine Übersetzung stets ein substituierender Prozess ist, in welchem ein Text in der Sprache A als Text in der Sprache B wiedergegeben wird. Dadurch war er der Ansicht, dass eher eine Bezeichnungsäquivalenz, statt einer Bedeutungsäquivalent angesehen werden sollten. Übersetzungen hatten somit, selten die gleiche Bedeutung, selbst wenn es nur um kulturelle Unterschiede geht, haben, aber in gewissen Situationen das selbe bezeichnen. So können Sprichwörter, welche selten direkt übersetzbar sind, übersetzt werden, indem man die Idee des Sprichworts benutzt und ein äquivalentes Sprichwort in der Zielsprache verwendet. \\

	\section{Übersetzen als Textarbeit - 21.03.2022}

	Eugene Nida erhielt von der Amerikanischen Bibelübersetzungsgesellschaft den Auftrag eine Grundlage der Bibelübersetzung zu schaffen. In seinem Buch \textit{Toward a Science of Translating} erschienen 1964, beschrieb Eugene Nida diese Grundlage, welche Jahrzente lang als Grundlage für die Erarbeitung neuer Bibelübersetzungen gesehen wurde. \\
	Die Grundlage seines Ansatzes bestand aus:
	\begin{itemize}
		\item{Das Weltbild der Sprachen, inspiriert von der Amerikanischen Antropologie und Ethnografie}
		\item{Die Vorraussetzung einer adäquaten Übermittlung der Botschaft der Bibel}
		\item{Die Annahme der Wahrheit des linguistischen Relaitivitätsprinzips}
		\item{Das \textbf{Prinzip der Übersetzbarkeit:} Dass alle Aussagen einer Sprache auch in einer anderen Sprache gemacht werden können, wenn die Form (=das Wort und seine Bedeutung) nicht ein wesentlicher Teil der Botschaft ist.}
		\item{Die Abkehr vom "heiligen Original" und dem Grundsatz der Unveränderlichkeit des Ausgangstextes. So nahm er an, dass Anpassungen in Grammatik und Wortschatz nicht nur erlaubt, sondern sogar notwendig sind.}
	\end{itemize}

	Diese Grundsätze sind auch außerhalb der Bibelübersetzung nutzbar und werden als \textbf{dynamische Äquivalenz} bezeichnet. Nida und Taber sagten:
	\begin{quote}
	"Translating consists in reproducing in the receptor language in the \textbf{closest natural equivalent} of the source language message, first in terms of meaning and secondly in terms of style." - Nida/Taber 1969:11
	\end{quote}
	Nida beschrieb die Übersetzungsmethode als dreistufiges Modell:
	\begin{itemize}
		\item{Der Analyse: Der Einschätzung der Bedeutung des Satzes: "Das sind für mich böhmische Dörfer." - Unverständnis}
		\item{Transfer: Anpassung der Bedeutung in Bezug auf die ZS}
		\item{Synthese: Übertragung der Elemente in die ZS: "Se on minuelle täyttä hepreaa." - Das ist Hebräisch für mich}
	\end{itemize}

	Wichtig bei Nida sind stets:
	\begin{itemize}
		\item{Die Bedeutung des Wissen um den kulturellen Kontext}
		\item{Zuerst muss die Sache verstanden werden, bevor es in Wörter gefasst werden kann}
		\item{Eine Übersetzung ist nicht endgültig. Eine Übersetzung entsteht in einem historischen Kontext und kann sich stets ändern.}
	\end{itemize}
	Im Kontext der fehlenden Endügltigkeit der Übersetzung gibt es auch immer wieder Neuübersetzungen von bereits übersetzen Werken um diesen wieder "in die Gegenwart zu bringen".

	In den 1970er Jahren schuf \textbf{Werner Koller} das Konzept der \textbf{funktionalien Äquivalenz}, welches besagt, dass der Zweck der getreuen Übersetzung stets die Mittel des Übersetzers heiligt. So definierte er Typen der Äquivalenz:
	\begin{enumerate}
		\item{Der außersprachliche Sachverhalt - Die denotative Äquivalenz}
		\begin{itemize}
			\item{In welcher Reihenfolge sollte man auf einen Sachbezug eingehen?}
		\end{itemize}
		\item{Art der Verbalisierung - Die konnotative Äquivalenz}
		\begin{itemize}
			\item{Wahl zwischen Synonymen in Bezug auf Stilschicht, und dimensionen}
		\end{itemize}
		\item{Text- und Sprachnormen - textnormative Äquivalenz}
		\begin{itemize}
			\item{Gebrauchsnormen, welche abhängig der Textgattung sind}
		\end{itemize}
		\item{Empfänger und Empfängerinnen - pragmatische Äquivalent}
		\begin{itemize}
			\item{Abhängig der kulturellen Spezifika der Empfängerkultur}
		\end{itemize}
		\item{ästhetische, formale oder charakteristische Eigenschaften - formal-ästhetische Äquivalenz}
		\begin{itemize}
			\item{Übersetzung von Metaphern}
		\end{itemize}
	\end{enumerate}

	\subsection{Textsortenkonvention}
	Die textsortenkonvention befasst sich mit der grundlegenden Frage, was einen Text überhaupt ausmacht. Laut Radegunde Stolze ist eine Textsorte ein Sprech- und Schreibakttyp, welche an fixe Muster gebunden sind und somit bei ähnlichen Situationen auf vorherige Texte zurückgegriffen werden kann. Jedoch hat nicht jede Situation eine eigene Textsorte. (Stolze 1994:107)

	Zusätzlich werden Texte durch mikro- und makrostrukturelle Eigenschaften bezeichnet. Mikrostrukturelle Eigenschaften bezeichnen wie sich grammatikalische Eigenheiten äußern. So beschreibt man z.B. den Passiv oder Aktivkonstruktionen oder de Verwendung der Vergangenheit oder der Gegenwart. \\
	Makrostrukturelle Eigenschaften bezeichnen, in welcher Weise Teile eines Textes behandelt werden sollen z.B. wie man einen Brief beginnt und beendet. \\
	Durch diese Konventionen kann man Textsorten in inhaltsbetonte, formbetonte und effektbetonte Texttypen gegliedert werden können. \\
	Ein weiteres Modell ist Bühlers Organonmodell, welches von einem Sender und einem Empfänger spricht, sowie der expressiven und der denotativen Information. Anhand dieses Modells kreierte Katharina Reiß 1976 ihre Texttypologie. \\
	\begin{itemize}
		\item{Informativer Typ[[EXPAND]]}
		\item{Expressiver Typ}
		\item{Operativer Typ}
	\end{itemize}

	Mit dieser Typologie war Werner Koller nicht wirklich einverstanden, weshalb er dies auf zwei Textgattungen beschränkte, welche seiner Meinung nach mit der \textbf{Idealtypischen Unterscheidung} alles abdecken:
	\begin{itemize}
		\item{Sachtexte}
		\begin{itemize}
			\item{Diese können wiederum in allgemeinsprachliche, allgemein- und fachsprachliche und nur fachsprachliche Charakteristika geteilt werden. Also muss man bei einem Sachtext den erwarteten Wissensstand des Lesers in Betracht ziehen.}
		\end{itemize}
		\item{Fiktivtexte}
	\end{itemize}

	Die Unterscheidung zwischen den beiden Unterscheidungen sah er gegeben durch vier Kriterien:
	\begin{itemize}
		\item{Das Kriterium der sozialen Sanktion und der praktischen Folgen}
		\item{Das Kriterium der Fiktionalität}
		\item{Das Kriterium der Ästhetizität}
		\item{Intralinguistische, soziokulturelle und intertextuelle Bedeutungen}
		\begin{itemize}
			\item{Wenn ein fiktiver Text auf die kulturellen Gegebenheiten des Autors hinweisen.}
		\end{itemize}
	\end{itemize}

	Eine weitere Trennung, erarbeitet durch Juliane House 1977 ist die Unterscheidung zwischen Overt und Covert translations.
	Diese haben gewisse Parameter:
	\begin{itemize}
	 	\item{Worüber wird kommuniziert?}
	 	\item{Welche Beziehung wird zwischen den Kommunizierenden signalisiert?}
	 	\item{Mit welchem Code oder in welchem Medium wird kommuniziert?}
	 \end{itemize} 
	 Abhängig von diesen Parametern wird zwischen den beiden Arten unterschieden.
	\begin{itemize}
		\item{Over translation: Der Text ist als Übersetzung erkennbar und die Person weiß, dass sie ein sekundärer Rezipient ist}
		\item{Covert Translation: Der Ausgangstext ist verborgen. Eine Person hat den Eindruck, als ob sie der primäre Rezipient ist.}
	\end{itemize}
	Overt und Covert Translations gehören nicht in den Gegenstandsbereich der Translationswissenschaft, hat jedoch trotzdem eine wichtige Rolle innerhalb des Feldes der Übersetzung.

	\section{Funktionale Translationstheorien - 28.03.2022}

	Wiederum geschieht nun ein Perspektivenwechsel. Man geht nun fort von der Interpretation der Ausgangstexte, welcher versucht einen gleichen Zieltext zu erschaffen und fokussiert sich stattdessen auf spezifische Gebrauchssituationen. \\
	Die einflussreichsten Vertreter dieser Kategorie sind:
	\begin{itemize}
		\item{Hans J. Vermeer (1978) mit "Ein Rahmen für eine allgemeine Translationstheorie"}
		\item{Hans G. Hönig und Paul Kußmaul (1982) mit "Strategie der Übersetzung"}
		\item{Katharina Reiß und Hans J. Vermeer (1984) mit "Grundlegung einer allgemeinen Translationstheorie"}
		\item{Justa Holz-Mänttäri (1984) mit "Translatorisches Handeln: Theorie und Methode"}
		\item{Christiane Nord (1988) mit "Textanalyse und Übersetzen"}
	\end{itemize}

	Obwohl Reiß/Vermeer und Holz-Mänttäri ihre Forschung zum selben Zeitpunkt geschah, wird Holz-Mänttäris Theorie oft als Weiterentwicklung dessen angesehen. \\
	Ludwig Wittgenstein hat sich ebenfalls mit der Frage, was Sprache ist beschäftigt. Zum Beispiel was das Verhältnis zu Sprache und Wirklichkeit ist. Diese Frage kann mit Humboldt verbunden werden, also in welcher Weise diese Interagieren. Ebenfalls unternahm er philosophische Untersuchungen über die Weise, wie Sprache verwendet wird und wie wir damit umgehen. Dies nannte er Sprachspiel, da der Umgang der Sprache situationsabhängig ist. \\
	Ein weiterer Vertreter ist J.L. Austin mit "How to do things with words" in welcher er über performative Verbstrukturen spricht und den lokutionären und illokutionären Akten. Diese sind einerseits das Versprechen durch den Satz selbst, andererseits der Akt, welcher darauf folgt. \\
	Ähnliche Theorien brachte J.R. Searle mit der Sprechakttheorie hervor. Er nimmt an, dass Sprechen eine "Regelgeleitete Form des intentionalen Verhaltens" ist. Beispiele dieser Sprechakte ist das aufstellen von Behauptungen oder dem erteilen von Befehlen.
	Hierbei gibt es weitere Akte:
	\begin{itemize}
		\item{Der Äußerungsakt - Sprachliches Ereignis}
		\item{Der Propositionelle Akt - Referenz auf Aktion}
		\item{Die Illokutionäre Akt - Modalität, Intention, Behauptung, Vermutung}
		\item{Der Perlokutionäre Akt - Eine der Wirkungen}
	\end{itemize}

	Ein Beispiel dieser verschiedenen Akte ist das word "Passt" im Deutschen, oder wie ausgeführt wird, das finnische Wort "Nonii". So haben diese Worte die selbe Äußerung (Das Wort "Passt" oder "Nonii") können jedoch jeweils andere Illokutionäre Akte haben und somit andere Sachen benennen. \\
	\textbf{Hans J. Vermeer} führte den Begriff des \textbf{Skopos} ein. Dieser beschreibt, dass der Zweck der Translation die Form der sprachlichen Handlung bestimmt. Das Hauptkriterium hierfür ist die Adäquatheit der Translation in der gegebenen Kommunikativen Situation. \\
	\textbf{Hans G. Hönig und Paul Kußmaul} wiederum fassten Kommunikation in der Soziokultur, welcher die Wissensvorraussetzungen prägt. Sie sagten:
	\begin{quote}
	"Nicht die Sprache entscheidet, was man sagen kann und sagen soll, sie stell lediglich das nötige Material bereit." (Hönig/Kußmaul 1984:44) \\
	"Jeder Text kann als der verbalisierte Teil einer Soziokultur verstanden werden. Es ist unmöglich, ihn aus dieser Einbettung zu lösen, wenn man nicht weiß, zu welchem Zweck dies geschehen soll." (ibid.:58)
	\end{quote}

	Ein weiterer Grundstein ist der "Notwendige Grad der Differenzierung". Es soll nicht so genau wie möglich sein, sondern so exakt wie nötig. \\
	Ein Beispiel ist das Buch "Die spinnen, die Finnen". Ein humoristisches Werk über einen Deutschen, welcher in Finnland lebt und behandelt wie man ein Finne wird. Ursprünglich im Deutschen, wurde es schließlich auch ins Finnische übersetzt. Bei der Übersetzung sind jedoch nicht so viele Hintergrundinformationen notwendig, da der Durchschnittsfinne bereits darüber Bescheid weiß. \\
	Anspielungen können auch eine unterschiedliche Differenzierung benötigen. Wenn man zum Beispiel eine Allusion hat, welche in einer Kultur nicht allgemein bekannt ist, in einer anderen jedoch schon, muss man diese Extrainformation nicht explizit hinzufügen. \\
	Ebenfalls muss man zwischen \textbf{High und Low Context cultures} unterscheiden. In einer High-Context Culture wird viel Information erwartet, wodurch Information größtenteils implizit erbracht wird. In einer Low-Context Culture ist die Erwartung von Information viel geringer, wodurch erbrachte Information auch expliziter sein muss. \\
	Reiß und Vermeer besprechen das \textbf{Primat der Intentionalität}, dass zuallererst der Zweck der Übersetzung gegeben sein muss. \\
	Diese Beschreibung, dass "Der Zweck alle Mittel heiligt" ist ein Rückblick auf die Ansichten Martin Luthers zwecks seiner Bibelübersetzungen. \\
	Zusammengefasst entthront Vermeer den Ausgangstext und die Erwartung, dass dieser sprachlich so genau wie möglich übereinstimmt. Der Ausgangstext ist nur ein Informationsangebot und der Übersetzer verwendet dieses Angebot um ein eigenes Werk mit dem möglichst selben Angebot zu erschaffen. \\
	So ist die Äquivalent nur mehr von zweitrangiger Bedeutung, wobei stattdessen die Adäquatheit vorherrscht und die Übereinstimmung von Translat und Kommunikationsziel am wichtigsten ist. Diese Texte und Botschaften bekommen ihre Bedeutung dann erst, wenn sie gelesen werden und das individuelle Vorwissen eines Lesers eingebracht wird. \\
	Ein Beispiel dessen ist eine finnische Werbekampagne über "XPress-On Covers" für ihre Handys. Der finnische Slogan "Kullekin omansa" (jedem sein Eigenes) wurde als "Jedem das Seine" übersetzte. Dies ist der Spruch, welcher am Eingangstor des Konzentrationslagers Buchenwald hängt, und führte zu heftiger Kritik an Nokia, welcher den Slogan prompt zurücknahm. \\
	Also ist Verstehen strikt Kulturgebunden. Ein Translat muss mit dem Wissen, wie es in der Ausgangssprache aufgefasst wird und wie es in der Zeilsprache verstanden wird, erschaffen werden. \\
	Eine Anekdote des Herrn Kujamäki ist, dass, als er in der DDR seine Diplomarbeit schrieb, er von Freunden zum Essen eingeladen wurde. Da in Finnland Kaffee und Schokolade lange als Luxusgut galten, hatte es sich dort etabliert, dass man diesen als Geschenk bei einem ersten Besucht mitbringt. So brachte er der Familie eine Packung Kaffee mit, welche er aus Finnland mitgebracht hatte. Die Frau seines Freundes hat dies jedoch so interpretiert, dass er dachte, dass sie keinen Kaffee zuhause haben und erklärte ihm, dass sie von Freunden in der BRD mit Kaffee versorgt werden. Was er als respektvollen Akt sah, wurde in der DDR als Affront gesehen. \\

	\section{Translation als interkulturelles, kommunikatives Handeln - 04.04.2022}

	Justa Holz-Mänttäri schreibt, dass Texte als Botschaftsträger handeln sollen, und somit die translatorische Handlung Fallspezifisch behandelt werden muss. So steht nun die Frage, in welcher Situation und zu welchem Zweck die Übersetzung steht. 
	\begin{quote}
	"Spezifikationen sind somit ein fester Teil der Textbestellung und auch Teil des Vertrags zwischen Bedarfsträger und Produzent." - (Holz-Mänttäri 1986:351-2).
	\end{quote}
	Also ist der Ausgangstext nur ein Element, und in diesem Modell bestehen Botschaftsträger. Ebenfalls hebt Holz-Mänttäri hervor, dass das Übersetzen eine Expertentätigkeit ist, wie auch andere Vorgänge Expertentätigkeiten sind. Das bedeutet, dass bei der Interaktion mit anderen Elementen der Textproduktion, der Übersetzer gleichberechtigt ist und sich nicht unterzuordnen hat. \\
	Anhand von dieser Theorie kreierte sie die "Theorie des translatorischen Handelns" aus 1984. \\
	In diesem Modell beschreibt sie eine \textbf{Suprahandlung}, welche den Zweck des Schaffens zusammenfügt. Darunter geordnet sind die \textbf{Subhandlungen}, welche spezifische Vorgänge zur Vollendung der Suprahandlung sind. Durch diesen Kontext sind Übersetzer nicht mehr in der "dunklen Kammer" eingeschlossen, wo sie vor sich hin übersetzen, sondern ein fester Bestandteil der Textproduktion. \\
	Gleichzeitig bestehen jedoch auch Rollenträger im Allgemeinen und innerhalb des Handlungsfeldes der Translation. Im Allgemeinen sieht sie unter anderem die Ausbildung eines Übersetzers. Innerhalb des Translatorischen Handlungsfeldes besteht gleichzeitig ein Prozess, in welchem mehrere Parteien bestehen:
	\begin{itemize}
		\item{Der Bedarfsträger: Die Partei, welche eine Übersetzung benötigt}
		\item{Der Besteller: Eine Gruppierung innerhalb der Partei, welche spezifische Anforderungen hat}
		\item{Ausgangstextproduzent: Das Originaldokument, welches übersetzt gehört.}
		\item{Translator: Übersetzung der Dokumentation sowie Grafiken}
		\item{Zieltextapplikator: Die Übersetzung wird von u.a. Marketingagenturen in der Zielsprache angewandt.}
		\item{Rezipient: Der Konsument des Translats}
	\end{itemize}

	\subsubsection{Professionalität}
	Innerhalb einer Handlung als Experte ist auch Definition der Professionalität. Hierzu hat der Übersetzer als "Botschaftsträger" mehrere Vorraussetzungen:
	\begin{itemize}
		\item{Umfangreiches Einschlägiges Wissen}
		\begin{itemize}
			\item{Deklaratives Wissen: "Knowing What" - Kulturelles und Länderspezifisches Wissen über eine Kultur oder Region}
			\item{Prozedurales Wissen: "Knowing How" - Das Wissen der Aneignung des Wissens. Z.B. wie man ein Wörterbuch oder einen Glossar verwendet.}
		\end{itemize}
		\item{Bewusstsein über eigenes Kompetenz und ihre Grenzen}
		\begin{itemize}
			\item{Zu wissen, was man kann und auch was man nicht kann. Das Wissen über seine eigenen Grenzen lässt einen besser über seine eigenen Möglichkeiten einschätzen.}
		\end{itemize}
		\item{Reflektiertes und kontrolliertes Handeln}
		\begin{itemize}
			\item{Dass man erklären kann, warum man etwas getan hat und auch in der Lage ist darüber zu reflektieren, wenn man eine Sache nicht optimal gelöst hat.}
		\end{itemize}
		\item{Bewusstsein über eigene Position und Aufgaben im Handlungsgefüge}
		\item{Anerkennung der sozialen Rolle durch andere Beteiligte im Handlungsgefüge}
		\begin{itemize}
			\item{Dass man}
		\end{itemize}
	\end{itemize}
	Netflix und andere Streamingdienste sind notorisch dafür, den Produzenten der Untertitel relativ wenig zu bezahlen wodurch sich ein unglücklicher Trend in den Untertiteln zeigt. So haben Untertitel oft eine fragwürdige Qualität da Personen mit Sprachkenntnissen, aber nicht Übersetzungskenntnissen angestellt werden, da diese billiger sind. \\
	\subsubsection{Multimodalität der Kommunikation}
	Während man mithilfe von Texten oft gut kommunizieren kann, sind ein weiterer Weg der reibungslosen Kommunikation Piktogramme oder andere Bilder. Bei einer Toilette kann man beispielsweise mit einem Bild das selbe erreichen, als man mit einem längeren Text könnte. Holz-Mänttäri sieht auch 4 Aspekte der Kommunikation in dieser Form:
	\begin{itemize}
		\item{Ton}
		\item{Typographie}
		\item{}
	\end{itemize}

	Diese Multimodalität muss jedoch auch aufeinander abgestimmt werden. In einer Masterarbeit, welche von Herrn Kujamäki betreut wurde, wird über die Abweichungen der Abbildungen und der Begleitzeichnungen geschrieben. Diese sind als schwarz dargestellt, danach jedoch nur als graue Männer beschrieben. \\
	Holz-Mänttäris Modell mag als etwas unhandlich angesehen werden, jedoch findet es innerhalb von bspw. Gerichtsverhandlungen rege Anwendung. In Gerichtsverhandlungen sind oft eine Reihe an unterschiedlichen sozialen Rollen vorhanden und es ist wichtig, dass anerkannt wird, dass ein Übersetzer in diesem Kontext gleichberechtigt existiert. \\
	So werden Dolmetscher meist immer noch als "niedriger" angesehen. In einem Textbeispiel wird auf die Bitte einen Satz zu wiederholen, nur mit "Sie müssen es nicht verstehen, übersetzen Sie einfach." geantwortet. \\
	Aufbauend auf Holz-Mänttäris Modell hat Erich Prun\v c das Konzept der \textbf{Translationskultur} erschlossen. Dieses Modell der (derzeit utopischen) demokratischen Translationskultur beschreibt: Kooperation, [[ADD 2 Parts]] und Ökologie.\\
	Ein neueres Beispiel der Anwendung Holz-Mänttäris Forschung ist das \textbf{User-Centered Translations}, welches einen Fokus auf den Zieltext hat, dies jedoch in Anbetracht der heutigen Verwendung von Technologie zu erreichen versucht. \\
	\subsection{Uneinigkeit der Funktionalen Übersetzung}

	Nicht jeder war davon überzeugt, dass die Entthronung des Ausgangstextes kompromisslos sein sollte. Katharina Reiß äußerte ihre Bedenken, dass auch ein gewisser Respekt vor dem Ausgangstext bestehen sollte, damit eine höchstmögliche Kompatibilität besteht. \\
	Auch Christiane Nord wollte dem Ausgangstext nicht alle Relevanz absprechen. Sie verwies auf die beidseitige Loyalität, welche ein Übersetzer sowohl mit dem Autor als auch dem Leser hat, in welchem Fall dieses Prinzip verletzt werden würde. \\
	
	\section{Zwischenbilanz und Descriptive Translation Studies (DTS) - 25.04.2022}
	\begin{figure}[H]
		\centering
		\includegraphics{Images/Zwischenbilanz.png}
		\caption{Eine kurze Zusammenfassung der behandelten Translationstheorien}
	\end{figure}

	\subsection{Descriptive Translation Studies}
	Bei der deskriptiven Übesetzung steht vor allem die Textsorte des Ausgangstextes im Fokus. Zuvor wurden literarische Texte größtenteils ignoriert. Das liegt daran, dass die Interpretation dieser mithilfe der vorherigen Theorien recht schwer fiel. \\
	\textbf{Roman Jakobson}, ein Anhänger des Russischen Formalismus, lehnte den Begriff der Äquivalenz für literarisches Übersetzen grundsätzlich ab. Seiner Meinung nach kann zwischen den Übersetzungen keine Übersetzung stattfinden, sondern viel mehr eine Transposition. \\
	Wichtige Etappen der DTS wurden durch drei Konferenzen begründet: jeweils in Leuven (1976), Tel Aviv (1978) und ANtwerpen (1980). Aus diesem Grund spricht man auch von dem "Israelisch-Niederländisch-Belgischen Zeitalter". \\
	\textbf{James S. Holme} begründete mit dem Artikel
	In Israel und der "Tel Aviv Schule" waren \textbf{Itamar Even-Zohar} und \textbf{Gideon Toury} die wichtigsten Begründer. Even-Zohar weitete dies später auf seine Polysystemtheorie aus. \\
	In Belgien und den Niederländen waren es wiederum \textbf{José Lambert} und \textbf{André Lefevere} mit dem Text \textit{Translation, Rewriting, and the Manipulation of Literary Frame}. \\
	Während zuvor stets davon geredet wurde, dass "Der Zweck die Mittel heiligt", sprach man nun gar von "Manipulation" zur Erreichung des literarischen Ziels. Dies wird dadurch begründet, dass es stets die Sprecher des Zieltextes sind, welche eine Übersetzung haben wollen und sehr selten der Autor oder Sprecher der Ausgangssprache. \\
	\begin{wrapfigure}{l}{0.5\textwidth}
	\includegraphics[scale=0.5]{Images/Holmes.png}
	\end{wrapfigure}
	Ein Ergebnis dieser Modelle war \textbf{Holmes' Map of Translation Studies}, in welchem zwischen 'Angewandte' und 'Reine' Übersetzung unterschieden wird. Reine Übersetzung wird dabei wiederum in 'Theoretische' und 'Deskriptive' Übersetzungen unterschieden. \\
	Das Hauptmerkmal dieses neuen Systems ist, dass es anstatt präskriptiv zu sein, nun deskriptiv ist. \\
	Da ein kulturelles System (Wie z.B. Comics oder Krimis) in anderen Ländern einen anderen Stellenwert haben können, muss das in den Übersetzungsvorgang miteingeschlossen werden. \\
	Eine Übersetzung ist stets eine \textbf{zielkulturelle Tatsache}, also ist eine Übersetzung dann eine, wenn diese von der Zielkultur als solche angenommen wird. Also sind Äquivalenzbeziehungen zwischen den beiden Texten davon abhängig. Wie und wann diese Äquivalenz zutrifft, hängt von kollektiven sozialen Normen ab. Normen sind in diesem Kontext eine Struktur von Erwartungen der Gesellschaft im Bezug auf Translation und deren Produkte (Translate). \\
	\subsubsection{Polysystemtheorie}
	Die Systemtheorie besagt, dass literarische Phänomene ein organisiertes Netzwerk von Beziehungen zueinander und zu außerliterarischen Elementen wie Verlagshäuser, Agenten, politische Strukturen oder Ideologien bilden. Diese Systeme bestimmen dabei die Funktionen literarischer Texte und ihrer Übersetzungen innerhalb des Systems. \\
	Ein banales Beispiel dieser Theorie könnte sein, dass es in einem Kindergarten ein komplexes System an Normen und Gegebenheiten gibt, mit welchen die Kinder agieren. Diese spielen auch gleichzeitig Spiele und haben eine allgemeine Auffassung davon, wie diese Spiele aussehen und ablaufen. Nun könnte ein weiteres Kind in diesen Kindergarten kommen aber aus Finnland stammen und deshalb nicht mit den vorherrschenden Ideen und Normen vertraut sein. Jedoch bringt dieses Kind seine eigenen, in diesem Fall fremden, Vorstellungen mit und schafft so ein gewisses Chaos da die Normen miteinander kollidieren. Jedoch kann, wenn dieses Chaos erst einmal überwunden ist, zu neuen Errungenschaften innerhalb eines gewissen Feldes führen. In diesem Fall ist der Kindergarten ein bestehendes System und das finnische Kind ist eine Übersetzung, welches in das System eingebracht wird. \\
	Laut Even-Zohar haben Übersetzungen in diesem Kontext eine besondere Bedeutung falls:
	\begin{itemize}
		\item{Gerade der Prozess der Entstehung neuer Literaturen besteht}
		\item{Wenn Literatur in der Gesellschaft nur eine schwache Position hat}
		\item{Währen Krisen oder "lierarischen Vakuuen"}
	\end{itemize}

	Ein literarisches Vakuum besteht, wenn die Literatur eines Systems "abgestanden" ist und nur noch wenig Innovation besteht. \\
	\subsection{Normen}
	Bei der Definition von Normen gibt es zwei Ansätze: Top-Down und Bottom-Up.
	[[TODO NORMEN]]

	\section{Normenkonzept - 02.05.2022}
	\subsection{Normen}
	\textbf{Gideon Toury}, Teil der Israelischen Schule, befasste sich eingehends mit Normen und Normenkonzepten bzw. welche Rolle diese einnehmen. So fasste er die Normen in drei Kategorien zusammen:
	\begin{itemize}
		\item{Initialnormen}
		\begin{itemize}
			\item{Adäquatheit sowie Akezeptabilität im Bezug auf sprachliche und zielulturelle Normen}
			\item{Ist die Orientierung am Ausgangstext angemessen? Ist es angemessen auf sprachliche und kulturelle Normen Bezug zu nehmen?}
		\end{itemize}
		\item{Vornomen/Vorbereitende Normen}
		\begin{itemize}
			\item{}
		\end{itemize}
		\item{Operativnormen}
		\begin{itemize}
			\item{Matrixnormen: Wird die Struktur des Textes beibehalten oder eher an der Zielkultur orientiert.}
			\item{Textlinguistische Normen: Kann man die Sprache des Textes beibehalten, oder soll man sich an der ZK orientieren?}
		\end{itemize}
	\end{itemize}

	\textbf{Andrew Chesterman} erstellte sein eigenes Normenkonzept:
	\begin{itemize}
		\item{Social Norms}
		\item{Ethical Norms}
		\item{Technical Norms}
		\begin{itemize}
			\item{Produktbezogene Normen: }
			\item{Prozessbezogene Normen: Vertrauen sowie Loyalität der Übersetzer. Ebenfalls optimalität der Translation für kommunikative Ziele}
		\end{itemize}
	\end{itemize}

	Theo Hermans bezeichnet Normen ingesamt als \textbf{starke präskriptive (vorschreibende) Formen von sozialen Konventionen}. \\
	\subsubsection{Strenge:}
	Man kann zwischen relativen und absoluten Normen unterscheiden:
	Absolute Normen sind gesetzte und werden durch Richtlinien und Gesetze unterstützt, wie das Recht auf einen Dolmetscher für Migranten. \\
	Relative Normen regulieren sich hingegen selbst und werden oft durch ungeschriebene Konventionen etabliert. \\
	\subsubsection{Reichweite}
	Bei der Reichweite gibt es allgemeine Normen, welche über ein großes Gebiet in Kraft sind, und Idiosynkratische Normen, welche eventuell nur von einer einzige Person angewandt werden. \\

	\subsection{Cultural Turn}
	Mit dem Cultural Turn begannen Übersetzungen, weniger spezifisch auf dessen Funktion in der Originalsprache einzugehen und stattdessen auf das Produkt selbst, welches durch die Translation entsteht. So waren Translate nicht mehr ein Medium, welches eine fremde Kultur näherbringt und stattdessen ein Medium, welches einen gewissen Zweck erfüllt. \\
	Translation konstruiert Kulturen. So haben diese ein manipulatives Potential indem sie einerseits die Kultur selbst beeinflussen, als auch ein Bild einer "Fremdkultur" bildet. \\
	\textbf{Susan Bassnet} und \textbf{Andre Lefevere} erforschten die potentielle manipulative Wirkung. Sie sehen Übersetzungen stets als \textbf{doppelt kontextualisiert}, da ein Text damit stets innerhalb zweier Kulturen steht. \\
	Sie stellen fest, dass Übersetzungen einen großen Einfluss auf die Bildung von Weltkultur hatten und haben und nicht ohne diese existieren kann. \\
	Nach Lefevere existiert das literarische System, welches Einfluss darauf hat wie Kunst, wie literarische Werke entstehen können. Jedoch existieren auch Einflüsse, welche außerhalb dieses Systems bestehen:
	\begin{itemize}
		\item{Ideologie}
		\begin{itemize}
			\item{Das Wertekonzept und gesellschaftliche Normen, Konventionen und Idealen}
		\end{itemize}
		\item{Patronage}
		\begin{itemize}
			\item{Die Macht, mit welcher Institutionen das Lesen, Schreiben oder das "Rewriting" beeinflussen. Es besteht aus drei Komponenten:}
			\item{Die ideologische Komponente: In welcher Weise das literarische und soziale System miteinander agiert.}
			\item{Ökologische Komponente: Der Patron sichert dem Künstler den Lebensgrundsatz und kann diesen fördern, oder ausschließen}
			\item{Machtkomponente: Die Stellung, welche das literarische System inne hat.}
		\end{itemize}
	\end{itemize}
	Die sogenannte "Poetik" reguliert ebenfalls, wie das System agiert, indem es die Auffassung von "guter" Literatur festlegt. Ebenfalls ist das Wirklichkeitsmodell um das System wichtig, indem es festlegt worüber man überhaupt reden oder schreiben darf, nicht darf oder sogar muss. \\
	Der letzte Einfluss auf das System ist das "Rewriting", mit welchem Originalwerke neu geschrieben und auch manipuliert werden. So nehmen Kritiker, Satiriker oder Rezensenten diese Literatur und verändern oder verfälschen diese eventuell. Zusätzlich zu solchen professionellen Rewritings, wo auch Übersetzungen mitgezählt werden, gibt es auch inoffizielle rewritings wie Fanforen. \\
	Toury untersuchte solche Normen indem er das Stück \textit{Sonnets to a Young Man} auf Hebräisch 1995 interpretierte. Zu diesem Zeitpunkt war es nicht erlaubt über den Inhalt des Stücks, welches über die homosexuelle Liebe zweier Männer, öffentlich zu reden. \\
	Hr. Kujumäki brachte auch ein Beispiel vor indem er die deutsche Übersetzung des Buches \textit{Der Tochter der Ostsee} von Maila Talvio, vorlas, in welchem die Tochter eines finnischen Kaufmanns einen Juden heiratet. Das Buch, welches 1939 erschienen war, wurde schließlich in Deutschland mit einem Extrasatz veröffentlicht, welcher die Heirat als "unverzeilichen Fehler" verurteilt. \\
	Bassnet und Lefevere schließen das Thema mit dem Satz, dass \textbf{Neuschreibung eine Manipulation im Sinne der Macht sei} ab. \\
	Lawrence Venuti spricht davon, dass Übersetzungen immer ein domestizierender Prozess sind. Darin sieht er den Grund, weshalb Übersetzungen oft mit Misstrauen angesehen werden. \\
	Nach Venuti kann man die Manipulation in drei Phasen teilen:
	\begin{itemize}
		\item{Textauswahl}
		\begin{itemize}
			\item{Es werden thematische und inhaltliche Erwartungen gegenüber dem Zielpublikums getroffen}
			\item{Ebenfalls werden ideologische und poetische Auffassungen der Zielkultur angesehen}
			\item{Zusätzlich fließen auch Erwartungen der Verlagshäuser ein}
		\end{itemize}
		\item{Während des Übersetzens}
		\begin{itemize}
			\item{Eine unvermeidliche Anpassung an denotative, konnotative und diskursive Strukturen der Zielkultur und Sprache. Ein Beispiel hierfür wäre wieder die Nokiawerbung mit "Jedem das Seine"}
		\end{itemize}
		\item{Nach dem Übersetzen, bei der Verbreitung}
		\begin{itemize}
			\item{}
		\end{itemize}
	\end{itemize}
	Durch diesen langen Prozess, in welchem stets Einflüsse auf die Übersetzung einwirken, ist die Leserschaft immens von diesen Entscheidungen abhängig. So sind die Entscheidungen von:
	\begin{itemize}
		\item{Agenturen}
		\item{Büchermessen}
		\item{Verlagshäusern}
		\item{Verlagslektoren}
		\item{Übersetzer}
		\item{Kritiker}
		\item{Eltern}
	\end{itemize}
	abhängig, ob ein Kind ein Buch liest. So haben Normen direkt und indirekt einen Einfluss auf unsere Lesegewohnheiten. \\
	Venuti ist bei der Analyse von ins Englische übersetzer Japanischer Literatur zu dem Schluss gekommen, dass in den USA eine verzerrte und auch nostalgische Sicht auf japanische Kultur geschaffen wurde. So zeigte die in den USA veröffentlichte Literatur ein Bild von Samurai und Kirschblüten. Diese Selektionen wurden dann jeweils auch nach Europa exportiert und bildeten auch dort die Meinung über die japanische Kultur. Dies geschah, weil viele Übersetzer in bspw. Europa nur Übersetzungen aus dem Englischen vornehmen konnten und keinen Zugriff auf die Originalversion hatte. Dadurch wurden die Anglo-Amerikanischen Normen indirekt auch in Europa verbreitet.\\
	Die finnische Tradition der Übersetzung begann erst wirklich mit der Unabhängigkeit des Staates 1918. Vor dem 19. Jahrhundert waren Übersetzungen ins finnische nahezu nur biblischer oder erzieherischer und aufklärerischer Natur. Erst bei der Hälfte des 19. Jh. begannen andere Texte aufzutauchen. Nachdem Finnland mit Ende des zweiten Weltkrieges schließlich seine eigene Nation wurde, wurden im großen Stil fremdsprachige Werke übersetzt um eine eigene literarische Kultur zu schaffen. \\
	\textbf{Lawrence Venuti} nimmt an, dass die Unsichtbarkeit der Urheber der Übersetzung, meist die Schuld der Übersetzer selbst ist, da die \textit{fluid translation} (flüssige Übersetzung) der Leserschaft keinen Grund gibt sich mit der Tätigkeit des Übersetzens auseinanderzusetzen. Er schlägt vor, dass Übersetzungen sogenannte "Stolpersteine" enthalten sollten, welche die Fremdheit des Ausgangstextes wiederspiegeln sollten. Vor allem im Anglo-Amerikanischen Raum gibt es sehr wenig Offenheit für "fremdes" wodurch von Übersetzungen erwartet wird, dass sich diese anpassen. Gleichzeitig hebt er jedoch hervor, dass andere Kulturen sehr wohl offener für fremdes in ihrer Literatur sind. \\
	\section{Postcolonial Studies im Cultural Turn - 9.05.2022}
	Postcolonial Studies stellt eine Neuausrichtung der Ansichten über kulturelle Gegebenheiten. Währen man währen des Kolonialismus die europäische Kultur als "Norm" sah und jede andere fremd und exotisch war, versuchen postcolonial studies einen objektiveren Blickwinkel zu erzeugen. Sprache stellte während des Kolonialismus ebenfalls ein Werkzeug der Einflussnahme auf die Kolonien dar, indem diesen die Sprache aufgezwungen wurde. \\
	Wichtige Vertreter der postcolonial studies sind \textbf{Edward Said} mit \textit{Orientalism}, sowie \textbf{Gayatri Spivak}, \textbf{Homi Bhabha} und \textbf{Doris Bachmann-Medick}.\\
	Homi Bhabhba führte den Begriff des "dritten Raums" ein, welcher als Zwischenraum der Kulturen gesehen werden kann, in welchem die Erwartungen der neuen Kultur mit der alten vermischt werden, anhand welcher man sich verhält.
	Die Hauptanliegen der postkolonialen Translationswissenschaften sind:
	\begin{itemize}
		\item{Das Zeigen der Mechanismen und der Kraft welche Translationen auf Kulturen und Kolonien hatten und haben.}
		\item{Das Zeigen der kulturellen Ungleichheit zwischen Kolonialmacht und Kolonie.}
	\end{itemize}
	\textbf{Tejaswini Niranjana} zeigt in ihrem Buch \textit{Siting Translation} wie Übersetzungsstrategien dazu führten, dass Indien als das "exotische fremde" dargestellt wurde, was die Manipulierbarkeit der Bevölkerung ermöglichte. In einem anderen Beispiel zeigt sie ebenfalls, dass indische Rechtstexte in ihrer englischen Übersetzung oft den erwarteten status quo des Britischen Reiches widerspiegelte, anstatt nur eine Übersetzung des Textes zu sein. Sie fordert hierbei, dass Übersetzungsstrategien in Betracht ziehen, in welcher Weise Übersetzungen die  Verbindung zwischen zweier Kulturen beeinflussen können. \\
	In diesem Kontext schreibt Prun\v c, dass Translation als Repräsentation des Fremden nicht in einem Vakuum stattfindet, wodurch die Unparteilichkeit der Übersetzungen in Frage gestellt werden. \\
	\subsection{Translationsprozessforschung}
	Hermeneutik ist die Wissenschaft der Textinerpretation als kreativer Prozess. Die Hermeneutik hebt hervor, dass die Perspektive der lesenden Person wichtig für das Textverständnis ist. So bestimmt die Perspektive stets die Interpretation. \\
	Begriffe innerhalb der Hermeneutik sind:
	\begin{itemize}
		\item{Multiperspektivität: Der Umstand, dass jeder andere Erfahrungen hat, wodurch nicht die eine Interpretation die richtige ist.}
		\item{Übersummativität: Ein Text kann in mehreren, voneinander abweichenden, Wegen interpretiert werden, welche je von den Erfahrungen des Lesers abhängen.}
		\item{Horizontverschmelzung: Wenn außenstehende Information mit dem eigenen Wissen zusammenkommt um Information interpretieren zu können, während das Fehlen eines der Elemente eine korrekte Interpretation verhindert.}
	\end{itemize}
	Wichtige Vertreter der Hermeneutik sind:
	\begin{itemize}
		\item{Friedrich Schleiermacher}
		\begin{itemize}
			\item{Verknüpfte erstmals die Psyche des Autors mit der historischen Wahrheit. So war der Autor selbst auch Teil der Interpretation des Textes}
		\end{itemize}
	\end{itemize}
	Aus der Hermeneutik entstand danach, nach empirischer Orientierung, die \textbf{Translationsprozessforschung}, welches kognitive Prozesse beim Übersetzen und Dolmetschen zu erforschen versucht. \\
	Oft wird die Kognitionswissenschaft mit der Translationsprozessforschung gleichgesetzt
	\textbf{Hanna Risku} schrieb hierbei, dass die Kognitionswissenschaft zu erklären versucht, wie Menschen und unter anderem auch Tiere, denken, lernen oder verstehen. So versucht es \textit{Wahrnehmung, Gedächtnis, Problemlösung oder Handeln} zu erforschen. \\

	
	


	



	   
























	
\end{document}