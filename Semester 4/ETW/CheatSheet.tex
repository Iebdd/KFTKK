\documentclass{article}

\usepackage[landscape]{geometry}
\usepackage{makecell}
\usepackage{array}
\usepackage{multicol}
\usepackage[ngerman=ngerman-x-latest]{hyphsubst}
\usepackage{setspace}
\usepackage{changepage}
\usepackage{booktabs}
\usepackage{graphicx}
\usepackage{float}
\newcolumntype{?}{!{\vrule width 1pt}}
\renewcommand\theadalign{tl}
\setstretch{1.10}
\setlength{\parindent}{0pt}

\geometry{top=12mm, left=1cm, right=2cm}
\title{\vspace{-3cm}21S 520.230 Deutsch: Mutter-/Bildungssprache: Textanalyse und Textproduktion Gruppe 3}
\author{Andreas Hofer}

\begin{document}
	\begin{tabular}{| l | l |}
		\toprule
		Name & Disziplin \\ \midrule
		J. Lei Weisgerber & Eigentümlichkeiten gewisser Sprachen -> Sapir-Whorf-Hypothese \\ \hline
		Zeichentheorie (de Sassure) & \makecell[l]{ Langue (Sprachinventar) \\ Parole (Tatsächliche Äußerungen) \\ Zeichen (Wörter der Sprache)} \\ \hline
		Werner Koller & Äquivalenzen (1:1, 1:Viele, Viele:1, 1:0, 1:1/2) \\ \hline
		Vinay und Darbelnet & Redwendungen (Wörtlich Substitutierend/ Nichtwörtlich Paraphrasierend) \\ \hline
		J.C. Catford & Translation Shifts \\ \hline
		Eugene Nida \& Charles Taber & Dynamische Äquivalenz und Übersetzungsmethode in drei Stufen (Analyse, Transfer, Synthese) \\

		\bottomrule
	\end{tabular}
	
\end{document}