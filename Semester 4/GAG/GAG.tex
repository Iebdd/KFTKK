\documentclass{article}

\usepackage{geometry}
\usepackage{makecell}
\usepackage{array}
\usepackage{multicol}
\usepackage{setspace}
\usepackage{changepage}
\usepackage{booktabs}
\usepackage{wrapfig}
\usepackage{float}
\usepackage{graphicx}
\newcolumntype{?}{!{\vrule width 1pt}}
\renewcommand\theadalign{tl}
\setstretch{1.10}
\setlength{\parindent}{0pt}
\graphicspath{ {./Images/} }

\geometry{top=12mm, left=1cm, right=2cm}
\title{22S GSB.02005UB Grundprobleme der alten Geschichte}
\author{Andreas Hofer}

\begin{document}
	\section{Einführung - 03.03.2022}
	\section{Der Alte Orient - 10.03.2022}
	Der geographische Raum des alten Orients umfasst drei größere Gebiete, in welchem sich die Teile abspielen:
	\begin{itemize}
		\item{Das Zweistromland zwischen Euphrat und Tigris. Mesopotamien heißt Land zwischen den Zwei Flüssen}
		\item{Das Nilland und das alte Pharaonenreich. Dies wird auch durch die Ägyptologie vertreten}
		\item{Das Gebiet der heutigen Türkei. Im wissenschaftlichen Bereich wird es auch \textbf{Kleinasien} genannt.}	
	\end{itemize}
	In Einheit 3 wird auch noch die Ägäiswelt der Minoischen und Mykonischen Kultur besprochen. \\
	Vom Skript ausgehend wird der Prüfungsrelevante Stoff auf die Zeit bis zum ausgehenden 2. Jahrtausend vor Christus beschränkt. \\
	Die Kapitel 6, 7 und 8 über Neuassyrisches Reich, Achämenidenreich und Spätzeit von Ägypten wird in einem späteren Zeutpunkt behandelt. \\
	\subsection{Mesopotamien}
	Die geographischen Vorraussetzungen Mesopotamiens sind außergewöhnlich besonders, da das fruchtbare Schwemmgebiet der beiden Flüsse zur Entstehung der ersten Hochkultur geführt hat. \\
	Diese fruchtbaren Böden waren in dieser Region besonders wichtig, da es in einem recht niederschlagsarmen Gebiet liegt. Das führt auch heute noch zu großen Geopolitischen Problemen, welche den Wasserreichtum der Flüsse kontrollieren wollen. Diese Machtdynamik ist auch in Ägypten vorhanden. \\
	Im Gegensatz zum landwirtschaftlichen Reichtum in der Region ist es sehr arm an Bodenschätzen, wodurch der Handel in der Region stark ausgeprägt war. \\
	Diese Umstände führten zu einer sehr frühen Urbanisierung in dieser Region. Zusätzlich existierten auch oft Stadtstaateb, mit ausgeprägter Aristokratie. Dadurch wurde die Rivalität zwischen den Städten angefacht, was wahrscheinlich zu einer beschleunigten Wissenschaftsaneignung geführt hat. \\
	Einer der größten Stadtstaaten war die Stadt von \textbf{Uruk} und stellt den ersten Höhepunkt der Machtausdehnung dieses Gebiets da. Dieser Höhepunkt fand im dritten Jahrtausend vor Christus statt. \\
	Eine wichtige Entwicklung in dieser Zeit war die Einführung der Keilschrift, welche schon ab dem 4. Jahrtausend verwendet wurde. \\
	Diese Schrift wurde von mehreren Sprachen verwendet und ist nicht nur auf eine Sprache bezogen. Die Schrift war größtenteils als Verwaltungssprache verwendet um Sachen zu katalogisieren. Trotzdem gab es schon eine literarische Verwendung dessen. Dies war jedoch mehr Propagandaschrift und zur Machtfestigung gedacht, als für Prosa oder Lyriktexte, welche zu diesem Zeitpunkt noch nicht existierte. \\
	Uruks Dominanz wich im frühen 3. Jt. für eine polyzentrische Struktur, in welcher mehrere Städte eine eigene Position einnahmen, jedoch immer noch in Konkurrenz standen. \\
	Es ist emblematisch für Geschichte Macht zu zentralisieren und es gibt stets Beispiele, in welcher Staaten oder Stadtstaaten in der Lage waren ein Gebiet zu beherrschen. \\
	Das zweite größere Reich ist das Reicht von \textbf{Akkad} welches ab etwa 2340 in der Region dominierte. Diese Dominanz bestand für nahezu 200 Jahre. Akkad vergrößerte sein Reich jedoch in den Norden bis in das heutige Syrien. Die Ausdehnung führt in den Norden bis nach \textbf{Subartu} bis \textbf{Elam} in den Osten. Diese Expansion fand unter dem Herrscher Sargon statt, welcher ein weit größeres Heer von bis zu 5000 Mann hielt. \\
	200 Jahre mag in Antiken Maßstäben nicht wie viel wirken, ist jedoch für moderne Verhältnisse eine recht lange Zeit. \\
	Das dritte große Reich war das Reich von \textbf{Ur III}. III steht für die dritte Dynastie von Ur.\footnote{f.d steht für Frühdynastisch} \\
	Ur III fand seine größte Ausbreitung um etwa 2000 v. Chr. und ungefähr 100 Jahre anhielt. \\
	Das letzte große Reich dieser Region war das Reich von \textbf{Babylon}, welches kurz nach Niedergang des Reiches von Ur seinen Höhepunkt fand. Babylon hatte eine ähnliche Ausdehnung wie das Reich von Akkad seinerzeit. Diese Ausdehnung geschah wiederum unter dem Herrscher \textbf{Hammurabi} von Babylon. Hammurabi ist zudem für eine Einführung der Keilschrift als kulturelles Gut verwantwortlich. Die \textbf{Stele von Hammurabi} ist eine 2 Meter große Stele, ausgestellt im Louvre, welche einen Teil des Codex von Hammurabi darstellt. Der Codex von Hammurabi und andere gefundene Texte zeugen von einem ausgeprägten Rechtstext, was ein Indiz einer Hochkultur ist. \\
	Die Keilschrift auf der Stele ist ansatzweise erahnbar und die unterschiedlichen Rechtsteile, welche man finden kann beinhalten das:
	\begin{itemize}
		\item{Liegenschaftsrecht}
		\item{Schuldrecht}
		\item{Erbrecht}
		\item{Sklavenrecht}
		\item{Eherecht}
	\end{itemize}

	Das Babylonische Reich von Hammurabi bestand von 1790 v.Chr. bis 52 v.Chr. und stellt das längstwährende Reich der Region da. Obwohl spätere Herrscher schwächer waren als Hammurabi selbst, konnte es sich einige Zeit etablieren. \\
	Babylon wurde letztendlich von den Hethitern aus der heutigen Turkei erobert. \\
	Die Religion stellte in dem Gebiet auch immer wieder die Legitimation der Herrscher da. \\
	Durch die Verwaltungsschrift innerhalb der Stadtstaaten führte auch zu einem ausgeprägten Beamtenapparat, welcher später auch in Ägypten eine wichtige Rolle spielte. \\
	Durch die Übernahme Babylons nahm das \textbf{Hethitische Reich} eine Vormachtstellung ein und dehnte sich bis in das heutige Kairo aus. \\
	Zu dieser Zeit baute das Reich auch die Hauptstadt \textbf{Hattu[[ADD s with thingy]]a} signifikant aus. \\
	Die Hethiter führten auch mehrer kriegierische Auseinandersetzung mit den antiken Pharaohnen um ihre Macht auszudehnen. \\
	Bei den Hethitern war es jedoch wiederum lediglich eine militärische Expansion, wodurch auch die Herrscherposition und Legitimation von der Expansion abhängig wurde. \\
	Im Gegensatz zu den Babylonern, wo der Herrscher durch eine Aristokratie gestützt wird, war im Hethitischen Reich der Herrscher selbst viel mächtiger. \\
	Ebenfalls hatte die Gemahlin des Hethiterherrschers eine ausgeprägte Funktion, wodurch sie selbst in der Lage war Verträge mit anderen Nationen zu unterzeichnen. \\
	\subsection{Ägyptisches Reich}
	Ägypten, obwohl als Territorialstaat angenommen, ist und war früher auch fast nur das Reich des Nils, also die Bevölkerung begrenzte sich fast ausschließlich auf die Gebiete unmittelbar um den Nil. \\
	Die Griechen bezeichneten die Ägypter als Geschenk des Nils, obwohl sie nicht wussten woher die Überschwemmung des Nils stammt. Diese bracht stets Nährstoffe und Sedimentgestein mit sich, was das Schwemmland sehr fruchtbar machte. \\
	Der Umstand, dass das Ägyptische Reich von Wüste umgeben war, war auch eine hevorragende Verteidigung gegenüber einfallender Mächte, da jede Armee zuerst die Wüste durchqueren musste, und die meisten Herrscher ihre Armeen gar nicht über einen solchen Zeitruam ernähren konnten. \\
	Das Ägyptische Reich wurde über seine lange Geschichte trotzdem öfters erobert:
	\begin{itemize}
		\item{Zuerst durch das Persische Reich}
		\item{Durch die Griechen unter Alexander dem Großen}
		\item{Durch die Römer}
		\item{Durch Moslemische Mächte}
	\end{itemize}

	Ägypten kann in drei Zeitabschnitte geggliedert werden:
	\begin{itemize}
		\item{Das Alte Reich: 2650 - 2160 v.Chr.}
		\item{Das Mittlere Reich: 2025 - 1650 v.Chr.}
		\item{Das Neue Reich: 1550 - 1070 v. Chr.}
	\end{itemize}

	Zusätzlich gab es zwischen den beiden ersten Dynastien jeweils zwei Zwischenzeiten, in welcher keine größere Macht eine Prägung hatte. Ebenfalls wird nach dem Neuen Reich die Spätzeit definiert, welche von 1070 - 525 v. Chr. dauerte. Zu dieser Zeit gab es einige schwächere Pharaohnen, welche über Ägypten herrschten. \\
	
	Ungefähr zur selben Zeit als die Keilschrift in Mesopotamien enstand (Die Keilschrift wird als etwas älter angesehen), fand auch die Ägyptische Schrift ihren Ursprung. Für lange Zeit galt das Ägyptische als nicht lesbar, jedoch wurde während der Zeit Napoleons der \textbf{Stein von Rosetta} gefunden, welcher in Ägyptisch sowie Altgriechisch geschrieben war. Nach etwa 20 Jahren, im Jahre 1822, wurde das Ägyptische schließlich vollständig entziffert und brachte das Fach der Ägyptologie hervor. Ebenfalls führte es zu einem Boom des Interesses in der westlichen Elite für die Ägyptische Zeit. \\
	
	Die staatliche Ordnung in Ägypten wurde bereits sehr früh durch den Pharaoh geprägt, welcher der Herrscher und Eigentümer des Landes ist. Unter diesem Pharaoh war das Reich geeint. Der Pharaoh hatte, nach heutigen Maßstäben, eine absolutistische Macht und wurde durch Erbfolge geregelt. Später war der Pharaoh auch durch Gott gesandt und erhielt so seine Legitimation. \\
	Dies wurde durch die zentrale Religion des Reiches ermöglicht, welcher ein wichtiger Teil der Gesellschaft war. \\
	Der Staat ist sehr zentralistisch organisiert um einiges strikter als es im Hethitischen Reich der Fall war. Die Befehlskette war hierarchisch von oben nach unten geregelt, wodurch ein gemeiner Bürger praktisch keinen Einfluss auf die Geschehen hatte. \\
	Jeder Abschnitt des Landes war in Gaue gegliedert, welcher durch einen Statthalter verwaltet wurde und wieder seinen eigenen Verwaltungssitz hatte. \\
	Ebenfalls war die Besteuerung der Bevölkerung ein zentraler Bestandteil des Reiches. Die Steuern waren stets an die höhere soziale Position abzugeben und von dort weiter verteilt. \\
	Die Legitimation dieses Machtapparats und Rechtfertigung für Steuern war die Sicherheit und Rechtsordnung innerhalb der Region. Dies war einerseits durch die Religion, aber auch durch die militärische Macht gewährt. \\
	Wirtschaftlich war Ägypten außerordentlich effektiv, was durch die Konstruktion der Pyramiden ersichtlich wird, welche ein enormer menschlicher und wirtschaftlicher Aufwand waren. Die Pyramiden waren gigantische Grabbauten, wobei die Cheops-Pyramide einen Grundriss von 200x200 Meter maß und 150 Meter hoch war. \\
	Das Ende des Alten Reiches wurde durch eine wachsende Macht der regionalen Eliten eingeleitet, was wiederum den Pharaoh schwächte. Schließlich kam es mit dem mittleren Reich zu einer neuen zentralen Machtstruktur. \\
	Die zweite Zwischenzeit war wiederum durch Machteinflussnahme von außen geprägt, in welcher das Reich der Hyksos über Ägypten herrschte. Mit dem Neuen Reich kam es jedoch schließlich wieder unter eigene Kontrolle. \\
	
	Im Neuen Reich von Ägypten konnte nicht nur die Macht im eigenen Reich gefestigt werden, sondern versuchte diese Macht auch nach außen zu projezieren. Diese Projektion äußerte sich in Kämpfen zwischen den Hethitern, aber auch den Babylonieren und den Griechen. \\
	Was hinzugefügt werden sollte ist, dass, obwohl ein Pharaoh ein absoluter Herrscher war, dieser nicht unbedingt eine Stärke im Reich gezeigt hat. Diese wurden oft durch die Aristokratie alsbald entledigt. Ein solches Beispiel ist \textbf{Tutenchamun}, dessen Grab ungeöffnet war. \\
	Nach Ende des Neuen Reichs findet langsam ein Zusammenbruch des intranationalen Machtsystems statt, welcher durch die Ägäis ausgeht und eine langfristige Schwächung der Herrscher dieses Gebiets führt. Von dieser Zeit aus sind nicht nur in Ägypten sondern auch im Zweistromland Gebiete oft durch fremde Herrscher besetzt, anstatt Selbstbestimmung zu haben.

	\section{Wirtschaft im Alten Orient und frühe Ägäis - 17.03.2022}
	\subsection{Wirtschaft: Mesopotamien und Ägypten}

	Die Wirtschaft in Mesopotamien war zwar nicht so trocken wie in Ägypten, hatte jedoch eine ausgeprägte Wasserwirtschaft und führte oft zu Stadtstaaten die sich an einem der Flüsse ansiedelt. Mesopotamien war eine \textbf{Redistributionswirtschaft}, was bedeutet, dass der Boden zum Großteil im Besitz des Stadtstaates, also dem Herrscher und der Aristokratie, liegt und die gemeine Bevölkerung diesen verwenden kann, aber diese auch zu Kriegs- und Bauzwecken herangezogen werden konnte. Gleichzeitig wurde die Bevölkerung jedoch für diese Dienste wiederum mit Waren entlohnt, woher der Name rührt. \\
	
	Die Wirtschaft in Mesopotamien beruhte noch nicht auf einer Währung sondern rein auf Warentausch, weshalb auch oft der Wert von waren untereinander abgewogen wurden, also wie viel eine Länge Leinen im Vergleich zu Weizen wert war. Für Rohstoffe musste jedoch, wie zuletzt besprochen, mit anderen Völkern getauscht werden, was auch einen regen Handel in der Region ankurbelte. \\ \\
	
	In Ägypten gab es eine größtenteils ähnliche Situation wie in Mesopotamien. Ägypten ist zuallererst das Land am Nil, obwohl oft der gesamte Saharateil auch mitgezählt wird, jedoch gibt es einen scharfen Schnitt zwischen fruchtbarem Land und trockener Wüste. Die Bewässerung war ein Grundpfeiler der altägyptischen Gesellschaft. Dies geschah zuerst durch die natürliche Überflutungszeit des Nils, aber später auch durch eigens angebaute Kanäle, welche das Wasser weiter verbreiten. Die Nilschwemme rührt von den Monsunwinden, welche extrem viel Wasser in die Afrikaregion einführen, was wiederum um das Einzugsgebiet vom Nil abregnet und zu dieser jährlichen Überflutung führt. Während dies heute bekannt ist, wurde es im antiken Ägypten als "Geschenk des Nils" angesehen. \\
	Zurück zur Wirtschaft: Auch in Ägypten gab es eine ähnlich funktionierende Redistributionswirtschaft, wo Bauern fremdes Land bewirtschaften und einen Teil davon abgeben und hierfür auch anderes wie Baumaterialien erhalten. Während es ähnlich von Mesopotamien war, war es ungleich größer als das Zweistromland, was die Administration komplexer werden ließ. \\
	Rohstoffmäßig war Ägypten besser gestellt als Mesopotamien und war nicht abhängig vom Handel um essentielle Ressourcen zu erhalten. Trotzdem wurde reger Handel, vor allem mit dem nubischen Reich, getrieben, größtenteils um Waren zu erhalten, welche in Ägypten nicht erhätlich waren (Größtenteils Gold). \\
	
	\subsection{Frühe Ägäis}
	Während aus Sicht der Hethiter und Ägypter die Ägäis zuerst eher als Randregion angesehen wurde, gewann das heutige Griechenland mehr Einfluss. Kleinere Zentren innerhalb der Region dominieren mit der Zeit immer mehr und bringen auch bald selbst Kulturen hervor. Zu dem Zeitpunkt wurde Bronze als Metall angesehen, welches lange Zeit dominant war. Es folgte auf die Kupferzeit, was jedoch eher als Randnotiz angesehen wird, da diese nicht so lange währte. Während Kupfer ein Metall ist, ist Bronze eine Legierung aus Zinn und Kupfer. Dies verlangt einige Sachen um zu funktionieren, da man Vorkommen beider Metalle, sowie Möglichkeiten diese zu legieren und bearbeiten benötigt. Dies zeugt viel mehr von einer Hochkultur im Vergleich zur Verwendung von bloßem Stein oder Kupfer. \\
	Die Bronzezeit im Ägäisbereich ist in drei Teile geteilt:
	\begin{itemize}
		\item{Frühe Bronzezeit - 3100 - 2100 (FK, FH, FM)}
		\item{Mittlere Bronzezeit - 2100 - 1700 (MK, MH, MM)}
		\item{Späte Bronzezeit - 1700 - 1100 (SK, SH, SM)}
	\end{itemize}
	Gleichzeitig wird es in drei geographische Gliederungen:
	\begin{itemize}
		\item{Kykladen/Kykladikum}
		\item{Festland/Helladikum}
		\item{Kreta/Minoikum}
	\end{itemize}
	Diese werden wieder in Früh-, Mittel-, Spätzeiten aufgeteilt: Frühes Kykladikum, Mittleres Kykladikum und Spätes Kykladikum etc. \\
	Wenn man diese Zeitspannen mit den Zeiten von Mesopotamien und Ägypten vergleicht, sieht man, dass diese zwar etwas früher beginnen, jedoch im Endeffekt um 1100 konvergieren. \\
	Während Hieroglyphen in Ägypten zu einer umfassenden Datierung geführt haben, ist das in der Ägäis nicht so genau zu datieren, weshalb die meisten der Zeiten grobe Schätzungen sind. \\ \\
	
	Zusätzlich ist in der Ägäis die Art der Quellensuche eine sehr unterschiedliche wie in Mesopotamien. Unter anderem gibt es in der Ägäischen Frühzeit schriftliche Quellen, sogenannte \textbf{Linear A und B Tafeln}, wovon Erstere jedoch noch nicht entschlüsselt worden sind. Zusätzlich gibt es viele archäologische Quellen. Besonders in der Keramik konnte man anhand der Art der Herstellung oder der Zusammensetzung selbst, genaue Entwicklungen feststellen. \\
	Die erste erwähnenswürdige Stadt ist \textbf{Troja}, welches im Vergleich zu den drei Kulturen etwas außenvor steht. Jedoch ist es aufgrund der intensiven Ausgrabungen, welche in dem Gebiet durchgeführt wurden, ein Paradebeispiel der griechischen Archäologie. Vor 150 Jahren wurden bereits die Ausgrbaungsstätten gefunden, welche auf Troja Rückschluss gaben. Während es der Ort der Lyrik von Homer war, hat dieser jedoch auch eher im Retrospekt geschrieben, weshalb diese wahrscheinlich nicht geographisch akkurat sind. \\
	In Troja konnte man die Bildung der Siedlung in außerordentlichem Detail rekonstruieren. Diese werden als Phase I bis Phase IX bezeichnet. Um 3000 in Phase I gab es zwar nur einige Hütten, hatte jedoch bereits eine Verteidigungsmauer. In Phase II gibt es eindeutige panmäßige Erweiterungen, mit Terrassierungen und Umfassungsmauern. In Phase III ist Troja deutlich weniger dicht besiedelt, wahrscheinlich durch Zerstörungen, welche nur langsam wiederaufgebaut wurden. In Phase IV hat sich dort eine andere Bevölkerungsgruppe angesiedelt, mit Zuwanderern aus Anatolien und die Stadt langsam wieder aufgebaut wird. In Phase V ist ein deutlicher Aufschwung erkennbar, wahrscheinlich angeregt durch Handel mit Kreta und anderen Kulturen. Jedoch wird zum Ende auch ein Verlassen der Stadt erkenntlich, was jedoch nicht auf Zerstörungen zurückführen ist, sondern wahrscheinlich eher auf Krankheiten. In Phase VI wird Troja erneut durch eine neue Bevölkerungsgruppe besiedelt, diesmal [[Nubiern]] und markiert die Blütezeit der Stadt. Zu dieser Zeit nimmt Troja einen Knotenpunkt ein, da es für den Nord-Süd Verkehr essentiell war. Dies rührte daher, dass bei der Fahrt nach Süden oft Nordwinde herrschen und man diesen abwarten muss um weiterfahren zu können. In dieser Phase kommen auch immer wieder Brände vor, was jedoch nicht zu einem Verlassen führt. In Phase VII sind größere Brandkatastrophen nachweisbar und auch die Besiedelung einer weiteren Bevölkerungsgruppe wird nachgewiesen : Luwier. In Troja wurde zuerst recht grob vorgegangen, da man nocht nicht genau über diese Vorgänge Bescheid wusste, bzw. welchen Schaden dieser verursacht. Der \textit{Schliemann-Graben}, welcher in Troja gegraben wurde, ist in der Archäologie noch heute ein Begriff. \\ \\
	
	Um nun wieder auf die Hochkulturen der Ägäis zurückzukommen. Die entscheidenden Epochen der Bronzezeit sind das Mittelminoische und das Späthellenikum, welche auch in größerem Detail besprochen werden. Der Rest wird größtenteils zusammengefasst.\\
	\subsection{Ägäisgebiete:}
	\subsubsection{Frühes Kykladikum}
	Das Frühkykladikum, in den zentralen Inseln Griechenlands, auch als Kykloskreisinseln bezeichnet, gab es kleiner Küstennahe Siedlungen, welche später ins Hinterland übersiedelten und auch stärker befestigt wurden. Man kann hier auch einen Minoischen Einfluss nachweisen, größtenteils durch Handel. Die Wirtschaft war recht rudimentär und bestand größtenteils aus Fischerei und Landwirtschaft. Jedoch ist der Boden in dieser Region eher unfruchtbar, weshalb die Bevölkerungsanzahl begrenzt war. Ein wichtiges Gut der Region ist Obsidian, also Vulkanglas, besonders von der Insel Milos. Dieser wurde oft als Messer verwendet. \\
	\subsubsection{Frühes Minoikum}
	Kreta unterschied sich von den Kykladeninsel, da der Boden bedeutend fruchtbarer war weshalb auch größere Bevölkerungen ernährt werden konnten. Hier lassen sich auch Handelskontakte zu Ägypten und dem Mesopotamischen Raum nachweisen. In Kreta wurde auch eine staatliche und hierarchische Struktur angenommen, welche auf der Ägyptischen basiert, was die Handelsbeziehungen unterstützt. \\
	Später gab es auf Kreta größere Zerstörungen, welche wahrscheinlich auf die Luwier zurückzuführen sind. \\
	\subsubsection{Frühes Helladikum}
	Am Festland Griechenlands waren Siedlungen größtenteils an den Flüssen. Zu diesem Zeitpunkt gab es auch schon Handel mit den Kykladen und den Kreten. Die Grundlage der Wirtschaft ist, wie immer, Landwirtschaft, aber auch Gewerbe da das Festland Griechenlands sehr unterschiedlich ist. Die furchtbarsten Gebiete waren um Mykenes \\
	Die Luwier sind eine Bevölkerungsgruppe aus dem Westasiatischen Raum, welche in Teilen der heutigen Türkei lebten. Diese drängten oft in die Ägäis und beeinflussten diese auch. \\
	\subsubsection{Mittleres Kykladikum}
	In der Mittelbronzezeit werden die Siedlungen in den Kykladen oft wieder zerstört aber auch wieder aufgebaut. Hierdurch entstanden umfangreiche Zentren und Befestigungsanlagen um zu widerstehen. Es gab jedoch auch Handelsbeziehungen zum Festland sowie nach Kreta. Der Einfluss von Kreta ist zu diesem Zeitpunkt bedeutend stärker als noch in der Frühbonzezeit. \\
	\subsubsection{Mittleres Minoikum}
	In Kreta selbst ist ein bedeutendes Bevölkerungswachstum nachweisbar. Ebenfalls wird der Verwaltungsapparat straffer. Hier werden in Teilen Kretas, vor allem in Knossos, große Paläste gebaut, welche als Residenz und Handelszentrum dienten. \\
	In Kreta gab es wiederum wieder eine Redistributionswirtschaft wobei die Paläste als Zentrum agierten. Der Handel mit Ägypten und dem Orient bleibt aufrecht und nimmt sogar bedeutend zu. Gleichzeitig findet jedoch auch eine Kolonialisierung dieser Gebiete statt, wobei Handelsaußenposten gebaut werden. Kreta hatte zu diesem Zeitpunkt etwas, was oft als \textbf{Thalassokratie} bezeichnet wird. Kreta hatte eine umfangreiche Seeherrschaft und übte auch großen Einfluss auf seine Umgebung aus. Später sind größere Zerstörungen auf Kreta nachweisbar, was jedoch eher Naturkatastrophen zuzuschreiben ist. Die Paläste, welche zerstört wurden, werden noch größer und pompöser wiederaufgebaut. Dies wird auch als die \textbf{Neupalastzeit} bezeichnet. Aus dieser Zeit stammen auch die angesprochenen Linear-A Tafeln, welche trotz fehlender Entschlüsselung als Admministrations und Wirtschaftstafeln gesehen werden. \\
	Auf Kreta gab es umfangreiche Palastzentren. Die größten Zentren waren Knossos, Zakros und [[ADD MORE PALACES]]. Die Paläste waren umfangreiche Bauten, welche als Residenz, Handelszentrum und theologischer Ort verwendet wurden. \\
	Zu dieser Zeit war auch die Palastanlage in Zakros aufgrund der guten Lage von großem Einfluss. \\
	\subsubsection{Mittleres Helladikum}
	Während Kreta zu dieser Zeit aufblühte, war das Festland eher durch kleinere Dörfer besiedelt, welche jedoch zahlreicher waren. Hier ist im Vergleich ein Rückgang des Wohlstands zu verzeichnen. \\
	\subsubsection{Spätes Minoikum}
	In der späten Bronzezeit ist dieses Verhältnis jedoch umgekehrt und das Festland dominiert über Kreta. In Kreta dominieren die Paläste zwar lokal, verlieren jedoch im großen an Einfluss, was Knossos zur einzigen größeren Stadt macht. Später sind auf Kreta umfangreiche Plünderungen und Zerstörungen der Paläste nachzuweisen. Das Festland nimmt die Insel Kreta hier komplett ein und ist unter dessen Kontrolle. \\
	\subsubsection{Spätes Helladikum}
	Am Festland gab es unterdessen eine Blütezeit, wobei Mykenes als dominante Stadt etabliert wird, was auch zur Bezeichnung dieser Ära zeugt: Die Mykenische Zeit. \\
	Siedlungen welche von der Bevölkerung der Mykenischen Welt gebaut wurden sind oft viel befestigter und größer ausgebaut, was diese zu einflussreichen Zentren des Handels macht. Aus dieser Zeit stammen die Linear-B Tafeln. Von Mykenes ausgehend übernimmt diese Kultur größtenteils die Ägäis und führt zu einer recht kulturell homogenen Gesellschaft. Diese Einigung führt später dazu, dass Griechenland geeint mit den Hethitern agieren kann. Dies ist aus Schriften der Mykener und Hethiter ersichtlich. Mykenes und das Hethiterreich sind schließlich auch in mehrere Kämpfe verwickelt, während dieses mit den Ägypten kämpft. \\
	Der Niedergang der Mykenischen Welt, welcher auch zu einer Schwächung des Hethiterreichs und Ägypten führt, passiert durch einige Naturkatastrophen in der Umgebung, aber auch durch Auseinandersetzungen zwischen einiger der Paläste in der Ägäis, welche zu Einbußen im landwirtschaftlichen Ertrag führte. In den Linear-B Texten wird dies beschrieben, während jedoch auch von Druck aus dem Norden geschrieben wird. Mit dem Wegfall des Mykenischen Reich fallen für Ägypten wichtige Handelskontakte weg. Das Hethiterreich verschwindet zu dieser Zeit nahezu komplett. Es wird vermutet, dass dies durch Druck aus Nordwesten geschehen war. Die Ägypter sprachen von den "Seevölkern". Welche Völker dies waren ist unersichtlich. Es könnte sein, dass es die Mykener selbst waren, welche aufhörten Handel zu führen. Nach dem Untergang des Mykenischen Reichs wird die Zeit oft als "Dark Ages" der Umgebung bezeichent, da sehr wenig darüber bekannt ist, da es fast keine Hinweise gibt.
	
	\section{Geometrische Zeit und Archaik - 24.03.2022}

	Die Geometrische Zeit ist ein recht archäologischer Begriff. Die Geometrische Zeit (ca. 900-750 v. Chr.) ist nach archäologischen Funden in Athen benannt. Diese Funde hatten sich danach auch auf weitere Zentren in Griechenland ausgeweitet. So gabes weitere Funde aus Korinth, Argos, Boiotien, Euboia und den Kykladen. Die Funde sind Vasen und Skypha mit sehr kantigen Mustern, welche diese geometrischen Formen bilden. Am Anfang der geometrischen Zeit waren die Bemalungen recht spärlich. In späteren Vasen bemerkt man, dass diese schon bedeutend enger gemalt wurden. In der spätgeometrischen Zeit (750 - 700 v. Chr.) ist schließlich die nahezu ganze Vase mit Formen bemalt. Eine der bekanntesten Vasen dieser Zeit ist die Dipylon-Vase. Der Name rührt auch daher, dass ein Großteil des Verständnisses dieser Zeit aus der Archäologie stammt. 
	\begin{figure}
	\includegraphics[scale=0.2]{Dipylon.jpg}
	\end{figure}
	Beim Übergang der geometrischen Zeit in die Archaik (600 v. Chr.) ist wiederum eine Änderung der Bemalung, wo eine Entwicklung weg von geometrischen Figuren und mehr zu Personen und Figuren bemerkbar ist.
	\subsection{Das Perserreich}
	Um die Antike zu verstehen, muss man zuerst über das Perserreich reden, da diese ewigen Widersacher der Griechen sind. Persien nahm einen recht bescheidenen Anfang zu 557 a.c. etwas östlich von Mesopotamien nimmt, weitet es sich über die nächsten 18 Jahre bis nach Griechenland aus. Diese Ausweitung geschah unter \textbf{Kyros II. dem Großen}. Noch weiter ausgedehnt wurde es durch Dareios um 500. a.c. Zu diesem Zeitpunkt drang das Perserreich bereits in das Einflussgebiet von Griechenland vor, wodurch es immer wieder ungewollte Kontakte mit ihnen gab. \\
	Das Perserreich wurde zu dem was es war durch die Herrschaft von Kyros II. Ein Meilenstein dessen war die Einnahme von Babylon. Diese Schlacht ist gut nachvollziehbar, da Kyros in seinem sogenannten \textbf{Kyros-Zylinder}, eine Propagandaschrift verbreitete. Die Schrift ist höchst propagandistisch und selbstverherrlichend, in welchem er sich als großen Herrscher Babylons kührte, welcher das Volk vom Joch befreite. Großer Punkt der Rhetorik ist die göttliche Sphäre, zu welcher Kyros sich berief. Der Gott Marduk erkannte, dass die Herrscher Babylons ihre Stadt, Bevölkerung und Kultstätten vernachlässigte, wodurch er Kyros dazu berief sie zu befreien. Zusätzlich geschah die Einnahme der Stadt anscheinend ohne Schwertstreich, was natürlich sehr unwahrscheinlich ist. \\
	Diese Praxis der Propagandaschrift wird durch seinen Nachfolger Dareios fortgeführt, in welcher er sich in ähnlichem Stil beweihräuchert. In der \textbf{Behistun-Feldschrift} kann man ähnliches lesen wie zuvor von Kyros. In dem Fels ist auch eine Depiktion, welche Dareios zeigt wie er die Stämme unterwirft, überwacht von Ahura Mazda. Wieder ernennt er sich den großen König und beansprucht die Unterstützung Ahura Mazdas für sich, weshalb die Länder ihm Tribut brachten. Er flehte Ahura Mazda um Hilfe um die Herrschaft, welches seinem Geschlecht genommen worden war, wiederzuerlangen wodurch er die Harmonie wiederherstellt. Diese Schrift ist jedoch nicht für menschliche Augen gedacht, da es in einer recht großen Höhe eingraviert ist, sondern soll eher von dne Göttern gelesen werden. Dies macht im Persischen Kontext Sinn, da der Herrscher von Persien seine Herrschaft nur dadurch legitimisiert, dass er den Rückhalt der Götter hat, wodurch er als Mittler zwischen der göttlichen und erdlichen Sphäre fungiert. Doch obwohl die Schrift für göttliche Augen bestimmt ist, ist die Schrift selbst in vielen Sprachen abgebildet, was die verschiedenen Dialekte und Sprachen des persischen Vielvölkerstaats widerspiegelt. Das ist wahrscheinlich geschehen, da die Götter sehen sollen, dass sich Dareios für sein Volk einsetzt. Auch ist es an einer zu dem Zeitpunkt oft verwendeten Handelsstraße gebaut, wodurch es wahrscheinlich von vielen reisenden Händlern gesehen wurde. \\
	Ein Problem über das Verständnis der Perser ist, dass es keine inneren Schriften gibt. Jegliche Schriften über das Perserreich sind entweder Propagandaschriften aus dem Perserreich, oder Erzählungen aus Griechenland, welche mit dem Perserreich verfeindet waren. In Griechenland war der Schriftsteller, welcher sich am meisten damit befasst hat, Herodot, welcher auch ein Buch über das Perserreich schrieb. Er selbst war jedoch nie dort und nahm sein Wissen aus Erzählungen von Reisenden welche dort waren. \\
	Eine Schrift von Herodot sind die Steuerpraktiken. Er schreibt über die Steuern, welche das Perserreich von seinen Tributaren einbrachte, was er mit 14560 Talenten an Silber einschätzte. (1 Talent = 26kg) Jedoch muss erwähnt werden, dass man  die Menge in Kilogramm nicht 1:1 sehen kann, da zu dem Zeitpunkt Silber viel seltener und teurer zu verarbeiten waren. \\
	Mit Xerxes began das Perserreich große Prachtbauten, meist in Persepolis, zu errichten, welche zuvor so nicht existiert hatten. Bei der Konstruktion sind Elemente aus Ägypten, aber auch aus Griechenland zu sehen, zu welchen die Perser Handelsbeziehungen hatten. Diese Tempel müssen Unsummen gekostet haben, da Holz aus Kreta und andere weit entfernte Rohstoffe verwendet wurden. \\
	Weiters wurden Reliefe dort gefunden, welche zwar eindeutig einen persischen Stil hatten, aber mit griechischen Handwerkspraktiken errichtet wurden. So wurden Griechen aus dem Ionischen Reich angeheuert, da die Perser solche Technologie nicht hatten. Die Reliefe aus dem Perserreich unterscheiden sich trotzdem drastisch von den griechischen Arbeiten und sind um einiges kälter und distanzierter. \\
	Im Gegensatz sind die Arbeiten, welche die Griechen über die Perser erschafften, viel dynamischer und in Bewegung, was etwas ist, was im persischen Reich nie gesehen wurde. Gleichzeitig sind die Perser jedoch als "exotisch" und "fremd" dargestellt, da diese oft eigenartig anumutende Ausrüstung zu tragen scheinen. \\
	\subsection{Der Ionische Aufstand}
	Die Quellen aus der Zeit der Persischen Kriege sind etwas umstritten, da Herodot der einzige war, welcher umfassend darüber schrieb. Das Perserreich hatte einige Wege um Kolonien zu verwalten, unter anderen den Satrap, welcher ein Perser ist und aus der Familie des Herrschers eingesetzt wird. Laut ihm wurde der Krieg angezettelt, da ein Statthalter des Perserreiches in Milet und der Satrap von Sardeis, Naxos übernehmen wollten, da dieses zu dem Zeitpunkt großen Reichtum besaß, da dort Marmor von hoher Qualität in aller Welt verschifft wurde. Diese Bestrebung ging furchtbar schief, wodurch es große wirtschaftliche Einbußen gab und ein paar tausend Soldaten verloren hatte. Um von diesem Problem abzulenken, trieben sie die Ionischen Griechen zur Revolte. Dies funktionierte, weil es in dem Gebiet zum relativen Erliegen des Handels mit den Griechen gekommen war, diese aber trotzdem den horrenden Tribut entrichten mussten. Normalerweise würde sich eine Tributarstadt nicht gegen das viel stärkere Persische Militär richten. Dieser Aufstand begann 500 v. Chr. und wurde um 494 v. Chr brutal beendet indem Milet völlig zerstört wurde. Der Grund, weshalb es 6 Jahre dauerte ist wahrscheinlich, da der persische Herrscher andersweitig beschäftigt war. \\
	\subsection{Großmächte der Archaik in Griechenland}

	Im Antiken Griechenland gab es zwei große Mächte: Sparta und Athen. \\
	Sparta hatte ähnlich wie der Perserherrscher eine Mythologie, welche bereits zur Zeit der Stadt entstand. Dieser Mythos rührte daher, dass Spartaner eine extreme staatliche Kontrolle hatte. Nahezu alles was in der Stadt geschah, wurde durch den Staat übersehen. Kinder, Mädchen und Jungen, wurden von klein auf darauf trainiert von starkem Körper und Willen zu sein. Männer mussten rigides Training erfahren um gute Soldaten zu sein. Frauen mussten dieses Training erfahren, da Sparta der Ansicht war, dass nur ein starker Körper starke Kinder gebähren kann. \\
	In Sparta gab es ein starres Kastensystem, welches in drei Teile geteilt wurde:
	Die Spartiaten waren Vollbürger und es war ihre Pflicht körperlich fit zu sein, da sie für Sparta kämpfen mussten. Die Periöken hingegen waren zugezogene, welche Tätigkeiten ausführten, welche ein Spartiate niemals machen würde, wie Handwerk oder Handel. Heloten waren hingegen die unterste Kaste und können auch als Staatssklaven bezeichnet werden. Heloten waren essentiell für reibungsloses Dasein innerhalb Spartas und Sparta war extrem abhängig von ihnne. Libanios schrieb im 4. Jh.n.Chr., dass Spartiaten Heloten in ständigem Misstrauen überwachen und entwaffnen, da es ständig Bedenken eines Aufstandes gibt. \\
	Jedoch gab es auch viele übertriebene Geschichten über Sparta, da Sparta daran interessiert war, dass umgebende Völker Ehrfurcht und Angst vor den Spartanern fühlen. Zum Beispiel gibt es selbst heute noch den Mythos, dass Sparta Neugeborene umbrachte, falls diese nicht perfekt waren. Jedoch zeigen Ausgrabungen in Sparta, dass es Skelette von Spartiaten gab, welche große kongenitale Fehlbildungen besaßen und wahrscheinlich nicht ohne Hilfe gehen konnten. Da diese ein beachtliches Alter erreichten, muss man annehmen, dass Sparta sich um sie gekümmert hat. Ein weiteres ist, dass Sparta keine Walle brauchte, da sie eh nicht angegriffen werden würden. \\
	Bekannte Literatur aus Sparta sind sogenannte \textbf{Kampfparänesen}, welche dem Zweck dienten vor einem Kampf zu motivieren, aber auch verbreitet wurden um den Ruf Spartas weiter zu verbreiten. \\
	Spartiaten hatten ein ausgeprägtes Verständnis von \textit{homoioi}, dass nur Spartiaten einem selbst ebenbürtig waren. \\
	
	\section{Geometrik und Archaik - Teil 2 - 31.03.2022}

	Während wir das letzte Mal Sparta behandelten, wird dieses Mal der Stadtstaat Athen besprochen. In Athen gab es am Anfang der Stadt nahezu einen Tyrannen, also einen dispotischen Einzelherrscher, weshalb 594 v. Chr. Solon zum Archon ernannt wurde. Es wird vermutet, dass es in Athen größte Unruhen gab, dessen Ursachen sind unbekannt, jedoch kann angenommen werden, dass es zu einer länger anhaltenden Dürre gekommen ist, wodurch Bauern ihren Lebensunterhalt nicht mehr bestreiten konnten, wodurch sie gegenüber der Großgrundbesitzer in Schuldlast fielen. Dies führte under anderem zu Schuldknechtschaft. Der Staat Athen unter Solon beschloss deshalb einen Schuldenerlass, da eine große Anzahl der Bevölkerung in Schuldknechtschaft schlecht für den Staat ist. \\
	Das bedeutet, dass Großgrundbesitzer entscheiden mussten, welche Schulden noch offen sind, was natürlich für die Schuldenhalter schlecht war, jedoch als Notwendigkeit angesehen wurde. Ebenfalls kaufte der Staat auf eigene Kosten Leute, welche im Ausland in Schuldknechtschaft gingen, wieder zurück. Weiters wurden 4 verschiedene Kasten erschaffen, welche die soziale Stellung zeigen und entscheiden, welche Steuern zu entrichten sind. Die Einschätzung der Steuerlast wurde in 4 Phyle aufgeteilt, wobei jede Gruppe 100 
	Ratsmitglieder enthielt. Die meisten dieser Entscheidungen werden heutzutage als sehr demokratisch angesehen, war jedoch zu dem Zeitpunkt nicht die Intention, sondern diente alleine um einen Tyrannen zu verhindern. \\
	Wirtschaftliche Maßnahmen waren unter anderem auch Exportverbote von Gütern, weshalb nur Olivenöl nach außen verkauft werden durfte. Dies geschah, da der Staat selbst viele Güter nur in begrenzten Mengen besitzt. \\
	Tyrannos war zu dem Zeitpunkt kein negativ behaftetes Wort. Ursprünglich stammt es aus dem viluvischen und bedeutet "Der Gerechte" oder "Der Richter". Als es um 700 v.Chr. in das Griechische kam. Im Griechischen stand es für eine Person, welche in Umgehung der geltenden Gesetze an die Macht gelangte. Das war jedoch alles und zeigte sich dadurch wie es später behandelt wurde. \\
	Diese Maßnahmen waren langfristig höchst effektiv und führten zu einer lange anhaltenden Machtposition von Athen. Kurz und mittelfristig war es sehr unpopulär. \\
	Dies sieht man unter anderem darunter, dass kurz (40 Jahre) später ein Tyrann an die Macht kam. \textbf{Peisistratos} wird von Herodot als Tyrann bezeichnet, stellte sich jedoch später als sehr guter Tyrann heraus (Was wie ein Widerspruch wirkt). \\
	Peisistratos förderte Kunst und Kultur und legte unter anderem die Homerischen Werke neu aus. Er schuf internationale Verbindungen indem er seine Kinder strategisch verheiratete. \\
	Athen selbst wird auch grundlegend umgebaut. Die Akropolis, die Agora sowie das Olympieion werden erweitert. Zusätzlich werden Wasserleitungen, Abwasserkanaäle sowie Brunnenhäuser gebaut. Auch erweiterte er den Hafen Peiraios, welcher zuvor ein kleiner Hafen war, und erweiterte es mit einer Festung. \\
	Unter Peisistratos wurde der Zeustempel, das Olympieion, begonnen (um 550v.Chr.) jedoch erst unter Hadrian (132 v.Chr.) fertiggestellt und war eines der imposantesten Gebäude seiner Zeit. \\
	Seine Söhne \textbf{Hipparchos und Hippias} hatten jedoch, als ihr Vater im hohen Alter starb, nicht die selbe Stärke, was in einem Attentat, welches während des Panathenäenfestes, ersichtlich wurde. Hipparchos wurde, laut Thukydides, von zwei eifersüchtigen und gedemütigten Personen getötet. Hippias wurde danach sehr paranoid und 4 Jahre später von den Spartanern und den Alkmainoiden abgesetzt. Die Spartaner taten dies natürlich nicht uneigennützig, da sie somit ihre Macht festigen konnten. Die Alkmainoiden waren ein verbannter Stamm in der Region Athens. \\
	Da nun die Tyrannen wieder verschwunden waren, entstand ein Machtvakuum, was wieder zu einem potentiellen Tyrannen führte. \\
	Um dies zu verhindern gab es wieder Reformen, dieses Mal unter \textbf{Kleisthenes}.\\
	Unter ihm wurden die 4 Phylen in 10 Phylen erweitert. Diese Phylen wurden jedoch in drei weitere Teile aufgeteilt, die sogenannte Trittyen-Teilungen. In einer Phyle war ein Teil stets an der Küste, in der Stadt und im Hinterland. Dies bedeutet, dass die Phylen nicht mehr topografisch sondern "virtuell" aufgeteilt wurden. Dadurch, dass die Teile unterschiedliche Teile des Landes ausmachten waren innerhalb eines Wahlkörpers Fischer, Hirten und Adelige. Weiters wurden große Teile des Landes, welches meistens von einem oder zwei Adelshäusern gehalten wurde, aufgespalten, da diese zuvor, als ein Phylus ein topografisches Gebiet war, stets ihre abhängigen Personen beeinflussten, welche keine andere Wahl als für sie zu stimmen hatte. Danach stimmten jedoch Leute, welche keine persönliche Abhängigkeit besaßen, für Richtlinien die sie für gut erachteten. Dieses 10 Phylen System beeinflusst die Ordnung von Athen für sehr lange Zeit. Dieser Vorgang kann wieder als sehr demokratisch angesehen werden, jedoch war das wieder zu dem Zeitpunkt nicht die Absicht, sondern nur um einen Tyrannen zu verhindern. \\
	Die Phylenaufteilung hatte auch im militärischen Sinne Auswirkungen wodurch es 10 Heeresabteilungen gab, welche als Einheit agieren und somit mehr Zusammenhalt erhalten sollten. Auch hatte jede der 10 Abteilungen einen eigenen Strategen, wodurch die Strategie der Armee von diesen Personen geleitet wurde. \\
	Jedoch war ein Nachteil, dass die Verwaltung der einzelnen Teile auch recht aufwändig waren. Jedoch, und das wurde nicht als Beweis angesehen, hatte Kleisthenes nicht uneigennützig die Besitze seines Hauses in einzelne Phylen gebündelt. \\
	\subsection{Die Klassik}

	Mit der Klassik beginnt die Formierung des \textbf{Attisch-Delischen Seebundes}. Danach gab es zunehmend Konflikte zwischen Athen und Sparta, bis zu Beginn des Peloponeischen Krieges. Diese Zeit wird als \textbf{Pentekontaëtie (50 Jahre)} bezeichnet.
	Innnenpolitisch gab es wieder Innenpolitische Reformen:
	\begin{itemize}
		\item{Die Telesinos-Reform 487}
		\begin{itemize}
			\item{Hier wurden Archonten, welche das höchste Amt von Athen war und nur aus Adeligen bestand, fortan nicht mehr gewählt sondern verlost.}
		\end{itemize}
		\item{Ephialtes-Reform 461}
		\begin{itemize}
			\item{Befugnisse von Archonten wurde auf die Volksversammlung und Volksgerichtshöfe verlagert, womit das Amt weiter geschwächt wurde}
		\end{itemize}
		\item{Perikleische Gesetze ab 450}
		\begin{itemize}
			\item{Führte das Epigamie-Gesetz ein, wodurch nur die Kinder von zwei Athenern Athener waren. Jedoch wird angenommen, dass es Ausnahmefälle gab, da seine Frau nicht Athenerin war, sein Sohn jedoch trotzdem ein Amt als Stratege innehatte. Auch führte er die \textbf{Diäten} ein, welche ein Tagessatz war, welcher als Entschädigung für offizielle Tätigkeiten bezahlt wurde. Diese Diäten wurden irgendwann auch für den Besuch von Theateraufführungen bezahlt, was eventuell auf ein verringertes Interesse schließen lassen kann.}
		\end{itemize}
	\end{itemize}
	Ein weiterer Mechanismus war der \textbf{Ostrachismus}, bei welchem 6000 Personen auf Tonscherben Namen schrieben und diesen danach abgaben. Die Person die am öftesten genannt wurde, wurde für 10 Jahre verbannt und diente wieder de Verhinderung eines Tyrannen. Es wird oft angenommen, dass Kleisthenes diesen einführte, jedoch gibt es Anzeichen dafür, dass der allererste Ostrachon erst viel später stattfand, was bedeuten würde, dass es von jemand anderem eingeführt wurde. Es bedeutete auch nicht, dass man sein Vermögen verlor sondern nur den Ausschluss aus dem politischen Geschehen. \\
	Im Krieg von Athen gegen die Perser wurde angenommen, dass die Schlacht bei Marathon das Ende der Kriegshandlung bedeutete. \textbf{Themistokles} hingegen sah nur den Anfang und erreichte, dass das Silber aus den staatlichen Bergwerken nicht meh verteilt, sondern zur Errichtung von Schiffen verwendet wurden. So wurden 100 Trieren gebaut, welche im großen Stil im Kampf gegen Xerxes verwendet wurde. \\
	Er baute auch den Hafen von Piräus aus und errichtete die \textbf{Lange Mauer} zwischen dem Hafen und der Innenstadt. \\
	Nach der Gründung des \textbf{Attisch-Delischen Seebundes} war Athen eine wahre Seemacht und auch in der Lage das Gebiet zu verteidigen. Die Bündner genossen diesen Schutz im Gegenzug für Tribut, welcher nach Delos entrichtet werden musste, wo auch Bündnissitzungen gehalten wurden. Die größte Person dieser Zeit war Kimon, welcher den Bund zu seiner größten Zeit führte. Als Athen die Thasier besiegte und belagerte, baten diese die Spartaner um Hilfe, welche jedoch größere Probleme hatten. Dort gab es zum gleichen Zeitpunkt einen Aufstand der Heloten (Sklaven), was dazu führte, dass Sparta 2/3 seiner Heloten ziehen lassen musste.\\
	Insgesamt verhielt sich Athen sehr vorteilhaft gegenüber seinen Bündnern. So musste eine Stadt, welche nicht mehr Teil des Bundes sein wollte, die Mauern abreißen, alle Schiffe übergeben, fortweilend Tribut zahlen und alle Besitztümer am Festland abzugeben. Auch mussten Bündner alle 4 Jahre (Aber optimalerweise jährlich) nach Athen reisen, dort nächtigen und Teil des Siegeszuges sein, alles auf eigene Kosten. Es ist verständlich, dass die Bündner, welche die überfließenden Reichtümer Athens sahen, nicht glücklich darüber waren, was mit ihrem Tribut geschieht. Gleichzeitig bestrafte Athen Bündner, welche das Bündnis verraten auf das schlimmste. Ein Stadtstaat, welcher rebellierte wurde dem Erdboden gleichgemacht, alle Männer getötet und Frauen und Kinder in die Sklaverei verkauft. \\
	\section{Athen - Sparta Krieg - 07.04.2022}
	Die \textbf{Chalkidier} auf Chalkis, welche ebenfalls rebellierten wurden gezwungen einen Knebelvertrag einzugehen. \\
	\begin{wrapfigure}{l}{0.55\textwidth}
	\includegraphics[scale=0.5]{Images/Athen_Antike.jpeg}
	\caption{Ausbau unter Perikles}
	\end{wrapfigure}
	Athen baute zu dieser Zeit massiv aus. Man kann zu Recht sagen, dass kein Stein auf dem anderen geblieben war. Die Akkropolis wurde nach den Perserkriegen erweitert. Jegliche Bauten bestanden aus feinstem Marmor mit goldenen Intarsien. Die Parthenonstatue, welche Teil des Staatsschatzes war, bestand aus großen Mengen an Gold und Elfenbein. All dies wurde finanziert durch die Einnahmen aus dem Attisch-Delischen Seebund und gibt eine Perspektive über den Ausbruch des Peloponnesischen Krieges. \\
	Diese Maßnahmen in Athen wurden größtenteils von \textbf{Perikles} geleitet. Durch diesen großen Bauwahn wird es auch als das Perikleische Bauvorhaben genannt. Genaue Abrechnungen zeigen, dass tausende Leute aus dem gesamten Hellenischen Raum angeheuert wurden um Athen aufzubauen. Jeder Athener genoss zu diesem Zeitpunkt großen Reichtum, weshalb Perikles auch immer wieder das Vertrauen ausgesprochen wurde. Dies änderte sich natürlich mit Beginn des Krieges. \\
	Währenddessen erfuhr Sparta eine völlig andere Entwicklung. Nachdem der Perserkrieg mit 479. v.Chr. endete begann etwa 20 Jahre später der \textbf{Helotenaufstand}, welcher oft als Turning Point Spartas gesehen wird. Durch diesen Aufstand musste Sparta einen Großteil seiner Sklaven ziehen lassen, was Sparta ungemein schwächte, da Spartiaten keine niederen Arbeiten erledigen wollten. \\
	Durch diesen Aufstand hatte Sparta auch Befürchtungen, dass Athen diese Schwäche ausnützen könnte, wodurch sich Sparta zunehmends abkapselte. \\
	\subsection{Peloponnesischer Krieg}
	Mit 431. v.Chr. begann der Peloponnesische Krieg. Dieser kann in drei Phasen geteilt werden:
	\begin{enumerate}
		\item{Archidanischer Krieg (431 - 421)}
		\begin{itemize}
			\item{Einfall der Spartaner in Attika, was durch Themistokles' Ausbau des Hafens abgewehrt werden kann. Dadurch wurden sehr viele Spartaner gefangen genommen, was für die kleine Spartiatische Truppe ein großes Problem war.}
		\end{itemize}
		\item{Nikias-Friede (421 - 413) "Der Faule Friede"}
		\begin{itemize}
			\item{Athen benötigte Zeit um ihre Intrigen fortzuführen und Sparta war gezwungen sich zurückzuziehen. Dadurch enstand ein Friede, welcher von beiden Seiten bald zu brechen versucht wurde.}
			\item{Zum gleichen Zeitpunkt begeht Athen eine Expedition nach Sizilien, was sich als monumentaler Fehler herausstellte, da Athen keine Ahnung über die Begebenheiten des Gebiets hatte. Hier verlor Athen Unmengen an Geld, Personal und Strategen.}
		\end{itemize}
		\item{Dekeleischer Krieg (413 - 404)}
		\begin{itemize}
			\item{Sparta greift Attika erneut an, dieses Mal jedoch mit Unterstützung von Alkibiades, welcher zuvor in Athen wegen Frevel angeklagt wurde, wonach er zu Sparta überlief.}
			\item{Gleichzeitig mischt sich Persien erneut ein und zahlt Sparta eine kleine Summe dafür, dass sie Kleinasien aufgeben. Dies geschah jedoch erst nachdem sich Athen auch mit Persien zerwarf. }
			\item{Zwischendurch gab es in Athen auch einen Umsturz der Macht wonach ein oligarchischer Rat der 400 die Macht ergriff. Dieser währte jedoch nur kurz, da die Seeflotte sich diesem widersetzte.}
		\end{itemize}
	\end{enumerate}

	Nachdem der Krieg endete sah sich Sparta als Sieger, jedoch haben im Endeffekt beide Parteien zu große Verluste erlitten. Beide Seiten verloren durch diesen Krieg zunehmends an Einfluss und Macht. Sparta beginnt dadurch sich noch mehr abzukapseln. \\
	Nach dem Spruch "Wenn sich zwei streiten, freut sich der dritte", mischen sich durch diese Schwäche zwei weitere Kräfte in Griechenland ein. \\
	Persien weitete seinen Einfluss in Griechenland aus und nimmt einige Teile des Landes ein. \\
	\textbf{Theben} konnte durch dieses Machtvakuum zur Großmacht aufsteigen, was ohne den Krieg nicht geschehen hätte können. \\
	Im 400. Jh.v.Chr. hat Sparta wieder Bestrebungen seinen Einfluss nach Kleinasien auszuweiten. Persien, welches stets Bestrebt war keine der beiden Parteien zu stark werden zu lassen unterstützt dadurch Athen mit Gold, damit sich diese wehren können. Dadurch geht Athen eine Allianz mit Korinth, Boiotien und Argos ein um sich wehren zu können. Während Athen zuerst recht erfolgreich ist, ist Sparta letzendlich erfolgreich und Athen verliert die Vorherrschaft im Ägäisraum. \\
	Im Korinthischen Krieg [[EXPAND]] \\
	Mit dem zweiten Attischen Seebund wollte Athen seine Macht wieder expandieren, genau 100 Jahre nach Bildung des ersten Seebundes. Eine Änderung war, dass es einen Bundesrat (\textit{synhedrion}) gab, in welchem alle Bündner ein Stimmrecht hatten. Athen war zwar immer noch in der Vormachtsstellung innerhalb des Bündnisses, stand jedoch allen Bündern autonomien zu. Dies musste geschehen, da Athen wegen Sparta nicht mehr so viel Macht hatte. Dies geschah durch den \textbf{Königsfrieden}, wodurch Kleinasien nicht tangiert wird. So treten mit der Zeit 70 Stadtstaaten dem Bündnis zu. Dies geschah nicht weil Athen einen guten Deal bot, sondern weil Sparta einen noch schlechteren bot. Die Politik der Spartaner war nicht sehr beliebt mit den Griechen und viele Staaten traten dem Seebund bei, nicht weil sie Athen mochten, sondern weil sie Sparta noch weniger mochten. \\
	Obwohl es diese neuen Maßnahmen gab, wiederholten sich die gleichen Fehler des ersten Seebundes erneut und nahezu alle Fehler werden wiederholt. \\
	\subsection{Thebanische Hegemonie}
	Theben hat nach dem Machtvakuum Hegemoniebestrebungen. Um 371 erfuhr Theben einen großen Sieg über Sparta und konnte so ungestört fortfahren. Die zwei Personen verwantwortlich hierfür sind \textbf{Pelopidas} und \textbf{Epameinondas}. Zweiterer speziell, da er für die militärischen Erfolge verantwortlich war. \\
	Durch Thebens plötzliche Macht sehen einige Bündner eine Chance sich von Athen loszusagen. So wenden sich \textbf{Chios, Rhodos und Byzantion} Theben zu. Dies war ein Desaster für Athen, da sie die Schwarzmeerroute und den Zugang zu Holz verloren. Durch diesen plötzlichen dritten Spieler, verbünden sich Sparta und Athen um Theben zu unterwerfen. Theben, zwar mit Unterstützung Persiens, konnte dem nicht widerstehen und wird unterworfen. \\
	Trotz diesen Erfolges finden Athen und Sparta nie wieder zur alten Macht. Beide verloren Teile ihrer Armee und Athen behandelte die übrigen Bündner ebenso schlimm wie zuvor. Dies mündete schließlich im \textbf{Bundesgenossenkrieg}, welcher Athen das Kreuz brach. \\
	Auch der "Sargnagel Athens" genannt, greifen die ehemaligen Bündner Rhodos, Ko und Chios mit Byzantion Athen an. Mausolos und Persien unterstützen die Bündner ebenfalls. Auch Makedonien im Norden hat Bestrebungen für Griechenland. Als Konsequenz dieses Krieges wird Athens Macht massiv geschwächt und Athen zu Friedensverhandlungen gezwungen, was das Ende des unabhängigen Griechenlands bezeugt. \\
	\subsection{Makedonisches Reich}
	Vor Phillip war Makdeonien nur ein Nebenspieler für die Griechen. Es gab zwar Handelsabkommen, wurde jedoch nicht als echter Spieler anerkannt. Zu diesem Zeitpunkt sahen die Griechen Makedonien nicht als ihresgleichen an. Sie sprachen zwar einen griechischen Dialekt, waren jedoch keine "echten" Griechen. Das Land war ebenfalls rein landwirtschaftlich geprägt. Es gab ebenfalls große Holzvorkommen, wodurch Griechenland erpicht darauf war mit Makedonien zu handeln. In dieser Hinsicht hatten die Griechen Recht, darin, dass keine echte städtische Kultur gab und das Land ingesamt etws rückständig war. Ebenfalls gab es immer wieder interne Konflikte, wodurch der König keine absolute Kontrolle aufbauen konnte. Auch wurde das Reich immer wieder von den Ilyrern bedroht. Auch militärisch gesehen war Makedonien keine außergewöhnlcih großer Spieler. Das erkennt man schon daran, dass Makedonien sich nicht wirklich gegen die Illyrer wehren kann. \\
	Jeder kennt zwar Alexander den Großen, jedoch wäre nichts von dem passiert, wenn Phillip nicht den Grundstein gelegt hätte.
	\textbf{Phillip II.} übernimmt als Vormund des jungen Königs Amynthas zuerst die Macht. Dieser übernimmt zwar die Macht, greift den König selbst jedoch nie an, größtenteils da er es nicht nötig hatte. Phillip löst nahezu alle Probleme, welches Makedonien hatte. So reformiert er das Heer und führt die Sarissa ein, welche längere Lanzen besitzen. So können Infanterie und Kavallerie besser kombiniert werden. Auch wird die Belagerungstechnik ausgebaut. Vor Makedonien gab es nicht wirklich effektive Belagerungstechniken und er baut den Grundstein hierfür. Jegliche Errungenschaften Alexanders bauen auf den Grundfesten der Reform seines Vaters auf. \\
	Ebenfalls bindet Phillip das Heer an sich selbst, indem er größzügig mit seinen Soldaten umgeht und Offiziere mit Land beschenkt. \\
	Dass Phillip als "Rohling" dargestellt wird, ist wahrscheinlich eine glatte Lüge, das erkennt man an seinen Reformen. Er war zwar völlig anders als sein Sohn, mit dem er sich überhaupt nicht verstand, gab diesem jedoch was er für seinen Feldzug benötigte. \\
	Nach diesen Reformen offenbarte sich Phillip eine Gelegenheit um sich zu beweisen: Dem 3. Heiligen Krieg. \\
	Heilig heißt in diesem Kontext das Heiligtum von Delphi, welches von Phokischen Söldnern besetzt wird um sich zu bereichern. Dieses Heiligtum wird von diesen Söldnern eingenommen und geplündert. Gleichzeitig ist Athen zu beschäftigt um sich um das Problem zu kümmern. Verhandlungen innerhalb der Region sind ebenfalls nicht sehr erfolgreich. \\
	Enter Phillip: Er kümmert sich um das Problem und gewinnt dadurch groß an Macht. So kann er ein Bündnis mit Athen eingehen und gewinnt dadurch die Silberminen von Athen. Ebenfalls gewann er einen Sitz im Griechischen Rat, welcher nur Griechen vorbehalten war. So wurde er in einem Zug zu einem Griechen und gewann viele Verbündete innerhalb des Ägäisraums. \\
	\subsection{Griechischer Frieden}
	Danach wird Makedonien als Spieler in Griechenland anerkannt. So wird über den \textbf{Koine Eirene (Griechischer Frieden)} debattiert. Sokrates und Denosthenes debattieren so, ob sie Phillip als Herrscher der Griechen anerkennen sollen. Sokrates Position ist, dass falls sie sich nun nicht verbünden, sie später unterworfen werden und er ruft Phillip den Griechen als Wohltäter zu begegnen. Denosthenes hingegen argumentiert, dass wenn sie ihn nun hineinlassen, er nie wieder gehen wird und sie sich gegen ihn für ihre Freiheit wehren müssen. \\
	Denosthenes kann seine Position erfolgreich verteidigen und sie versuchen sich Makedonien gegenüberzustellen. Dies mündet in der \textbf{Schlacht bei Chaironeia (338)} in welcher Griechenland bedeutend geschlagen wird. Hier hat Alexander auch seine erste Position, was seine strategische Macht beweist. \\
	Nach Chaironeia wird der \textbf{Korinthische Bund} gegründet, welcher ein Bund der Hellener gegen die Perser ist, natürlich unter der Führung Makedoniens. \\
	Dieser Bund war ein Herrschaftsinstrument Philipps um auch die restlichen Griechen in Kleinasien zu "befreien". Philipp hatte eine hegemoniale Stellung gegenüber den Bündnern und war im Kriegsfall auch der alleinige Herrscher. \\
	Theben wird mit einer Garnison von Makedoniern versehen (und wird später dem Erdboden gleichgemacht). Das war eine Warnung, was mit rebellierenden Staaten passiert. Mit Athen hingegen geht Makedonien sehr milde um und zwingt sie nur den Seebund aufzulösen. Makedonien zeigt jedoch höchstwahrscheinlich nur Milde, da sie Schiffe und Matrosen benötigen, welche Athen hat. Zu dieser Zeit ist nahezu der gesamte Hellenische Raum in der Macht Makedoniens. Nur Sparta wurde nicht unterworfen, größtenteils weil die Macht Spartas grundlegend gebrochen war. Nun ist der einzige Ort, welcher noch nicht unter der Herrschaft Makedoniens steht Kleinasien und somit Persien. \\
	Jedoch erlebt Phillip dies nie und wird kurz zuvor auf einer Hochzeit ermordet. Alexander nimmt somit die Führung und beginnt den Persischen Feldzug (Nachdem er innerpolitische Probleme erledigt hat). \\
	Zum gleichen Zeitpunkt versuchen die Griechen jedoch einen letzten Versuch der Unabhängigkeit mit dem \textbf{Lamischen Krieg}, in welchem Athen erneut geschlagen und somit endgütlig gebrochen wurde. Als Konsequenz dieses Aufstandes wird der Korinthische Bund aufgelöst und ganz Griechenland unter direkte Kontrolle von Makedonien gestellt. \\
	\section{Der Hellenismus - 14.04.2022}
	
	Die Epoche des Hellenismus erstreckt sich von Phillip dem zweiten von Makedonien und Alexander dem Großen bis zu dem Eingreifen der Römer im Hellenischen Raum und dem letzten großen Reich von Ägypten: Den Ptolymäern und Cleopatra der Siebten. \\
	Die Grundlage des Hellenismus ist die große politische Umwälzung in dem Raum durch Phillip den Zweiten. Diese Epoche ist auch das Bindeglied zwischen dem griechischen Zeitalter und der Alleinherrschaft der Römer. \\
	Der Begriff Hellenismus ist eine Neuschöpfung, welcher aus einem Missverständnis entstanden ist. Dieser basiert auf Hellenicei, welcher "Griechisch sprechen" bedeutet. Dies rührt daher, dass das Makedonische Reich weit über Mesopotamien herrschte und somit auch die griechische Sprache als Verwaltungssprache verbreitete. \\
	Bei den Quellen der Hellenistischen Epoche ist gut zu erkennen, dass die moderne Ansicht der historischen Epochen wesentlich von den überlieferten Werken abhängt. Im Hellenismus gab es zwar ein literarisches Schaffen, was weit über die vorherigen Epochen ging und es sind über 1000 Autoren bekannt, wovon jedoch nur die wenigsten die Jahrhunderte überlebt haben. Der wichtigste überlieferte Autor ist \textbf{Polybios}, von welchem vergleichsweise viele Werke bekannt sind. Ein Großteil dieser Werke sind jedoch nur Inschriften in verschiedenen Stadtstaaten, welche mühsam entziffert und zusammengefügt werden mussten. Aus diesem Grund ist der Hellenismus lange am Rand der historischen Forschung gewesen. Weitere Quellen sind die Papyrusurkunden, die sogenanntend \textbf{Papyri}, welche aufgrund des trockenen Klimas die Zeit überdauert hat. Da Papyrus aus Ägypten stammte, ist die Beziehung der zwei Länder gut erforscht, da sich diese oft darauf konzentrierten. \\
	Ausgangspunkt ist, wie erwähnt die Eroberungen von Alexander dem Großen. Alexander war laut Überlieferungen stets an vorderster Front zu finden und befehlte die Reiterei direkt. \\
	Die Phalanx, welche den Grundstein der griechischen Armee bildete, wurde durch die Makedonische Phalanx ersetzte, welche weitaus längere Speere besaß und so effektiver und sicherer agieren konnte. \\
	Die Armee der Makedonier soll 40.000 Mann stark gewesen sein. Der Begleitzug soll diese Zahl wiederum verdoppelt haben. Es war eine strategische Meisterleistung knapp 100.000 zu unterhalten und zu führen. Seine ersten Ziele waren Städte, welche von Satrapen der Persier gehalten wurde. Während dieses Kriegszuges sammelte Alexander und die Armee insgesamt wertvolle Erfahrung, welche ihm später sehr nützlich kommt. \\
	Schließlich trifft Alexander bei der \textbf{Schlacht bei Issos} nicht mehr nur auf Satrapen, sondern auf die Armee des Großkönigs von Persien, \textbf{Dareios III.}, welche der makedonischen Armee zahlenmäßig weit überlegen war. Nur durch militärische Taktiken konnte er sich gegen dieses Heer durchsetzen. Alexander hatte jeweils zwei "Flügel", welche aus der Reiterei bestanden, mit welchen er versuchte die gegnerische Armee einzukesseln. Dies geschah mit großem Erfolg und er schlug die feindliche Armee. Von diesem Punkt an hatte er zwei Möglichkeiten:
	\begin{itemize}
		\item{Er konnte entweder das Herzgebiet der Perser in Persepolis, die sogenannte Persis, direkt angreifen}
		\item{Er konnte jedoch auch zuerst die Peripherie, wie Ägypten einnehmen.}
	\end{itemize}
	Im Endeffekt beschloss er zuerst die Peripherie einzunehmen und ging so zuerst auf Ägypten vor. Der Plan war, zuerst eine Seeherrschaft aufzubauen, welche zuvor zwar schon existiert hatte, jedoch bald aufgelöst worden war. Nach Einnahme von Ägypten, setzte er sich als "alter legitimer Herrscher", als Pharaoh, ein. Von Ägypten aus nähert er sich schließlich Mesopotamien, welches den Griechen nur als altes Land bekannt war. Mesopotamien grenzte an Persis, wo er nun als nächstes vorging. \\
	In Persis trifft er erneut auf Dareios und schlägt ihn in der Schlacht von Gaugamela erneut empfindlich. Der Großkönig flieht daraufhin und wird im Endeffekt nicht von Alexander, sondern von seinen eigenen Satrapen erschlagen. \\
	Diese schnelle Ausbreitung führte dazu, dass Makedonien in kürzester Zeit zu einem Vielvölkerstaat wird, wodurch, wie zuvor bei den Persern Richtung Osten, Richtung Persepolis, nun Richtung Westen in den Hellenischen Raum geschah. \\
	Im Endeffekt erobert das makedonische Reich das gesamte Perserreich, welches sich bis in das heutige Afghanistan zieht. Alexander hatte Bestrebungen das "Ende der Welt" zu erforschen und zog so sogar bis in das heutige Indien und die Ausläufer des Himalaya vor. Nachdem klar wurde, dass es von hier aus noch viel weiter geht, entsteht eine kleine Meuterei in den Truppen, welche ihn zur Umkehr zwingt. Bei der Rückkehr dringt er noch bis an den Persischen Golf vor und nimmt diesen ebenfalls ein. Das Makedonische Reich hatte für seine Zeit eine fantastische Ausbreitung und setzte den Grundstein für das Römische Reich die gesamte Mittelmeerküste einzunehmen. Alexander selbst stirbt nur ein Jahr nach Beendigung seines Feldzuges in Babylon. Die Umstände seines Todes sind zwar unbekannt, jedoch wird angenommen, dass die großzügigen Trinkgelage mit seiner Armee, welche auch zu Wettbewerben führte, hierbei geholfen hat. Alexander hatte zu diesem Zeitpunkt keinen Erben ernannt, wodurch es an seinen Inneren Zirkel fiel. Seine Geliebte hatte zwar einen Sohn, dieser war jedoch geistig nicht imstande das Reich zu leiten. \\
	Im Endeffekt überschatteten die individuellen Interessen der Mitglieder des Zirkels und das Reich zerfiel in verschieden Diadochenreiche. Die Diadochen, welche "Nachfolger" bedeuten, führten zuvor jedoch blutige Kriege der Vorherrschaft. Dies endete erst nach der Schlacht von Ipsos 301. v.Chr., 22 Jahre nach Alexanders Tod. So entstanden vier kleinere Reiche, jeweils unter der Herrschaft eines Diadochen. 
	\begin{figure}[H]
	\includegraphics{Diadochen.jpeg}
	\caption{Diadochenreich nach der Schlacht von Ipsos 301 v.Chr.}
	\end{figure}
	\vspace{1cm}
	Die Reiche erstreckten sich in etwa:
	\begin{itemize}
		\item{Im Kerngebiet von Griechenland}
		\item{Eine Hälfte von Kleinasien}
		\item{Das ursprüngliche Perserreich}
		\item{Ägypten}
	\end{itemize}
	\subsection{Diadochenreiche}
	30 Jahre später waren noch drei dieser Reiche vorhanden:
	\begin{itemize}
		\item{Seleukidenreich}
		\begin{itemize}
			\item{Persien sowie Kleinasien. Wird bald durch Östliche Feldzüge verkleinert.}
		\end{itemize}
		\item{Antigonidenreich}
		\begin{itemize}
			\item{Teile von Griechenland, jedoch nicht alles}
		\end{itemize}
		\item{Ptolemäerreich}
		\begin{itemize}
			\item{Ägypten sowie Teile der Küste in Richtung des heutigen Tunesien}
		\end{itemize}
	\end{itemize}

	Zusätzlich existiert später noch das Reich von Pergamon, welches nicht als Diadochenreich bezeichnet werden kann, da es erst später etabliert wurde. Zusätzlich existierten noch immer zahllose Stadtstaaten in den Gebieten.
	\subsubsection{Ptolemäerreich}
	Ptolemaios, welcher der Herrscher seines Reiches war, hatte von Anfang an Bestrebungen Ägypten zu übernehmen. Er war sich der Abgeschiedenheit und daraus resultierenden einfachen Verteidigung, bei großer wirtschaftlichen Bedeutung, bewusst und wollte sein Reich sichern. Das Kerngebiet ist Ägypten, zieht sich jedoch noch weiter und hielt ebenfalls Teile von Kleinasien und Zypern sowie der Küste von Nordafrika. Das Ptolemäerreich hatte eine deutliche Überproduktion von Weizen, was größtenteils exportiert wurde und somit zu einem großen Spieler im Handel wurde. Zentrum des Reiches war \textbf{Alexandria}, welches von ALexander gegründet wurde, als er es eingenommen hatte. Der große Ausbau geschhah jedoch erst nachdem das Ptolemäerreich gegründet wurde und wurde durch den Handel in kürzester Zeit zur größten Stadt des Mittelmeerraums. \\
	Die zuvor existierende Stadt von Herakleios verlor durch die Versandung des Nils zusehends an Bedeutung, welches von Alexandria abgelöst wurde. \\
	Auch waren die Nordwinde, welche im Ägäisraum herrschten ein Faktor in der Schifffahrt in der Umgebung, da man von Griechenland durch diese Winde oft bis nach Alexandria, manchmal noch weiter bis in das heutige Palästina, getragen werden konnte. Während die Griechen davon nie voll profitieren konnten, da Teile nicht in ihrem Besitz war, konnte das Makedonische Reich dies nun ausnutzen. Gleich wie Alexander davor, wurde auch das politische System beibehalten, während lediglich die Makedonische Elite auf die Spitze gesetzt wurde. So lebten die Ptolemäer als Könige Ägyptens ebenfalls als deren Pharaoh. Die Könige waren stets Nachfolger von Ptolemaios und nahmen ebenfalls den Namen Ptolemaios an, wodurch diese heute nummeriert werden. Kleopatra war ebenfalls Teil dieses Hauses und sie war nicht die erste ihres Namens, sondern die Siebte. \\
	Das Volk blieb in dieser Zeit größtenteils das Selbe und es gab keinen Austausch von Ägyptern durch Griechen. Es wurden jedoch ein paar griechischen Städte gegründet. \\
	Für die Ägypter änderte sich zu diesem Zeitpunkt nicht viel, zuvor wurden sie durch einen Satrapen regiert und nun waren es die Makedonier. \\
	\subsubsection{Seleukidenreich}
	Das Seleukidenreich, begründet durch Seleukos, erstreckte sich über große Teile von Persien und Kleinasien und war im Gegensatz zum Ptolemäerreich ein sehr heterogenes Reich. Durch das Zweistromland ist es auch sehr fruchtbar und hat großen Einfluss in dem Gebiet. Das Seleukidenreich war zwar am Anfang mit Abstand das größte der Diadochenreiche, verlor jedoch sehr bald an Land. Vor allem im Osten werden große Teile sehr bald zurückerobert. Ebenfalls verlieren sie Teile Kleinasiens bald an das römische Reich. 150 Jahre später ist von dem Seleukidenreich nur noch eine kleines Gebiet in der Levante um das heutige Lebanon übrig. Anders als das Ptolemäerreich gründete das Seleukidenreich sehr viele Städte zur Machtfestigung. Ebenfalls prägte sich eine Aristokratie und Herrscherstruktur, welche sich an der Regierung beteiligt. Gleich wie bei Ptolemaios in Ägypten, konnte Seleukos auch seine Herrschaft etablieren indem er sich an die bestehende Machtstruktur anband. Mit der Zeit sagten sich jedoch sehr viele Lokalherrscher innerhalb des Reiches von dem Großherrscher los und gründeten ihre eigenen Reiche. Mit der Eroberung wurden auch Führungseliten aus Griechenland und Makedonien in das Reich gebracht, welche die Administration übernahmen. Diese Eliten wurden an strategischen Punkten innerhalb des Reiches eingesetzt um die Verwaltung zu übernehmen. Auch dies wurde einfach von den Persern mit dem Satrapensystem übernommen und fand keine große Ablehnung in der Bevölkerung.
	\section{Hellenistische Diadochenreiche - 05.05.2022}
	\subsection{Die Antigoniden - Makedonisches Kernland}
	Der Name geht auf Antigonos II. zurück. Zwar hatte Alexander Nachkommen, diese wurden jedoch bald innerhalb der Nachfolgekriege beseitigt wodurch diese Linie endete. Im alten Makedonien gab es während der Diadochenkriege mehrere Machtwechsel und erst 270v. Chr., also 40 Jahre nach Alexanders Tod konnte sich Antigonos durchsetzen, was größtenteils seinen persönlichen Fähigkeiten zu verdanken ist. Antigonos entstammte einer einflussreichen Adelsfamilie aus Makedonien. Sein Vater war ein Spieler innerhalb der ersten Diadochenkriege, wo er jedoch alsbald unterlag. Während seine Verwandten zuvor unterlagen konnte er sich hingegen durchsetzen und sein eigenes Reich gründen. Währen der anderen beiden Reiche gleich groß blieben oder, wie das Seleukidenreich stetig kleiner wurden, vergrößerte sich das Antigonidenreich leicht indem es in dem Ägäisraum expandierte. Die Stadtstaatenreiche waren zu dem Zeitpunkt sehr schwach und auch unabhängig, was es dem Reich einfach machte. Mit der Zeit verbünden sich jedoch viele Stadtstaaten um diesem Einfluss widerstehen zu können, wodurch mit der Zeit der Einfluss solch expandierenden Herrscher zurückgedrängt wird.
	\subsection{Attalidenreich}
	Das Reich von Attalos dem Ersten, ausgehend von Pergamon, breitet in der selben Zeit seinen Einfluss aus, ist jedoch historisch gesehen von den Hellenistischen Reichen unabhängig. Die größte Ausbreitung fand in das Hinterland von Kleinasien statt, konnten sich jedoch nicht gegen die Hellenistischen Reiche durchsetzen. Es war auch Pergamon, welches um ein weiteres Bündnis zu erlangen, Rom in dem Bereich ins Spiel brachte. Es ist Rom, welches danach in kürzester Zeit die Vorherrschaft in dem Gebiet erlangt, das wird jedoch erst später aus römischer Sicht näher behandelt. \\
	Während die Diadochenkriege eine sehr kriegerische Zeit war, nahm dies nach dessen Beendigung nur unmerklich ab und es gab ständig weitere Vorherrschaftskämpfe. Pergamon konnte sich hier zuteils behaupten und seinen eigenen Herrschaftsbereich, und nicht Königreich etablieren. Der Burgberg von Pergamon war der Herrschaftspalast des Reiches und unterhielt auch eine große Anzahl an Wissenschaftlern, Künstlern und Philosophen, konnte jedoch nicht mit Ägypten und Alexandria konkurrieren, da diese einfach größere Geldmittel hatten. \\
	\subsection{Koiná - Bundesstaaten}
	Ein Koinon (Plural Koiná) ist in etwa ein Bundesstaat, jedoch nicht mit heutigen Bundesstaaten zu verwechseln, in welchem sich mehrere Stadtstaaten zu einem Bündnis verbanden. Die beiden größten Koiná waren jeweils der \textbf{Aitolische Bund} und der \textbf{Achäische Bund}. Ein wichtiges Merkmal war, dass die Mitglieder der Stadtstaaten nur ihre Außenpolitik aufeinander abstimmten und diese auch gemeinsam militärisch verteidigten. Die Innenpolitik blieb jedoch eine Sache der einzelnen Stadtstaaten. Die Größe dieser Bündnisse war jedoch keineswegs statisch und fing mit anfänglich kleinem Ausmaß zu einer relativen Großmacht. Speziell der Achäische Bund umfasste zu einem Zeitpunkt die gesamte Peloponesische Halbinsel. Durch diese Machtstrukturen konnte die Stadtstaatenwelt für eine relativ lange Zeit bestehen bleiben konnten. Selbst als diese Teil von größeren Reichen wurden, konnten sie ihre Tradition und Kultur größtenteils behalten. Trotz diesen Umstandes konnten jedoch nur wenige Stadtstaaten langfristig unabhängig bleiben. Das Paradebeispiel hierfür ist \textbf{Rhodos}. Rhodos hatte eine Schlüsselposition im Handel zwischen den größeren Reichen, wodurch es mit einer relativ kleinen Flotte frei von außenpolitischen Einflüssen bleiben konnte. \\
	Zur gleichen Zeit setzt sich die Demokratie in vielen der Stadtstaaten durch und es gibt nur mehr sehr wenige Oligarchien. In diesen Stadtstaaten konnten die Bürger, welche sich oft nur auf ein paar tausend belief, Einfluss auf die Politik nehmen, inklusive Volksversammlungen und Umfragen. Dies rührt daher, dass Bürger der Stadtstaaten nicht nur politische, sondern auch militärische Bürger waren und es wurde von ihnen erwartet, dass diese es im Notfall auch verteidigen, wodurch sie auch eine Einflussnahme an Außenpolitischen Fragen erwarteten. \\
	\subsubsection{Stadtstaaten in Kleinasien}
	Zu diesem Zeitpunkt wurden auch viele Stadtstaaten an der Küste Kleinasiens gegründet, während diese jedoch selten unter Eigenverwaltung standen und oft von dem sendenden Reich kontrolliert wurden. Dies bedeutete jedoch auch, dass die Kultur der Stadtstaaten in den Osten getragen wurde. \\
	\subsubsection{Hellenistische Militärentwicklungen}
	Mit der Hellenistischen Ära kamen auch zahllose militärische Neuerungen. Zusätzlich zu der bereits bekannten Makedonischen Phalanx wurden komplexe Belagerungstürme gebaut. Klassische griechische Mauern konnten diesen sehr selten standhalten, da diese einfach überwunden wurden.
	\subsection{Handelswege}
	Zu diesem Zeitpunkt veränderten sich auch die Handelswege zunehmends. So wird der Ostägäische Raum immer vernetzter und Resourcen und Güter werden aus immer entlegeneren Gebieten importiert und in das Gebiet eingespeist. So bilden sich einige neue Handeslbeziehungen. Später kommen hierzu auch Routen im westlichen Mittlemeerraum hinzu, wo Rom die Vorherrschaft hat. So wird der gesamte Mittelmeerraum sehr vernetzt und Handel in diesem Gebiet blüht auf. Im Zentrum dieser kleinen Globalisierung liegen wiederum die griechischen Stadtstaaten.
	Auch verwenden die Königreiche ihre Resourcen um ihre eigenen Münzen zu prägen, welche oft die Gesichter der Herrscher tragen, um damit zu handeln und auch ihre Armee unterhalten zu können. \\
	\subsection{Alexandria}
	Alexandria hat zu diesem Zeitpunkt den Status einer der wichtigsten Städte zur Forschung und Philosophie erlangt. Dies wurde ermöglicht durch die großzügigen Gelder der Ptolemäer. Teil dieses Komplexes war die \textbf{Bibliothek von Alexandria}, welche es sich zur Aufgabe gemacht hatte das gesamte Wissen der bekannten Welt zu sammeln und zu diesem Zeitpunkt das wahrscheinlich wichtigste Wissenszentrum war. \\
	\subsection{Pergamon}
	Ähnliches gab es in Pergamon. Der Herrscheraltar, welcher das Kronjuwel des Kultes in Pergamon war, steht heute in Berlin um welches das Pergamonmuseum gebaut wurde, nachdem es im späten 19. Jh. in Pergamon abgebaut und nach Deutschland gebracht wurde. \\
	Der \textbf{Antikythera-Mechanismus} ist eines der Beispiele herausragender hellenistischer Technik und Wissenschaft. Gefunden als korrodiertes Stück Metall, konnte man durch Analysen ein System aus 30 Zahnrädern erahnen, welches relativ akkurat Ereignisse wie Olympische Spiele und Sonnenfinsternissen vorhersagen konnte. Technik von diesem Level soll erst weider 1500 Jahre später entwickelt werden. \\
	Eine weitere Errungenschaft in der Architektur ist die \textbf{Stoa von Attalos}, welche von Pergamon in Athen gespendet wurde. Viele der Heiligtümer zu griechischen Göttern lassen auch große Gebäude erahnen, wie das Asklepios-Heiligtum in Kos, oder das Heiligtum der Götter in Delos. Die Architektur der Heiligtümer ist dezidiert nicht orientalisch, sondern baut auf dem klassischen Stil auf.\\
	\subsection{Römisches Reich}
	Rom, welches als kleines Dorf begann kann in grobe zeitliche Teile gebildet werden:
	\begin{itemize}
		\item{Das legendäre Gründungsdatum des römischen Reiches ist 753. v.Chr., was jedoch nicht bewiesen werden kann.}
		\item{Nach der "Gründung" dauerte die Königszeit bis 509. v.Chr.}
		\item{Danach wurde die Römische Republik gegründet:}
		\begin{itemize}
			\item{Die frühe Republik bis ca. 287.v.Chr.}
			\item{Hohe/Mittlere Republik bis ca. 133 v.Chr.}
			\item{Späte Republik bis zur Schlacht von Actium und dem Aufstieg Oktavians um 27 v.Chr.}
		\end{itemize}
	\end{itemize}
	Eine weitere mögliche Gliederung sind die Entwicklungen Roms.
	\begin{itemize}
		\item{Die politische Organisation Roms}
		\begin{itemize}
			\item{Die zentralen politischen Ämter Roms}
			\item{Der \textit{cursus honorum}}
		\end{itemize}
	\end{itemize}
	In der inneren Entwicklung hatte Rom mehrer große Meilensteine. Innerhalb der römischen Gesellschaft gab es eine Aristorkatie und die restlichen Bürger (Patrizier und Plebejer), welche oft zu Auseinandersetzungen kamen. Als Konsequenz dieser Auseinandersetzungen enstand schließlich die Nobilität, welche sich aus Patriziern und Plebejern zusammensetzte. \\
	Doch auch danach gab es innenpolitische Auseinandersetzungen. Ausgelöst durch die Verarmung von Teilen der Bevölkerung wurde versucht diese wieder zu verstärken. Später erlangt Oktavian/Augustus die Macht und wandelt Rom in eine quasi-Monarchie um. \\
	Die Frühzeit Roms ist historisch gesehen sehr unklar. Bis zum Aufstieg Roms als vorherrschende Macht des Gebiets gab es sehr wenige Aufzeichnungen aus erster Hand. Die wichtigsten Autoren dieser Zeit sind \textbf{Polybios} und \textbf{Livius}. Selbst Polybios, welcher um 200 v. Chr. lebte musste 400 Jahre in die Vergangenheit schreiben. Livius, welcher zu Beginn der Kaiserzeit um 0 lebte schrieb so über einen Zeitraum welcher bereits 700 Jahre zurücklag und sich auch viel auf Polybios bezieht. \\
	Interessanterweise ist das Ende der Römischen Königszeit, in etwa dem selben Zeitraum geschehen wie die Stürzung der Tyrannen im klassischen Griechenland. Zwar ist der genaue Zeitpunkt dieses Endes unklar, man kann jedoch annehmen, dass es etwa gleichzeitig und wahrscheinlich unabhängig geschah.
	\section{Rom - 12.05.2022}
	Rom entstand als Teil der Hochkultur der Etrusker und war im Vergleich sehr unscheinbar. In der Region gab es unzählige 
	Sprachen, wie das Etruskische und das Okzitanische, wobei Latein eigentlich nur einen sehr kleinen Bereich einnahm. \\
	Rom ist am Tiber gelegen und war ein regulärer Stadtstaat. Das Forum, welches später das Zentrum wurde, war zuerst ein sumpfiges Gebiet außerhalb der Mauern.\\
	Rom als Stadtstaat entstand in einer Welt, welche von Phöniziern und Griechen regiert wird. Es gab zahlreiche Siedlungen im italienischen Raum, gegründet von diesen zwei Kulturen. Zu diesem Zeitpunkt war es nicht unbedingt ersichtlich, dass Rom zur vorherrschenden Macht im mediterranen Raum werden würde.
	\subsection{Innere Entwicklung}
	Rom als Republik, nachdem der König entfernt wurde, sah sich selbst als Staat, welcher nicht von einer Person regiert werden wollte und die Macht in der Hand viele lag. So wurde die "Kollegialität", dass stets zwei \textit{Konsuln} als Kollegen das oberste Amt einnahmen, sehr bald danach eingeführt. Unter diesen standen die 8 \textit{Prätoren}, welche zusammen mit den Konsuln auch Kriegsgewalt hatten und ein Imperium befehlen konnten, jedoch auch für die Rechtssprechung verantwortlich waren. Unter diesen standen jeweils 2 \textit{kurulische Ädilen} und 20 \textit{Quästoren}. Dies war hierarchisch aufgestellt, sodass der jeweils höher immer dem unteren Befehle geben konnte. \\
	Die Plebejer hatten eine eigen Volksversammlung und Ämter, welche nur von Plebejern eingenommen werden konnten. So gab es 10 \textit{Volkstribune} und 2 \textit{Plebejische Ädilen}, welche jedoch gleichgestellt waren, auch wenn das Schema diese übereinander aufstellt. Später erfuhren die Volkstribune speziell noch viel mehr Aufmerksamkeit. Außerhalb dieser beiden Systeme existierten die \textit{Zensoren}, welche stets nur von den höchsten Amtsträgern eingenommen werden konnten. Ihre Aufgabe war, dass diese den Senat ernennen und auch wirtschaftliche Geschicke lenken. \\
	Speziell erwähnt werden sollte das Amt des \textit{Diktators}, welches einem der beiden Konsuln übergeben werden konnte. Dieser wurde unterstützt von einem \textit{Magister equitur}. Während Diktator heute eine sehr negative Konnotation hat, war es zu diesem Zeitpunkt ein sehr ehrenvolles Amt, welches gebraucht wurde, um Gefahren abzuwenden, welche den Staat grundsätzlich gefährden. Während dies in der frühen Republik zutraf, entartete dies mit der Zeit zunehmends indem mehr und mehr Befugnisse dem Diktator übertragen wurde, was dann eher dem Bild entspricht was wir heute von Diktatoren haben. \\
	Diese politische Struktur diente jedoch nicht nur um innerpolitische Geschicke zu leiten, sondern auch, oder eventuell noch mehr, um eine effektive militärische Führungsspitze zu besitzen. Hier waren die Konsuln und Prätoren von instrumentaler Bedeutung. Es war auch das finale Ziel der Aristokratie an die Spitze dieses Systems als Konsul zu gelangen. \\
	Wie erwähnt, ist die Frühzeit von Rom nur sehr schwer fassbar. Es ist bekannt, dass Rom Krieg mit den latinischen Städten geführt hat. Das erste Ziel, ein etruskischer Stadtstaat, \textbf{Veji} wird eingenommen, die Bevölkerung beseitigt, und das Gebiet komplett in Rom einverleibt. Durch diese Eroberung wird Rom zu einem der größten Reiche im lokalen Gebiet. Kurz danach fallen die Kelten in die Umgebung ein, und nehmen Rom auch für kurze Zeit ein. Jedoch ziehen sich diese bald wieder in die Po Region zurück. Dies ist die erste Lektion, dass unvorhergesehene Bedrohungen, welche viel stärker sind, einfach auftauchen können. \\
	Schließlich, im Jahre 300 v. Chr., legt sich Rom mit anderen latinischen Städten an um seine Vormachtstellung auszubauen. \\
	\begin{figure}
	\includegraphics{Samnitenkriege.jpeg}
	\caption{Die Kriege und Ausdehnung des Römischen Reiches}
	\end{figure}
	Die erste größere Auseinandersetzung Roms sind die \textbf{Samnitenkriege}. Die Samniten sind ein Oberbegriff für unabhängige Stämme, gegen die Rom stets Kriege geführt hat. Hier erlitt Rom auch öfters Niederlagen, welche dazu führte, dass Rom seine militärische Taktik anpasste und die Phalanx der Griechen zugunsten andere Strategien aufgab. \\
	Ein Detail ist, dass Rom, selbst nach einer Niederlage, diese, mit einem neu aufgestellten Heer, stets erneut angriff und so im Endeffekt siegen konnte. Dies ist auch wichtig bei den Punischen Kriegen. \\
	Erfolgreich durch diese Kriege weitet Rom sein Einflussgebiet stets weiter aus. Schließlich zieht Rom gegen \textbf{Tarantum} vor, welches dem nunviel größeren Staat nichts entgegensetzen konnte. Aus diesem Grund riefen sie den Griechen \textbf{Pyyrhos} um ihnen in ihrem Krieg zu unterstützen. Pyyrhos war zwar einigermaßen erfolgreich und konnte auch Siege feiern, war jedoch irgendwann nicht mehr in der Lage ein Heer aufzustellen. Hier rührt der Begriff des \textit{Pyyrhischen Sieges} her, welches ein Sieg ist, jedoch zu einem so hohen Preis, dass man im Endeffekt doch verliert. Pyyrhos kapituliert danach im Forum und uberlässt die griechischen Siedlungen Rom. \\
	Jetzt wo Rom den gesammten Mittel- und Süditalienischen Raum beherrschte, stellt sich die Frage, wie es alles beherrscht werden konnte. Dieses System der Eingliederung der eroberten Gebiete ist der entscheidende Faktor in der Herrschaft Roms. Es ist jedoch komplett anders als das später eingeführte System der Provinzen. Rom verhandelte mit jeder Macht einzeln, also gab es keinen allgemeinen Friedensvertrag, und wurde abhängig von der Schwere des Krieges, Anzahl der Verluste etc. abgewogen. Danach wurde jede Partei ein Bundesgenosse innerhalb des römischen Systems, weshalb man auch von dem italienischen Bundesgenossensystems spricht. Ein Genosse durfte jedoch nicht mit einem anderen Genossen Bündnisse schließen, was einen Verrat an Rom verhindern sollte. Zusätzlich verlangte Rom von jedem neuen Bündner, dass diese ihr Heer unterstützen. Hierher rührt auch die militärische Macht der Römer, da so die Last der Heeresstellung und dadurch auch die Verluste, stets auf viele Städte verteilt wurde. \\
	Um 225 v. Chr. kontrollierte Rom schließlich den gesamten italienischen Raum, wobei etwa 800.000 Personen im Reich lebten. Hierbei waren etwa 40\% davon Vollbürger, während die restlichen 60\% Genossen (Socii) waren. \\
	Rom gründete danach auch Kolonien in Italien. Dabei gab es unterschiedliche Stellungen der Siedlungen. Das Vollrecht genossen Kolonien des \textit{römischen Rechts}, während Kolonien des \textit{latinischen Rechts} weniger Rechte hatten. \\
	Wie bereits erwähnt gab Rom bald die Strategie der Phalanx zugunsten mobilerer Formen auf. Zusätzlich griff Rom in ihrer Kriegsführung auch auf die Plebejer zurück. In der Klassik war Krieg größtenteils eine Sache der Aristokratie, wobei normale Bürger hierbei keine Rechte besaßen. Indem Plebejer Teil des Heeres wurden, verlangten diese jedoch auch mehr Rechte im römischen politischen System, wobei diese Zugang zu mehr politischen Ämtern haben wollten. Dies mündete in den \textbf{Ständekriegen}, in welchen Plebejer Rechte für ihre militärische Teilnahme forderten, was in einem Auszug der Plebejer aus Rom gipfelte um diesen Maßnahmen zu protestieren. Schließlich erhielten Plebejer Zugang zu den politischen Ämtern und konnten somit auch Konsuln werden. Gleichzeitig erhielten Volkstribune zusätzliche Macht, indem die Beschlüsse der Plebejer für alle römischen Bürger galt, wodurch sie sich auch direkt gegen den Senat, das historische Machtzentrum, wenden konnten. \\
	Während dies relativ demokratisch wirkt, kann man nicht von einer Demokratie sprechen, da die Aristorkatie doch sehr einflussreich war. Zusätzlich war der Senat immer noch das lenkende Organ, welches auch von einem "Rat" der früheren Amtsträger beraten wurde, so jedoch sehr viel Einfluss auf die Geschehnisse hatten. So spricht man oft von einer sogenannten \textit{Mischdemokratie}, da es sowohl demokratische als auch monarchische Elemente enthielt. So hatten die Konsuln und auch der Diktator absolute Macht, was einem König gleicht. \\
	Durch den Zugang, welchen Plebejer so zu den politischen Ämtern hatten, entstand eine Mixtur aus Plebejern und Aristokraten, welche stets große monetäre Mittel hatten, nämlich die Nobilität. \\
	Nach der Eroberung Italiens und der Schlichtung der innerpolitischen Verhältnisse, war Rom über die nächsten 100 Jahre in zahllose Kriege verwickelt:
	\begin{itemize}
		\item{264 - 241 v. Chr. - Erster Punischer Krieg}
		\item{218 - 202 v. Chr. - Zweiter Punischer Krieg}
		\item{215 - 202 v. Chr. - Erster Römisch-Makedonischer Krieg}
		\item{200 - 197 v. Chr. - Zweiter Römisch-Makedonischer Krieg}
		\item{192 - 188 v. Chr. - Römisch Syrischer Krieg}
		\item{171 - 168 v. Chr. - Dritter Römisch-Makedonischer Krieg}
		\item{149 - 146 v. Chr. - Dritter Punischer Krieg}
		\item{146 v. Chr. - Zerstörung Korinths und das Ende des Achäischen Bundes}
	\end{itemize}
	Das Phönizische Reich, von den Römern als Punisches Reich bezeichnet, mit der Hauptmacht in Karthago, kontrollierte die küstennahen Gebiete in Nordafrika, Spanien und Korsika bzw. Sizilien. Rom war in Italien ausnahmslos eine Landmacht, während Karthago größtenteils eine Seemacht war. Mit der Zeit etablierte Rom sich jedoch auch als Seemacht und errang auch die Vormachtsstellung. \\
	Im \textbf{ersten punischen Krieg} ging es größtenteils um Sizilien, welches Rom entscheidend gewann. Ein überlieferter Siegesgrund waren Enterhaken, welche auf die Phönizischen Schiffe fielen wodurch Rom seine Truppen zu See effektiv einsetzen konnte. Jedoch ist diese Verwendung nur am Anfang des ersten punischen Krieges bewiesen.\\
	Nach dem Sieg Roms über die Phönizier im ersten Krieg, nehmen diese Sizilien, Sardinien und Korsika ein. Hier wird das vorher eingesetzte System der Bundesgenossen nicht mehr verwendet und stattdessen Provinzen gegründet, in welchen ein Statthalter eingesetzt wurde um sie zu verwalten. Diese Statthalter, welche anstelle eines Konsuls und Prätors, also als Prokonsul und Proprätor, in die Provinzen gingen, setzten dort den Willen Roms durch. Dies geschah, weil Rom wegen der Distanz es nicht effektiv in das Bundesgenossensystem eingliedern konnte. Somit wurden auch die hohen politischen Ämter in Rom immer einflussreicher da man somit nicht nur in Rom sondern auch in den Provinzen Einfluss ausüben konnte.

	\section{Römische Kaiserzeit - 19.05.2022}

	Der erste römische Kaiser war \textbf{Augustus}, zuvor Oktavian. Um Augustus zu verstehen muss man dessen Entstehungsgeschichte verstehen. Nachdem Cäsar ermordet worden war, wurde das zweite Triumvirat gegründet. Dieses Triumvirat bestand aus \textbf{Octavian, Marc Anton und Lepidus}. Auf dem Papier war Octavian der Erbe Cäsars, jedoch war es de facto Marc Anton, welcher dessen Erbe fortführte. Lepidus war ein General unter Cäsar und es war gut eine dritte Person in diesem Rat zu haben, jedoch hatte dieser weniger Einfluss als jeweils Octavian und Marc Anton. \\
	Innerhalb des Triumvirats gab es stets Streitigkeiten, was schließlich zum Zerfall des Triumvirats führte, was auch den römischen Bürgerkrieg auslöste. Octavian versuchte stets Marc Anton zu diskreditieren, indem er hervorhab, dass er ja eigentlich gar kein Römer mehr ist. Das erklärte er, weil so viel Zeit in Ägypten und mit Cleopatra verbrachte, er sich nicht mehr wie ein Römer kleidete, wie einer sprach und sogar dessen Kinder mit ihr anerkannte. Das war ein Problem, weil somit nichtrömer römisches Gebiet erben konnten. Zusätzlich wollte er in Alexandria begraben werden und nicht in Rom, was ein Skandal war. Dies fand Octavian nur heraus, weil seine Schwester eine Vesper war, welche den letzten Willen verwalteten. Die Veröffentlichung dessen war höchst illegal, jedoch interessierte das niemanden mehr, nachdem es bekannt geworden war. \\
	Octavian wusste, dass er nicht wie Cäsar vorgehen konnte, da dieser wegen seiner Betsrebungen ermordet worden war. Octavian war, auch, obwohl er gute Diplomat und Staatsmann war, kein militärischer Stratege oder General. Diesen Umstand versucht er stets zu verheimlichen, da es eine große Schande war. Um trotzdem militärisch etwas zu erreichen, stellte er seinen Kindheitsfreund \textbf{Agrippa} ein. Dieser war ein Universalgelehrter und zusätzlich als Autor und General Architekt, wodurch er auch die Baupläne des Pantheons in Rom geschaffen hat. Historiker beschrieben Augustus lange große militärische Siege zu, welche eigentlich von Agrippa begangen war. \\
	Nach Cäsars Tod, musste Augustus alle alten Politiker loswerden. Die größte Hürde war Lepidus selbst, da dieser ein enger Vertrauter Cäsars gewesen und \textit{Pontifex Maximus}, ein Amt auf Lebenszeit, war. Der Pontfiex Maximus war ein sakrales Oberamt. Dieser starb 12 v. Chr., wonach Augustus sich selbst zum Pontifex Maximus ernannte. Zusätzlich löste er das Triumvirat auf und gab sich selbst die Befugnisse des Senats, der Beamten und der Gesetzgebung. Adlige, welche diese Änderung eventuell nicht befürtworten würden, wurden großzügig mit Geldern und Ämtern ruhiggestellt. \\
	Auch gab es keine Kriege mehr, wodurch er Soldaten mit Land in Rom abspeißen konnte. So waren, als Octavian Augustus wurde, die wenigstens noch mit der Republik vertraut. Jüngere Römer waren nach der Schlacht bei Actium, und Marc Antons Untergang, geboren und selbst ältere Bürger kannten nur Bürgerkrieg. Nachdem die Bürgerkriege vorbei waren, fügte er alles wie es zuvor gewesen war, um den Eindruck zu erwecken, dass alles wie immer sei. \\
	\subsection{Machtbasen Augustus'}
	Die Verfassung Roms besagte, dass es keinen Alleinherrscher gab und auch Augustus hatte stets einen Amtskollegen als 'erster Bürger Roms'. Es gab drei große Machtstrukturen innerhalb des Römischen Kaiserreichs:
	\begin{itemize}
		\item{Innenpolitik}
		\begin{itemize}
			\item{Er kontrollierte das Konsulat und den Volkstribun, welcher ein Vetorecht besaß. So konnte er Kontrolle auf patrizische und plebejische Ebene ausüben.}
		\end{itemize}
		\item{Militär}
		\begin{itemize}
			\item{Augustus errichtete die \textit{aerarium militare}, eine Art Kriegskasse, welche er selbst finanzierte und Veteranen versorgte. So band er die Loyalität des Militärs an sich selbst.}
			\item{Augustus gab sich selbst das \textit{imperium proconsulare maius}, was ein Amt war, welches über den Prokonsuln stand, womit er die Kontrolle über alle Legionen im gesamten Reich nehmen konnte.}
		\end{itemize}
		\item{Religion}
		\begin{itemize}
			\item{Mit Lepidus' Tod wurde Augustus zum \textit{Pontifex Maximus}. Als Pontifex Maximus hatte er Kontrolle über alle sakralen Beamten und unter anderem auch direkten Einfluss auf den Festkalender. Dieser Kalender entschied, wann alle große Feste und Feiern gefeiert wurden.}
			\item{Zusätzlich wurden alle Gesetze sowie Kriegs- und Friedenserklärungen den Göttern (Also dem Pontifex Maximus) präsentiert wodurch er auch auf die Gesetzgebung Einfluss hatte.}
		\end{itemize}
	\end{itemize}
	\begin{wrapfigure}{L}{0.6\textwidth}
	\includegraphics{Kaiserreich_Augustus.jpg}
	\caption{Direkt dem Kaiser unterstehende Provinzen}
	\end{wrapfigure}
	Nun könnte man sich fragen wo Augustus das Geld her hatte um so großzügig zu sein. Ein Großteil des Reiches war direkt im Besitz des Kaisers, wodurch auch dessen Einkünfte direkt in seine Privatkasse flossen. Alles was nicht von dem Kaiser kontrolliert wurde, war in Hand von Statthaltern, sogenannten \textit{legati Augusti pro praetore}, welche ehemalige Konsuln und Prätoren waren, und so wieder Augustus loyal waren.\\
	Augustus war so erfolgreich, weil er sein eigenes Kaisertum nur im Hintergrund aufbaute und im Vordergrund immer noch die res publica existierte. Cäsar versuchte sich selbst zum \textit{rex}, zum König, zu ernennen. Die Bevölkerung Roms verband den Titel des Königs mit Fremdherrschaft, da Könige zuvor stets etruskisch waren und so von außen herrschten. Aus diesem Grund nannte Augustus sich selbst nie König oder gar Diktator. Ebenfalls wird nirgends erwähnt, dass es sich um eine Erbmonarchie handelte. Er nannte sich jedoch in der öffentlichkeit stets \textit{primus inter pares}, der Erste unter gleichen. So schuf Augustus dieses Kaiserreich, wobei sich niemand an die 'alte' Republik erinnern konnte und es als neues normal akzeptiert wurde. \\
	\subsection{Reformen Augustus}
	Augustus lehnte sich jedoch nicht zurück, nachdem er die Macht an sich gerissen hatte. So führte er Wachstationen in Bezirken ein um Brände zu verhindern. Man kann sagen, dass Augustus die Feuerwehr erfunden hat. Ebenfalls baute er Militärposten auf, wodurch viele nächtliche Überfalle verhindert werden konnten. Auch führte er sehr viele Gesetze der Moral ein. So wurde Ehebruch strenger geahndet und auch Keuschheit durchgesetzt. Er legte auch viel Wert auf die Verhinderung des Mischens von Blut von Ausländern, wodurch Bürger nur Bürger heiraten durften und das Bürgerrecht nur in seltenen Fällen vergeben wurde. \\
	In einem Propagandaschreiben names \textit{Res gestae Divi Augusti}, gefunden in Ankara, weshalb es auch \textit{Monumentum Ancyranum} genannt wird, zeigt wie er sich selbst als Herrscher beschrieb. Dass er auf Wunsch des Volkes seine Amtsgewalt akzeptiert hatte und trotzdem nur in persönlichem Einfluss über anderen stand, jedoch nicht in politischem Einfluss. \\
	Ein weiteres Mitglied von Augustus' Zirkel, war \textbf{Maecenas}, welcher für die Propaganda im Reich verantwortlich war. Bei Maecenas waren die zwei größten Dichter der Zeit, \textbf{Vergil} und \textbf{Horaz}, welche oft unfreiwillig für Augustus arbeiteten. Es ist überliefert, dass Augustus Vergil oft Briefe schrieb um ihn zur Vollendung seines mangum opus, der \textit{Aeneis} zu drängen. Die Aeneis war eine 'Zusammenfassung' der Römer und schrieb in einem Auszug:
	\begin{quote}
	Du bist ein Römer, dies sei dein Beruf: Die Welt regiere, denn du bist ihr Herr, Dem Frieden gib Gesinnung und Gesetze, Begnadʼge, die sich dir gehorsam fügen, Und brich in Kriegen der Rebellen Trutz. - Vergil, Aeneis (Das Zeitalter des Augustus)
	\end{quote}
	Doch auch in der Außenpolitik blieb Augustus nicht tatenlos. Die zwei wichtigsten Quellen für Getreide in Rom waren Sizilien und Ägypten, weshalb er sicherstellte, dass Ägypten Rom loyal war. Ein weiterer großer Erfolg war der Frieden mit den Parthern. Als Reaktion ließ er in großem Stil Münzen mit Augustus Abbild und den Worten \textit{Signis Receptis}, was auch der Name der Münzserie ist. Es ist deshalb von großer Bedeutung, weil der Frieden ohne Schwertstreich geschlossen worden war. \\
	Jedoch war Augustus Herrschaft nicht nur von Erfolgen gekennzeichnet. So wurde Rom um 9 n. Chr, welche eventuell im Teutoburgerwald geschlagen wurde, bedeutend geschlagen und es gab im Donauraum insgesamt sehr oft Aufstände. Man kann jedoch sagen, dass Augustus Außenpolitik von mehr Erfolgen als Fehlschlägen geprägt war. \\
	\subsection{Tugenden}
	Unter Augustus wurden auch die Tugenden sehr hervorgehoben:
	\begin{itemize}
		\item{\textit{clementia} (Milde, vor allem gegenüber Feinden)}
		\item{\textit{concordia} (Harmonie zwischen Rom und anderen Völkern)}
		\item{\textit{iustitia} (Gerechtigkeit)}
		\item{\textit{Mobilitas} (Würde, vor allem in der Öffentlichkeit)}
		\item{\textit{pietas} (Ehrgefühl gegenüber Göttern und Ahnen)}
		\item{\textit{virtus} (Tugendhaftigkeit beim Regieren und in der Gesellschaft)}
	\end{itemize}
	Diese wurden zwar verbreitet und das Volk angehalten sich an diese zu halten, jedoch hielt Augustus sich oft selbst nicht daran. Zu Augustus Herrschaftsbeginn wurden die \textbf{Proskriptionslisten} geschaffen. Jede Person, welche auf dieser Liste stand wurde als vogelfrei gekennzeichnet und dessen Mörder erhielt dessen gesamten Reichtum. Zusätzlich erzwung er eine Scheidung zwischen Livia und Tiberius Claudius Nero, damit er sie danach selbst heiraten konnte. \\
	\subsection{Augustus Nachfolge}
	Augustus Nachfolger war \textbf{Gaius Germanicus}, Sohn des Feldherrs, welcher große Erfolge in Germanien feierte. Er ist jedoch besser bekannt als \textbf{Caligula} und wird als "gesund in weder Geist noch Körper" beschrieben. So soll er sich selbst als Gottheit gesehen haben, wodurch er befahl Statuen aus Griechenland nach Rom zu bringen um deren Köpfe durch seine zu ersetzen. Zusätzlich befahl er, dass seine Statue aus massivem Gold jeden Tag gleich gekleidet werden soll wie er. Auch ächtete er oft geachtete Bürger um diese in Spielen gegen Tiere antreten zu lassen. \\
	Nach Caligula kam \textbf{Nero}, welchem eine ähnliche Geisteskrankheit nachgesagt wird. Er soll ein überzeugte Dichter und Autor gewesen sein, weshalb er befahl, dass ihn stets Personen umgaben, welche im Beifall gaben. So soll er Übermut, Wollust, Schwelgerei, Habsucht und Grausamkeit, gezeigt haben, welche nicht als die Tugenden eines Herrschers gesehen wurde. \\
	Jedoch sollte hervorgehoben werden, dass alle diese Überliefeurngen aus der Elite der Bevölkerung stammten. Vor allem Tacitus war ein angesehener Dichter der Aristokratie. Leider bestehen keine Überlieferungen aus dem Volk, welches besonders Nero wahrscheinlich sehr viel mehr wertgeschätzt hat als die Aristokratie, da dieser sehr großzügig gegenüber dem Volk war und stets Festivitäten und Spiele veranstaltete. Die Staatskasse war danach zwar leer, jedoch ging es den Leuten während der Zeit ziemlich gut. \\
	\begin{wrapfigure}{l}{0.6\textwidth}
	\includegraphics{Trajan_Reich.jpg}
	\caption{Reichsausdehnung unter Trajan}
	\end{wrapfigure}
	Ein Herrscher, welcher jedoch nicht negativ aufgefallen ist, war \textbf{Vespasian}. Er übernahm nach Nero, welcher die Staatskasse geleert hatte, und sanierte sie durch seine Reformen in kürzester Zeit. Auch war er großer Bauherr und errichtete den \textit{templum pacis}, sowie das Kolosseum, welches zu dem Zeitpunkt noch \textit{amphitheatrum flavium} genannt wurde. Da jedoch niemand als perfekt dargestellt werden konnte, wurde ihm Habsucht nachgesagt, was jedoch nicht bewiesen wurde. \\
	Ein weiterer Herrscher, welcher sich profilierte, war \textbf{Trajan}. Nach Augustus' Tod, wurde das Reich größtenteils nur verteidigt, konnte jedoch unter Trajan erweitert werden, wo auch viele Städte gegründet wurde. Trajan wurden oft homoerotische Affären nachgesagt, wodurch er als 'unmännlich' angesehen, was ein Führer nicht sein durfte. \\
	\textbf{Hadrian}, welcher nach Trajan folgte, kümmerte sich um die Reichssicherung und bildete mehrere Limes, welche das Reich verteidigen sollten. Auch er hatte anscheinend homoerotische Affären. Sein jugendlicher Liebhaber Antinoos verstarb anscheinend früh, wonach er das gesamte Reich mit Abbildern dessen flutete. \\
	\newpage
	Nach dieser Zusammenfassung über Römische Kaiser, kann man eine Liste erstellen, wie ein guter Kaiser zu handeln hat:
	\begin{itemize}
		\item{Ein guter Kaiser:}
		\begin{itemize}
			\item{Regiert mit Unterstützung des Senats}
			\item{Zeigt Milde gegenüber fremden Völkern}
			\item{Ist Ehrfürchtig gegegnüber den Göttern}
			\item{Ist tugendhafter Herrscher (Und vor allem ein mäskuliner)}
			\item{Ist treu und gerecht gegenüber Senat und Volk}
			\item{Basiert seine Handlungen auf weisen Entscheidungen}
			\item{Zeigt Selbstbeherrschung}
		\end{itemize}
		\item{Ein schlechter Kaiser hat jedoch:}
		\begin{itemize}
			\item{Hat keine Unterstützung des Senats}
			\item{Umgibt sich mit schlechten Einflüssen und unwürdigen Personen}
			\item{Verhält sich wie ein herrschsüchtiger Tyrann}
			\item{Hat keinen Respekt gegenüber Göttern, Ahnen oder Eltern}
			\item{Ist selbstverliebt und eigennützig}
			\item{Hiert nach Macht, Reichtum und Luxus}
			\item{Herrscht mit Übermut und Grausamkeit}
			\item{Missachtet den mos maiorum (altehrwürdige Sitte der Ahnen)}
		\end{itemize}
	\end{itemize}
	Anhand dieser Liste kann man in antiken Schriften akkurat einschätzen, was Dichter von römischen Kaisern hielten. \\

	\section{Römische Kaiserzeit II - 02.06.2022}

	Die zweite Einheit des römischen Kaiserreichs widmet sich primär dem römischen Heer, da es kein Rom ohne dessen Armee gibt. Zu 117 CE hatte das römische Reich seine größte Ausdehnung und es bedurf dem Heer dessen gesamte Aufmerksamkeit um die für seine Zeit extrem lange Grenze zu sichern. \\
	Rom hatte ursprünglich ein Milizheer, welches abhängig des Vermögens, in drei Klassen unterteilt war:
	\begin{itemize}
		\item{\textit{equites equo publico} (adlige Kavallerie)}
		\item{\textit{classic} (Schwerbewaffnete)}
		\item{\textit{infra classem} (Leichtbewaffnete)}
	\end{itemize}

	Um die Samnitenkrige um 300 v. Chr. wandelte sich dieses Heer von der Phalanxtaktik (Mit Kampfreihe und einer Schwerbewaffneten Reihe) zu einem Manipelheer, welches kleinere Unterteilungen hatte und beweglicher war. Zusätzlich musste Rom durch die punischen Kriege zur Seemacht werden. \\
	Mit dem 2 Jh.v.Chr. wurde dieses Milizheer dann zu einem Berufsheer

	Mit den Feldzügen gegen die Kimbern und Teutonen unter Marius wurden diese Manipel dann zu Kohorten zusammengefügt, wodurch mehr Kohäsion bestand. Gleichzeitig wurde auch das Trainingsregime angepasst und vereinheitlicht. Zusätzlich wurde die Dienstzeit pro Soldat auf 16 Jahre beschränkt, wobei auch die Bewaffnung vereinheitlicht wurde. Während diese Änderungen das Milizheer veränderte, wurde erst mit Augustus und seiner Kriegskasse das echte Berufsheer, welches normalerweise allgemein bekannt ist, gegründet. Gleichzeitig wuchs das Heer selbst stetig an: Während im 4.Jh.v.Chr. 4 Legionen bestunden, existierten zur Kaiserzeit 35 Legionen gleichzeitig. \\
	Mit Augustus wurde das Heer dann vereinheitlicht, wodurch das Heer stets aus 28 Legionen bestehen musste, wobei auch Auxiliarsoldaten aus lokalen Gebieten angeheuert wurden, da diese das Gebiet besser kannten. Innerhalb von Italien bestanden zudem 9 Prätorianerkohorten, welche je aus 500 Mann bestanden. Diese waren Elitetruppen, und später die Garde des Kaisers. \\
	Das Heer war auch ein großer Faktor in der Romanisierung des Imperiums, da es die römische Kultur im gesamten Reich verbreitete. Zum Beispiel beruht ein Großteil der Badekultur von Süd- und Mitteleuropa auf den Römern. Ebenfalls ist das Vulgärlatein, welches Soldaten sprachen, der Ursprung aller Romanischen Sprachen, wodurch auch dort der Einfluss immer noch besteht. Viele Soldaten erhielten auch Land in Provinzen, nachdem sie gedient hatten, während Auxiliaren das Bürgerrecht erhalten konnten. \\
	Die bereits angesprochene Legion bestand aus Fußsoldaten, Reitern und Hilfstruppen. Marius legte eine Legion als 10 Kohorten, 30 Manipel und 60 Centurien fest. Eine \textit{legio}, welche um die 6000 Mann stark war, hatte auch noch ein \textit{auxilium}, welches aus weiteren 5000 Mann bestand. Also bestand eine Legion aus etwa 11000 Mann und bei 35 Legionen unterhielt Rom ein riesiges Heer, welches jedoch nötig war um ein so großes Reich zu verteidigen. \\
	Ein einfacher Soldat musste am Tag etwa 30km per Fuß zurücklegen, während er 50kg Ausrüstung mit sich trug. Alles was ein Soldat für sich selbst brauchte musste ein Soldat selbst tragen. So kam zusätzlich zur Ausrüstung auch noch Flick- und Putzzeug hinzu. Diese konstante Belastung war extrem anstrengend, was auch an Knochenfunden von Zenturien ersichtlich ist, da diese oft sehr mitgenommene Knie haben. Jedoch musste ein Soldat nicht nur seine Ausrüstung tragen, sondern auch an täglichen Trainingseinheiten teilnehmen. Jeden Tag musste auch ein Lager aufgebaut werden, wobei das gesichert, aufgebaut und unterhalten werden muss. Bevor man sich schlafen legte bekam man noch etwas zu essen, musste am nächsten Tag in aller Frühe jedoch wieder mit dem Training beginnen, bevor es wieder Verpflegung gab. \\
	Die Ausrüstung eines Soldaten war auch seit Marius normiert und standardisiert, was die Wartung und Ersatz vereinfachte. Die \textit{Caligae}, welche römische Sandalen waren, sollen mit dünnen Socken ziemlich bequem gewesen sein. Das \textit{Scutum}, ein römisches Viereckschild, hatte ein Erhebung in der Mitte um Schläge abwzuwehren. Jeder Soldat trug auch ein \textit{Pteryges}, welches ein mit Metall beschlagenes Lederstück war. Interessanterweise gab es dieses bei den Etruskern nicht und bietet auch keinen ersichtlichen Zweck, außer, dass die Tunika nicht zu sehr flattert. \\
	Dies war jedoch nur die Ausrüstung eines gemeinen Soldaten. Jede Armee hatte zusätzlich spezielle Ausrüstung, welche durch spezielle Bannerträger, den \textit{Signifern} getragen wurde und oft den Kaiser oder Götter zeigte. Diese waren der Schutz der Armee und dessen Verlust bedeutete auch den Verlust des göttlichen Schutzes. \\
	Ein Heer hatte auch eine eigene Militärkapelle, welche den Takt des Heers angeben sollte, jedoch auch zur Motivation diente. Bei Siegeszügen spielten diese auch. \\
	Das moderne Bild des gestählten Legionärs ist etwas irreführend, da die meisten Soldaten sogenannten \textit{Puls} aßen, was ein fettiger Getreidebrei war. Dadurch hatten die meisten Soldaten zusätzlich zur Muskelmasse eine gesunde Fettschicht, welche als Barriere diente. \\
	\begin{wrapfigure}{l}{0.5\textwidth}
	\includegraphics{Militärlager.jpg}
	\caption{Ein klassisches permanentes Militärlager}
	\end{wrapfigure}
	Das Lager einer Armee war sehr normiert und änderte sich mit der Zeit nur sehr wenig. In Britannien kann man heute noch Überreste von Legionslagern finden und Aufschluss darüber geben, wie lange sie diese verwendeten. Auch Israel hat wegen der geographischen Begebenheiten auch noch gut erhaltene Überreste. Im römischen Reich gab es einige größere Militärlager, welche oft entlang der Grenze, aber auch an großen Knotenpunkten lagen. \\
	Jede Legion hatte auch seine eigenen Namen und Wappen, wodurch diese ihre eigene Identität hatten. Die zweite Legion \textit{Augusta} bestand von 43 v.Chr. bis 260 n.Chr., als es in die legio Brittannica II aufging. \\
	Jede Legion hatte einen Oberkommandanten, welcher oft bekannte Persönlichkeiten waren. Jeder Kommandant hatte 11 Offiziere, welche das höchste Amt waren, welches ein Ritter haben konnte. Zusätzlich hierzu konnte ein Ritter auch Präfekt von Ägypten werden, was die größtmögliche Ehre war, und Präfekt der Prätorianer in Italien. Jede Legion hatte eine extrem starre Hierarchie, welche niemals unterwandert werden durfte. Wenn man die Befehlskette verweigerte wurde dies oft mit dem Tod bestraft. Die Starrheit führte jedoch auch dazu, dass eine Legion sehr effizient war, da die Befehlskette stets gegeben war. \\
	Der Großteil einer Legion bestand aus Fußsoldaten, wobei es auch noch etwa 120 berittene Truppen gab. Dabei waren auch noch die 5000 Auxiliartruppen. \\
	Die Standardaufstellung einer Legion sah vor, dass die Fußsoldaten in der Mitte als Centurien marschierten. Die Vorhut bildeten die berittenen Truppen, gefolgt von den Auxiliareinheiten, welche ebenfalls berittene Truppen hatte. \\
	In der Bergfestung Masada, welche in Israel liegt und von Herodes begründet wurde, kann man heute noch die Belagerungsrampe der Römer sehen, mit welchen sie die Festung einnahmen. Die von den Römern benutzen Belagerungswaffen wurden jedoch nicht von ihnen erfunden, sondern waren bereits zu der Zeit der Griechen bekannt. \\
	Die wichtigsten Belagerungswaffen der Römer waren Ballisten, welche große Armbrüste sind, die durch zwei geschulte Männer transportiert und bedient werden konnte. Diese Ballisten hatten eine enorme Durchschlagskraft und konnten leicht Kettenhemden durchstoßen. Zusätzlich verwendeten sie auch Trebuchets und Katapulte um Städte zu belagern. Bei der Belagerung von Jotabata um 67n.Chr. beschreibt Flavius Josephus bildhaft die destruktive Kraft dieser beiden Maschinen. \\
	Das römische Heer musste auch nahezu konstant gegen Piraten kämpfen, welche Handelsschiffe plündern wollten. Rom musste Grundnahrungsmittel wie Getreide, Öl und andere Sachen konstant importieren und hatte seine Hände voll zu tun um diese abzuwehren. Ein weit verbeiteter Mythos ist, dass Sklaven zum Rudern von Schiffen verwendet wurden. Rudern ist eine aufwändige, koordinierte Aufgabe, welche ausgebildete Personen benötigte. Es mussten zwar keine Bürger sein, sondern konnte auch von Peregrinern besetzt werden, jedoch waren es nahezu nie Sklaven, welche nur in äußersten Notfällen zugezogen wurde. \\
	Oft wurden Schiffe jedoch nicht geentert, sondern nur gerammt, da diese oft kleiner waren und man so nicht das gefährliche Entern versuchen musste.

	\section{Randvölker - 09.06.2022}

	Der Begriff \textit{Randvolk} mag etwas negativ klingen, jedoch ist mit diesem Begriff in keiner Weise eine negative Behaftung gemeint, sondern lediglich Völker, welche an den Rändern des römischen Reiches lebten.
	\subsection{Germanen}
	Das wichtigste Randvolk sind die \textbf{Germanen}, welche in den Germanischen Tieflanden (Heutiges Mittel- und Norddeutschland) lebten. Sie waren umgeben von den \textit{Veneden} im Osten und den \textit{Kelten} im Süden und Western. Die ersten Auseinandersetzungen zwischen den Römern und Germanen kam nach einer Völkerwanderung der \textit{Ambronen, Teutonen und Kimbern} aus dem Jütland im 2.Jh.v.Chr. im heutigen Dänemark und Schweden. Sie waren zu diesem gezwungen, da ihre ursprünglichen Siedlungsplätze überflutet worden waren. Während sie zuerst die Römer um ein Siedlungsgebiet baten, nahmen sie es sich mit Gewalt nachdem sie abgewiesen worden waren. Dabei waren die Germanen relativ erfolgreich und sie konnten einige Siege erringen. Bei der \textbf{Schlacht von Noreia} wird das römische Heer beispielsweise empfindlich geschlagen. Jedoch konnten die Römer diese Angriffe eventuell zurückschlagen. Es wird angenommen, dass die Römer mit den Kampftaktiken der Germanen nicht vertraut waren und erst mit diesen vertraut werden mussten um ihnen etwas entgegenzusetzen zu können. \\
	Eine weitere Auseinandersetzung mit den Germanen hatten die Römer mit den \textit{Usipeten und Tenkterern} um 55 v.Chr.. Diese wollten sich ebenfalls im gallischen Gebiet ansiedeln und verhandelten laut Plutarch einen Friedensvertrag aus, welcher jedoch von den Germanen gebrochen wurde, weshalb Cäsar etwa 400.000 Usipeten und Tenkterern tötete. Andere Quellen besagen jedoch, dass Cäsar diese ohne Grund angegriffen hat, wodurch er den Friedensvertrag brach. \\
	Um 9 n.Chr. geschah die \textbf{Varusschlacht} bei Kalkriese in den heutigen Niederlanden. \textbf{Arminius}, welcher eine römische Geisel war und dadurch romanisiert werden hätte sollen, war Fürst der Cherusker und führte den Statthalter Germaniens \textit{Quintilius Varus} hinters Licht indem er ihm vortäuschte, dass es nur kleine Gruppen aus Wegelagerern sind, welche ausgemerzt werden müssen. So dringen 3 Legionen, nahezu 20.000 Mann, tief in fremdes Gebiet vor und werden in der Folge komplett ausgelöscht. Dies war eine der größten Niederlagen, welche das römische Reich jemals erfuhr und führte zu einem Rückzug aus den eroberten Gebieten in Germanien hinter den Rhein. \\
	Im 19. Jh. wurde Arminius im Zuge einer deutschnationalistischen Kampagne zu einem Volksheld stilisiert, welcher alle Germanen von den Römern befreien wollte. Dies ist jedoch nur eine Legende, da zu diesem Zeitpunkt kein geeintes Germanien bestand und die meisten Stämme miteinander verfeindet waren. \\
	Als Folge dieses Konflikts haben Rom und die Germanenvölker ständige Auseinandersetzungen, welche sich bis in das 4. Jh.n.Chr. fortführt. Julian, genannt der Apostat, konnte 350 n.Chr. zwar einige Erfolge feiern, der Konflikt wurde jedoch nicht beendet. \\
	Einige Quellen sprechen über die Zeit über die Germanen, welche den Großteil des historischen Wissens über die Germanen erfasst haben. Cäsar mit \textit{Bellum Gallicum} um das 1. Jh.v.Chr. und Tacitus mit \textit{Germania} um 100 n.Chr.. \\
	Andere Quellen sprechen zwar auch über die Germanen, beschäftigten sich jedoch nur am Rande mit ihnen. So muss man auf Cäsar und Tacitus zurückgreifen, was kein sehr objektives Bild der Germanen sein kann. Tacitus beispielsweise bedient sich im großen Stil dem \textit{interpretatio romana}, welches fremdes Wissen, in diesem Fall Germanisch, im römischen Weg  fehlinterpretiert um es dem römischen Volk näherzubringen. So nimmt er die Götter der Germanen, welche heute leider größtenteils unbekannt sind, da Germanen eine mündliche Kulturüberlieferung hatten, und gibt ihnen römische Äquivalente und generalisiert alle germanischen Völker als gleichaussehende nur zum Kämpfen gute, Babaren. Tacitus erwähnt auch speziell, dass die Germanen Frauen in Entscheidungen miteinbezog. So schreibt er 
	\begin{quote}
	"Da sie sogar glauben, daß (den Frauen) etwas Heiliges und Seherisches innewohnt, verwerfen sie weder ihre Ratschläge noch mißachten sie ihre Antworten" - Tacitus, Germania (100 n.Chr.)
	\end{quote}
	Dieser Umstand ist gut überliefert, da Römer es unverständlich fanden, dass Frauen im Krieg teilnahmen oder Entscheidungsträger waren und wurde oft erwähnt. \\
	Tacitus beschreibt auch die Faulheit der Germanen. Dass es verrufen war, für etwas zu arbeiten, wenn man es sich mit Gewalt nehmen kann sowie hohe Strafen für kleine Verbrechen.
	Jedoch lobt er auch oft Aspekte des Germanischen Lebensstils, was jedoch nur dazu dient um zu zeigen, dass, obwohl die Germanen ja Babaren sind, diese trotzdem Sittlichkeit und Ehre besitzen und keinen Ehebruch begehen. So will er aussagen, dass das römische Volk wieder zu alten Sitten zurückkehren soll. Dass es nicht dient um ein akkurates Bild von Germanien zu zeigen beweist der Umstand, dass er von Germanischen Liebesbriefen schreibt, obwohl es keine Beweise gibt, dass die Germanen zu diesem Zeitpunkt eine Schriftsprache hatten oder Liebesbriefe schrieben. \\
	So muss man römische Quellen anderer Völker immer mit einem gewissen Maß an Argwohn betrachten. Dass die Germanen nicht pelztragende Babaren waren beweisen auch Funde in Gräbern Germanischer Herrscher. Oft wurden teuer importierte Silber- und Goldwaren aus dem römischen Reich aber auch aus Fernost als Grabbeigaben gelegt, was beweist, dass solche Gegenstände durchaus geschätzt wurden. \\
	Germanen lebten oft in sehr lose gebauten Weilern, welche Wohn- und Wirtschaftsgebäude, welche andeinander gebaut worden sind. Nur wenn aktiver Platzmangel bestand wurde dichter gebaut wofür es einige überlieferte Beispiele gibt. \\
	Es werden zwei Gründe vermutet, weshalb die Römer so an Germanien interessiert waren:
	\begin{itemize}
		\item{Es gab stets den Wunsch von Römern Personen in den Bund einzubringen und zu romanisieren.}
		\item{Germanien war reich an Gütern welche aus Germanien importier werden musste. Einige der wichtisten Waren waren: \textit{Pelze, Bernstein, Honig und Wachs. Ein weiteres wichtiges Gut waren germanische Haare von Frauen. Diese Haare waren einerseits beliebt als Perücken, da blonde Haare in Rom recht selten waren. Ein weiterer Nutzen war auch die Verwendung für Sehnen für Ballisten, da dieses eingeölt anscheinend länger hielt als römisches Haar.}
	\end{itemize}
	Denn obwohl Rom lange mit den Germanen im Krieg war, erlag der Handel meistens nicht, denn durch Handel profitierten beide Parteien.
	\subsection{Die Reitervölker}

	Die Reitervölker zeichnen sich aus durch ihr Nomadentum, also dass sie keinen festen Wohnsitz haben. Als Grundlage für diese Lebensweise dienten gehaltene Tiere wie Pferde zur Fortbewegung und Vieh wie Ziegen und Schafe, da diese eine große Ausdauer hatten. Sie lebten in sogenannten Jurten, welche leicht auf- und abbaubare Zelte waren. Speziell an den Reitervölkern ist, dass es keine ethische Gruppierung ist, sondern eine Interessensgemeinschaft und das Zugehörigkeitsgefühl aus der Gruppe stammt. Anführer wurden nicht gewählt oder geerbt sondern oft war der stärkste Kriege gleichzeitig Anführer. Auch hielten sie selten Sklaven, außer wenn es um Kriegsbeute ging. \\
	Oft wurden nur Pfeil und Bogen als Hauptwaffe verwendet, welche reitend abgeschossen wurden. \\
	Die Religion basiert auf Naturphänomenen, welchen göttlicher Charakter zugeschrieben wird. Die Verbindung zwischen diesen Göttern und den Menschen wird durch Schamanen hergestellt. \\
	Die 4 wichtigsten Reitervölker waren:
	\begin{enumerate}
		\item{Kimmerier (8. Jh.v.Chr.): Das erste große Reitervolk, welches ursprünglich aus Südrussland stammte.}
		\item{Skythen (8./7. Jh.v.Chr.): Stammte ebenfalls aus Südrussland und war oft im Konflikt mit den Kimmeriern. Handelte 100 v.Chr. mit den Griechen. Tomyris ist wahrscheinlich nur eine Legende.}
		\item{Sarmaten (3. Jh.v.Chr.): Folgte den Skythen und diente als Söldner für die Griechen. Unterstützte auch die Daker und Vandalen gegen Rom}
		\item{Hunnen (3. Jh.v.Chr.): Existierte bereits in der vorchristlichen Zeit traf jedoch erst 400 n.Chr. in Europa als Einzelstämme ein. Erst mit Attila wurden die Hunnenreiche vereint, zerfällt jedoch kurz nach Attilas Tod erneut.}
	\end{enumerate}

	\section{Die Spätantike - 23.06.2022}
	
	Die Spätantike kann grob in 7 Teile geteilt werden:
	\begin{itemize}
		\item{Die Zeit der Soldatenkaiser (Nicht im Skript)}
		\item{Diokletian und die Tetrarchie (inklusive Reformen)}
		\item{Konstantin der Große}
		\item{Teilung des Römischen Reiches um 395}
		\item{Völkerwanderung (Vandalen etc.)}
		\item{Kaiser Iustinian (Kaiser des östlichen Reiches)}
		\item{Ausbreitung des Christentums}
	\end{itemize}

	Nach der großen Kaiserzeit des römischen Reiches, mit den Adoptivkaisern von Trajan, Hadrian und Mark Aurel, folgten schwächere Kaiser, welche auch nicht lange überlebten. In 193 folgt das zweite Vierkaiserjahr, wonach sich jedoch die severische Dynastie herausformt, welche sich bis 235 hält. \\
	Während dieser Severischen Zeit erfährt das römische Reich außerordentlichen Druck aus vielen Richtungen. Im Norden brechen germanische Stämme immer wieder in das Reich ein, während aus dem Osten das neugeformte Sassanidenreich, welches der Nachfolger des Partherreichs war und nun expandieren wollte. \\
	Nach 235 endete diese Dynastie, und die Zeit der Soldatenkaiser bricht an. In den 50 Jahren zwischen 235 und 284/5 sind 50 Kaiser, welche herrschten, überliefert. Manchmal gab es sogar mehrere gleichzeitig, welche gegenseitig in Rivalität standen. Hier starben auch selten Kaiser durch natürlichen Todes sondern nahezu stets durch gewaltsame. Zu dieser Zeit kann man auch nicht mehr von einer Dynastie reden, da es größtenteils Kaiser waren, welche durch das Militär gestützt wurden und sich so ihre Macht sicherten. \\
	\subsection{Diokletian und die Tetrarchie}
	Nach dem Ende dieser Zeit um 285 spricht man oft von der zweiten Spätantike, welche mit der Herrschaft von Diokletian beginnt. Er begründete die Herrschaftsform der \textbf{Tetrarchie}, also der Vierherrschaft. Diokletian war, gleich wie 50 seiner Vorgänger, ein Soldat, welcher sich hochgearbeitet hatte und in der Hierarchie aufstieg. Die Meinung des Volkes oder sonstiges existierte zu dieser Zeit nicht mehr wirklich und Soldaten riefen den Kaiser aus. Diokletian herrschte, anders wie seine Kompatrioten, für 21 Jahre und begründete in dieser Zeit umfassende Reformen. Er festigte seine Macht grundlegend indem er über seinen größten Rivalen, den Bruder des alten Kaisers, in einer Schlacht siegte. \\
	Die letzten 50 Jahre zeigten eine innerpolitische Schwäche des römischen Systems, welche Diokletian bestimmt erkannte, da er mit den Reformen diese größtenteils änderte. \\
	Diokletian ist in dieser Hinsicht eine Erwähnung wert, da er sich nicht nur miltärisch durchsetzte, sondern auch politisch. Gleich wie Augustus initiierte er Reformen, welche Zeit zur Umsetzung und Etablierung brauchten. Augustus hatte das \textbf{Prinzipat} eingeführt, welches von Diokletian durch das \textbf{Dominat} ersetz wurde. Das Prinzipat sah zwar den Kaiser im Mittelpunkt, wurde jedoch gleichzeitig vom Senat gestützt. Im Dominat hingegen wird dem Kaiser umfassende Macht gewährt, wodurch er im Grunde zum Alleinherrscher wird. \\
	Diokletian begründete auch die \textbf{Tetrarchie}, was ein Abgang von der früheren Zentralherrschaft in Rom ist und stattdessen dezentral organisiert ist, indem sie auf 4 Herrscher aufgeteilt wird. Diese waren natürlich in enger Absprache. Dieses System wuchs jedoch nur langsam von selbst. Ursprung dessen war die Ernennung Maximians als Zweitherrscher durch Diokletian, womit dieser die Innenpolitik festigen wollte. Diese Idee selbst war nicht unbedingt neu, da Mark Aurel um 200 n.Chr. Lucius Verus zum Zweitherrscher ernannte, damit dieser gegen das Partherreich vorgehen konnte, wobei Mark Aurel sich auf die Germanen konzentrieren konnte. \\
	6 Jahre nach dieser Änderung werden jedoch zwei weitere Kaiser ernannt, welche jedoch den ersten beiden Kaisern unterstellt sind. Diese sind jedoch designierte Nachfolger dessen, falls diese abtreten sollten. Hintergrund dieser Änderung ist Diokletians Überlegung, dass Nachfolgekriege das Reich stets sehr schwächten, wodurch er die Nachfolge direkt regelte. Diese Konstellation und direkte Klärung der Nachfolge sollte dem entgegenwirken. Die Nachfolge war jedoch nicht dynastisch, sondern fokussierte sich auf Interessensgemeinschaft. Jeder der beiden \textit{Caesari}, welche die beiden Nachfolgekaiser waren, waren auch noch einem \textit{Augustus} zugeteilt, sodass ein Caesarius stets auf einen Augustus folgte. Jedoch gab es auch unter den Augusti eine Hierarchie wobei Diokletian natürlich über Maximian stand. Diokletian verband sich selbst hierbei mit Jupiter, während Maximian mit Herkules nur ein Halbgott war. \\
	Gleichzeitig war jeder der vier Herrscher für einen Bereich des Reiches verantwortlich:
	\begin{itemize}
		\item{Diokletian: Der Osten (Hauptsitz: Nikomedien bei Istanbul)}
		\item{Maximian: Spanien, Italien und Afrika (Hauptsitz: Mailand)}
		\item{Constantius: Gallien und Britannien (Hauptsitz: Trier und York)}
		\item{Galerius: Illyricum, Makedonien und Griechenland (Hauptsitz: Sirium)}
	\end{itemize}

	Der Plan war, dass nach 20 Jahren die Augusti ihre Macht abgeben, wodurch die Caesari nachfolgen und hierbei neue Caesari ernennen, wodurch eine neue Viererkonstellation geschaffen wird. Hier war wieder weniger die familiäre Bindung wichtig und Caesari wurden anhand ihrer Fähigkeit ernannt. Diese Eigenschaft führte auch im Endeffekt zum Niedergang der Tetrarchie. \\
	Fürs Erste führt die Tetrarchie jedoch zu einer deutlichen Stabilisierung des Reiches, da die vier Herrscher sich auf ihr eigenes Gebiet konzentrieren konnten. \\
	In Venedig gibt es eine in einer Ecke eingelassene Statue, welche vier Männer zeigt, welche sich jeweils in Zweiergruppen gegenseitig die Schulter halten. Es ist nicht bekannt welche Tetrarchengruppe es darstellen soll, jedoch wird angenommen, dass durch die Existenz der Herrschaftsinsignien, diese vier Herrscher sind. \\
	Mit der Einführung der Tetrarchie wird auch das administrative System verändert. Diese sind außerordentlich umfangreich. Hierbei muss man auf das ursprüngliche System der römischen Republik zurückblicken, in welchem Gebiete außerhalb Italiens als Provinzen mit einem Statthalter etabliert wurden. Dieses System wurde erst mit Diokletian verändert. Es gab zu jedem Zeitpunkt die Möglichkeit die Größe von Provinzen zu verändern, wovon Diokletian umfangreichen Gebrauch macht und die Größe einzelner Provinzen deutlich verkleinert, dabei jedoch die Anzahl selbst verdoppelt von 50 auf 101. Gleichzeitig wurde auf die Rechtsform der Provinz, welche zuvor die größte Verwaltungsform war, weitere Ebenen hinzugefügt und diese 101 Provinzen in 12 größere Verwaltungsbezirke gegliedert, welche man Diözesen nannte. Über diese Diözesen wurden wieder 4 weitere Präfekturen gestellt, welche stets einem der Tetrarchen unterstand. Der Gedanke hinter dieser Änderung war eine höhere Effizienz des Verwaltungsbereiches. \\
	Die administrative Neugliederung ist bedeutend ähnlicher an modernen Aufteilungen von Ländern, anstatt einer zentralen Machtposition und dessen unterstellter Provinzen. \\
	Zu diesen Verwaltungsformen wurden auch Verwalter eingesetzt. Hierbei wurde militärische und zivile Gewalt mehr getrennt als zuvor. So gab es einen militärischen und administrativen Verwalter. Der militärische Verwalter war der \textit{dux} (Vgl. Duke/Duchy). Während auf ziviler Ebene die Provnzstatthalter innerhalb einer Provinz weiter die Macht besaßen, wurden diese den \textit{vicarii} unterstellt, welche die Kontrolle über eine Diözese erhielten, welche wiederum von Prätorianerpräfekten, welche schon aus der Kaiserzeit bekannt waren, verwaltet wurden. \\
	Während man im Nachhinein über Diokletians Änderungen spricht, ist jedoch nicht klar, ob er alle diese Änderungen selbst umsetzte oder nur eine Idee besaß und diese sein Nachfolger erst umsetzte. \\
	Der Plan, dass die Augusti nach 20 Jahren abtreten, geschah in seiner ersten Iteration reibungslos und Diokletian gab die Macht 20 Jahre nach seiner Machtübernahme ab. Er setzte sich in Split, Kroatien zur Ruhe im sogenannten \textit{Diokletianspalast}, welcher heute in seinen Grundmauern noch besteht. Dieser Ort war strategisch gewählt da er somit, trotz seines Ruhestandes, nahe am Geschehen war und dieses Beobachten konnte. \\
	\subsection{Konstantin 'der Große'}
	Konstantin der Große, Sohn des Constantius, welcher einer der ersten Caesari war kam nicht auf traditionellem Wege zum Herrscherthron. Nach Diokletians und Maximians Abtritt, werden Severus und Maximinus Daia zu Caesaren ernannt. Constantius, welcher erst 305 zum Augustus geworden war, starb nur ein Jahr später um 306. Somit wird Severus nicht nach 20, sondern nach einem Jahr, zum Augustus. Jedoch wird Constantius Sohn von den Truppen des Constantius statt Severus zum neuen Augustus ausgerufen und stand somit in direktem Machtkampf mit ihm. \\
	Um die Situation zu entschärfen, ernennt Galerius als Augustus Severus zum Nachfolger, ernennt dabei jedoch auch Konstantin zum neuen Caesarius um das Heer zu beschwichtigen. Da nun ein Sohn eines Augustus Machtansprüche erhielt, erhob auch Maxentius, Sohn des Augustus Maximian, Anspruch auf die Herrschaft, wobei er jedoch unberücksichtigt bleibt. Als Reaktion darauf unterwirft Maxentius zusammen mit seinem Vater Italien und beherrscht Rom. \\
	Nachdem Severus 308 auch stirbt und weitere Unruhen befürchtet werden, wird die \textbf{Kaiserkonferenz} in Carnuntum gestellt, wobei Diokletian als Schiedsrichter fungierte. Es kommt zwar zu einem Ergebnis, jedoch agiert Maxentius in Italien immer noch als Ursupator. Hierbei wird Maxentius jedoch bei der Schlacht an der Milvinischen Brücke von Konstantin geschlagen. Über die nächsten Jahre vereint Konstantin das gesamte Reich wiederum unter sich und verteilt stattdessen seine Kinder als Verwalter ein. Somit kam es wieder zu einer dynastischen Herrschaft in Rom, wobei jedoch durch diese Erfahrung klar wurde, dass man nicht als alleiniger Herrscher über Rom herrschen konnte. Konstantin begründete auch \textbf{Konstantinopel}, welches von Anfang an als zweiter Herrschaftssitz gedacht war. Während die alten Machtstrukturen in Rom noch bestanden, erschuf er auch einen neuen Senat, wiederum in Konstantinopel, welcher mit dem in Rom gleichgestellt war. Man kann annehmen, dass Konstantin von Anfang an vor hatte Konstantinopel als neuen Hauptmachtssitz zu etablieren, da die Statdplanung im Sinne einer Großstadt und Metropole geschah.


























	
\end{document}