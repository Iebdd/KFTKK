\documentclass{article}

\usepackage{geometry}
\usepackage{makecell}
\usepackage{array}
\usepackage{multicol}
\usepackage{setspace}
\usepackage{changepage}
\usepackage{booktabs}
\usepackage{graphicx}
\usepackage{float}
\newcolumntype{?}{!{\vrule width 1pt}}
\renewcommand\theadalign{tl}
\setstretch{1.10}
\setlength{\parindent}{0pt}

\geometry{top=12mm, left=1cm, right=2cm}
\title{\vspace{-3cm}21S 520.230 Deutsch: Mutter-/Bildungssprache: Textanalyse und Textproduktion Gruppe 3}
\author{Andreas Hofer}

\begin{document}
	\section*{\textbf{VI. Präpositionen}}
	\subsection*{\textbf{70} | Bitte setzen Sie die Präpositionen "dank", "infolge" oder "zufolge" ein und ergänzen Sie gegebenfalls die Endungen! Beachten Sie, dass mit "dank" nichts Negatives verbunden werden sollte, dass im Gegenteil diese Präpositione das Positive besonders betont!}
	\begin{enumerate}
		\item{Ein\textbf{em} Gerücht \textbf{zufolge} wird der Minister zurücktreten}
		\item{\textbf{Dank} sein\textbf{es} außerordentlichen Fleiß\textbf{es} schloss er sein Studium in kürzester Zeit ab.}
		\item{\textbf{Infolge} ein\textbf{es} schwer\textbf{en} Herzleiden\textbf{s} konnte er seinen Beruf nicht mehr ausüben.}
		\item{\textbf{Dank} staatlich\textbf{er} Finanzierungshilfen konnte das kostspielige Projekt durchgeführt werden.}
		\item{\textbf{Infolge} d\textbf{er} ständig\textbf{en} Klimawechsel litt seine Gesundheit.}
		\item{\textbf{Dank} ihr\textbf{es} persönlich\textbf{en} Engagement\textbf{s} gelang es den Studenten, die Probleme zu lösen.}
		\item{\textbf{Infolge} d\textbf{er} Entfremdung des Menschen von seiner Arbeit entwickeln sich häufig Aggressionen.}
		\item{Unbestätigt\textbf{en} Meldungen \textbf{zufolge} ist bei dem Reaktorstörfall Radioaktivität ausgetreten.}
	\end{enumerate}
	\subsection*{\textbf{71} | Bitte setzen Die die folgenden Präpositionen ein und ergänzen Sie fehlende Endungen!}
	\begin{enumerate}
		\item{\textbf{Infolge} neuer Zeugenaussagen konnte der Vorfall schließlich geklärt werden.}
		\item{Ich erwähne die Einzelheiten nur der Vollständigkeit \textbf{halber}}
		\item{\textbf{Hinsichtlich} d\textbf{es} Erfolg\textbf{s} bin ich skeptisch.}
		\item{\textbf{Um} unser\textbf{er} Freundschaft \textbf{willen} werde ich dich unterstützen.}
		\item{Wir haben den Entwurf \textbf{entsprechend} ihr\textbf{en} Änderungswünschen umgeschrieben}
		\item{\textbf{Wider} Erwarten gewann er den Wettkampf.}
		\item{\textbf{Mit Hilfe} d\textbf{er} neu\textbf{en} Kommunikationstechnik gelang es der Firma, ein weltweites Netzwerk aufzubauen.}
		\item{\textbf{Anlässlich} d\textbf{es} 250. Geburtstag\textbf{s} von Mozart finden weltweit Feierlichkeiten statt.}
		\item{\textbf{Aufgrund} d\textbf{es} frühen Wintereinbruch\textbf{s} kam es zu Versorgungsschwierigkeiten.}
		\item{\textbf{Innerhalb} Europa\textbf{s} werden die Grenzen abgebaut.}
		\item{Der Mann befand sich \textbf{inmitten} ein\textbf{er} erregten Menge, die ihn zu bedrohen schien.}
		\item{Selbst in Deutschland leben zahlreiche Menschen \textbf{unterhalb} d\textbf{er} Armutsgrenze.}
	\end{enumerate}
	\subsection*{\textbf{72} | Bitte setzen Sie die folgenden Präpositionen ein und ergänzen Sie fehlende Endungen!}
	\begin{enumerate}
		\item{\textbf{Infolge} d\textbf{er} Konjunkturabschwächung hatte die Firma hohe Verluste zu verzeichnen.}
		\item{\textbf{Innerhalb} d\textbf{es} Sperrbezirk\textbf{s} wurde erhöhte Radioaktivität gemessen.}
		\item{Der Peso gilt in Mexiko, aber nicht \textbf{jenseits} d\textbf{er} mexikanischen Grenzen.}
		\item{\textbf{Jenseits} d\textbf{es} Gebirge\textbf{s} lag fruchtbares Ackerland.}
		\item{\textbf{Anlässlich des} 60. Geburtstag\textbf{s} von Professor Funk wurde ihm eine Festschrift gewidmet.}
		\item{\textbf{Ungeachtet} d\textbf{er} Warnungen startete die unzureichend vorbereitete Expedition.}
		\item{\textbf{Infolge} stark\textbf{en} Schneefall\textbf{s} kam es zu einem Verkehrschaos}
	\end{enumerate}
	\subsection*{\textbf{73} | aus oder vor?}
	\begin{enumerate}
		\item{\textbf{Aus} Anlass des 50. Jubiläums wird eine Festschrift veröffentlicht.}
		\item{Wir wissen \textbf{aus} Erfahrung, dass Auswendiglernen wenig nützt.}
		\item{Er wusste \textbf{vor} lauter Arbeit nicht, wo ihm der Kopf stand.}
		\item{\textbf{Aus} den folgenden Gründen muss ich das Angebot ablehnen: Erstens bin ich überlastet. Zweitens habe ich auf diesem Gebiet wenig Erfahrung. Drittens scheint mir die finanzielle Seite nicht attraktiv.}
		\item{Er log \textbf{aus} Furcht vor Strafe.}
		\item{Er handelte \textbf{aus} Überzeugung, Geld spielte dabei keine Rolle.}
		\item{\textbf{Aus} verschiedenen Gründen trat er von dem Vorhaben zurück.}
	\end{enumerate}
	\subsection*{\textbf{74} | Bitte setzen Sie die richtige Präpositionenkombination ein!}
	\begin{enumerate}
		\item{\textbf{Von} heute \textbf{an} wird gearbeitet.}
		\item{Die Verwüstungen gingen \textbf{über} alles Vorstellbare \textbf{hinaus}.}
		\item{\textbf{Von} Geburt \textbf{an} kränkelte er.}
		\item{\textbf{Von} dem Fernsehturm \textbf{aus} hat man bei gutem Wetter einen weiten Blick über die gesamte Region.}
		\item{Der Saal war \textbf{bis auf} den letzten Platz gefüllt}
		\item{Im Rift Valley in Afrika steigen die Temperaturen \textbf{oft über} 50 Grad.}
		\item{All Stimmberechtigten \textbf{bis auf} einen stimmten für den Entwurf.}
		\item{\textbf{Bis auf} einen kleinen Kreis von Eingeweihten wusste niemand von dem Plan.}
		\item{\textbf{Um} die Stadt \textbf{herum} führte eine Stadtmauer}
		\item{\textbf{Auf} den Protest des Verteidigers \textbf{hin} wurde die Verhandlung unterbrochen}
		\item{Der halb Verhungerte aß alles \textbf{bis auf} den letzten Rest auf.}
		\item{\textbf{Auf} deinen Rat \textbf{hin} habe ich den ganzen Text noch einmal gründlich durchgelesen.}
		\item{\textbf{Von} seiner frühesten Jugend \textbf{an} war es sein Traum, den Weltraum zu erkunden.}
	\end{enumerate}
	
















\end{document}