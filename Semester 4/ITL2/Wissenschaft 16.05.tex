\documentclass{article}

\usepackage{geometry}
\usepackage{makecell}
\usepackage{array}
\usepackage{multicol}
\usepackage{setspace}
\usepackage{changepage}
\usepackage{booktabs}
\usepackage{graphicx}
\usepackage{float}
\newcolumntype{?}{!{\vrule width 1pt}}
\renewcommand\theadalign{tl}
\setstretch{1.10}
\setlength{\parindent}{0pt}

\geometry{top=12mm, left=1cm, right=2cm}
\title{\vspace{-3cm}22S 520.234 Deutsch: Intralinguale Textarbeit II}
\author{Andreas Hofer}

\begin{document}
	\section{Fremdwörter - deutschsprachige Wörter - Fachbegriffe}
	\subsection*{\textbf{a}  Welche Verben sind äquivalent? Ordnen Sie den Fremdwörtern ihre deutschsprachigen Entsprechungen zu.}
	\begin{itemize}
		\item[\textbf{1 - g}]{antizipieren - vorwegnehmen}
		\item[\textbf{2 - b}]{diskutieren - besprechen}
		\item[\textbf{3 - d}]{ich positionieren zu - seine Meinung zu etwas äußern}
		\item[\textbf{4 - a}]{sich positionierne in - sich verorten in}
		\item[\textbf{5 - e}]{determiniert sein durch - bestimmt werden durch}
		\item[\textbf{6 - c}]{evaluieren - beurteilen}
		\item[\textbf{7 - f}]{eruieren - herausfinden}
	\end{itemize}
	\subsection*{\textbf{b}  Ergänzen Sie zunächst den Artikel der Nomen. Bilden Sie anschließend Wortpaare. Welche Wörter sind äquivalent?}
	\begin{itemize}
		\item[\textbf{1 - f}]{das Fazit - die Besprechung}
		\item[\textbf{2 - b}]{der Aspekt - der Gesichtspunkt}
		\item[\textbf{3 - a}]{die Deskription - die Beschreibung}
		\item[\textbf{4 - c}]{die Position - der Standpunkt}
		\item[\textbf{5 - e}]{die Option - die Möglichkeit}
		\item[\textbf{6 - g}]{die Rezension - das Ergebnis}
		\item[\textbf{7 - d}]{die Situation - die Lage}
	\end{itemize}
	\subsection*{\textbf{c}  Vergleichen Sie die Bedeutung der Wörter \textit{Perspektive} und \textit{Basis} in den folgenden Beispielsätzen. Wählen Sie für jeden Satz das passende deutschsprachige Äquivalent aus.}
	\begin{itemize}
		\item[\textbf{1}]{Aus diesem \textit{Blickwinkel} ist keine Lösung zu erkennen.}
		\item[\textbf{2}]{Daraus ergibt sich eine veränderte \textit{Sichtweise}.}
		\item[\textbf{3}]{Ein weiteres Projekt hat keine \textit{Zukunft}.}
		\item[\textbf{4}]{Die Autorin präzisiert ihre Aussagen auf der \textit{Grundlage} ihrer Gespräche}
		\item[\textbf{5}]{Ein Zitat Schopenhauers dient als \textit{Ausgangspunkt} für den Aufsatz.}
		\item[\textbf{6}]{Es gibt keine gemeinsame \textit{übereinstimmende Meinung} für das weitere Vorgehen.}
	\end{itemize}
	\subsection*{\textbf{d}  Lesen Sie die Einleitung einer Monographie zum Thema Lebenslanges Lernen und markieren Sie alle Fachbegriffe}

	


























\end{document}