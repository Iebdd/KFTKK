\documentclass{article}

\usepackage{geometry}
\usepackage{makecell}
\usepackage{array}
\usepackage{multicol}
\usepackage{xcolor}
\usepackage{setspace}
\usepackage{ulem}
\usepackage{changepage}
\usepackage{booktabs}
\usepackage{graphicx}
\usepackage{float}
\newcolumntype{?}{!{\vrule width 1pt}}
\renewcommand\theadalign{tl}
\setstretch{1.10}
\setlength{\parindent}{0pt}

\geometry{top=12mm, left=1cm, right=2cm}
\title{\vspace{-3cm}22S 520.234 Deutsch: Intralinguale Textarbeit II}
\author{Andreas Hofer}

\begin{document}
	\section*{Fremdwörter - deutschsprachige Wörter - Fachbegriffe}
	\subsection*{\textbf{a}  Welche Verben sind äquivalent? Ordnen Sie den Fremdwörtern ihre deutschsprachigen Entsprechungen zu.}
	\begin{itemize}
		\item[\textbf{1 - g}]{antizipieren - vorwegnehmen}
		\item[\textbf{2 - b}]{diskutieren - besprechen}
		\item[\textbf{3 - d}]{ich positionieren zu - seine Meinung zu etwas äußern}
		\item[\textbf{4 - a}]{sich positionierne in - sich verorten in}
		\item[\textbf{5 - e}]{determiniert sein durch - bestimmt werden durch}
		\item[\textbf{6 - c}]{evaluieren - beurteilen}
		\item[\textbf{7 - f}]{eruieren - herausfinden}
	\end{itemize}
	\subsection*{\textbf{b}  Ergänzen Sie zunächst den Artikel der Nomen. Bilden Sie anschließend Wortpaare. Welche Wörter sind äquivalent?}
	\begin{itemize}
		\item[\textbf{1 - g}]{das Fazit - das Ergebnis}
		\item[\textbf{2 - b}]{der Aspekt - der Gesichtspunkt}
		\item[\textbf{3 - a}]{die Deskription - die Beschreibung}
		\item[\textbf{4 - c}]{die Position - der Standpunkt}
		\item[\textbf{5 - e}]{die Option - die Möglichkeit}
		\item[\textbf{6 - f}]{die Rezension - die Besprechung}
		\item[\textbf{7 - d}]{die Situation - die Lage}
	\end{itemize}
	\subsection*{\textbf{c}  Vergleichen Sie die Bedeutung der Wörter \textit{Perspektive} und \textit{Basis} in den folgenden Beispielsätzen. Wählen Sie für jeden Satz das passende deutschsprachige Äquivalent aus.}
	\begin{itemize}
		\item[\textbf{1}]{Aus diesem \textit{Blickwinkel} ist keine Lösung zu erkennen.}
		\item[\textbf{2}]{Daraus ergibt sich eine veränderte \textit{Sichtweise}.}
		\item[\textbf{3}]{Ein weiteres Projekt hat keine \textit{Zukunft}.}
		\item[\textbf{4}]{Die Autorin präzisiert ihre Aussagen auf der \textit{Grundlage} ihrer Gespräche}
		\item[\textbf{5}]{Ein Zitat Schopenhauers dient als \textit{Ausgangspunkt} für den Aufsatz.}
		\item[\textbf{6}]{Es gibt keine gemeinsame \textit{übereinstimmende Meinung} für das weitere Vorgehen.}
	\end{itemize}
	\subsection*{\textbf{a}  Lesen Sie die Einleitung einer Monographie zum Thema Lebenslanges Lernen und markieren Sie alle Fachbegriffe}
	Die vorliegende Einführung beleuchtet das Lebenslange Lernen in seinen vielfältigen Dimensionen. Im \textit{ersten} Kapitel wird \textcolor{red}{herausgearbeitet}, dass das Lebenslange Lernen zwar als gleichsam natürliches, mit dem Leben \textcolor{red}{konstitutiv} verbundenes Phänomen anzusehen ist, dass diese Selbstverständlichkeit aber mit der \textcolor{red}{Etablierung} eines \textcolor{red}{gesellschaftlichen Diskurses} zu diesem Thema verloren gegangen ist. Nun bildet das Lebenslange Lernen den Gegenstand eines \textcolor{red}{Diskurses}, in dem das Lernen des Einzelnen, die Inhalte und Formen, die Ziele und Funktionen sowie die sozialen und \textcolor{red}{institutionellen Kontexte} des Lernens beschrieben, \textcolor{red}{konzipiert} und \textcolor{red}{normativ gefordert} werden. Die Einbettung des Lebenslangen Lernens in den gesellschaftlichen Kontext wird besonders deutlich, wenn es in einer historischen Perspektive beleuchtet wird. Das \textit{zweite} Kapitel erörtert das Lebenslange Lernen als \textcolor{red}{bildungspolitisches} Programm und das \textit{dritte} Kapitel beschreibt es als Herausforderung für die pädagogische Praxis. Das \textit{vierte} Kapitel stellt zentrale \textcolor{red}{empirische Befunde} dar und das \textit{fünfte} Kapitel befasst sich mit den theoretischen Herausforderungen, die die Hinwendung zum Lebenslangen Lernen für die Erziehungswissenschaft und Bildungsforschung zu Folge hat. Im abschließenden \textit{sechsten} Kapitel werden (neue) berufliche Tätigkeitsfelder für Pädagoginnen und Pädagogen im Feld des Lebenslangen Lernens aufgezeichnet. \\
	\subsection*{\textbf{b} Unterstreichen Sie alle Nomen und Verben, die einen Bezug zur Wissenschaft haben. Schreiben Sie sie in eine Tabelle.}
	\begin{center}
	\begin{tabular}{| c | c |}
		\toprule
		\textbf{Nomen} & \textbf{Verben} \\ \midrule
		die Einführung, -en & beleuchten + A \\
		die Dimension, -en & herausarbeiten, dass + Nebensatz \\
		das Phänomen, -e & ansehen + erweiterte Verbform \\
		die Etablierung, -en & verloren + gehen \\
		der Diskurs, -e & bilden + A \\
		der Inhalt, -e & beschreiben \\
		die Form, -en & konzipieren \\
		der Kontext, -e & fordern \\
		die Einbettung, -en & deutlich + werden \\
		die Perspektive, -en & erörtern + A \\
		das Programm, -e & darstellen \\
		die Herausforderung, -en & befassen mit + D \\
		die Praxis, -en & zur Folge haben \\
		der Befund, -e & aufzeichnen \\
		die Hinwendung, -en & - \\
		die Erziehungswissenschaft, -en & - \\
		die Bildungsforschung, -en & - \\
		das Tätigkeitsfeld, -er & - \\
		die Pädagogin, -nen & - \\
		der Pädagoge, -n & - \\
		das Feld, -er & - \\
		\bottomrule
	\end{tabular}
	\end{center}

	\subsection*{\textbf{d} Ordnen Sie die Verben den Bildern zu}
	\begin{itemize}
		\item{1 - aufgreifen}
		\item{2 - zurückkommen}
		\item{3 - betrachten}
		\item{4 - sich wenden gegen}
		\item{5 - ergründen}
		\item{6 - entgegenhalten}
		\item{7 - folgen}
		\item{8 - heranziehen}
		\item{9 - abgrenzen}
		\item{10 - zusammenhängen}
	\end{itemize}
	\subsection*{\textbf{e} Welches Verb passt? Lesen Sie die Erklärungen und ergänzen Sie das Verb aus dem Schüttelkasten von Aufgabe d}
	\begin{itemize}
			\item[a]{auf ein Thema eingehen und es für sich auswerten bzw. daran anknüpfen: \textit{aufgreifen}}
			\item[b]{auf einen bereits benannten Fakt / ein bereits erwähntes Thema erneut zu sprechen kommen: \textit{zurückkommen}}
			\item[c]{etwas gegen etwas äußern / widersprechen: \textit{entgegenhalten}}
			\item[d]{etwas ablehnen / sich dagegen aussprechen : \textit{sich wenden gegen}}
			\item[e]{die Ursache einer Sacher herausfinden / etwas gründlich analysieren: \textit{ergründen}}
			\item[f]{etwas genau darstellen: \textit{betrachten}}
			\item[g]{nach einer Sache / einem Gesichtspunkt als nächstes kommen: \textit{folgen}}
			\item[h]{Beziehungen zwischen Dingen herausfinden: \textit{zusammenhängen}}
			\item[i]{etwas benutzen / sich etwas bedienen: \textit{heranziehen}}
			\item[j]{sich von einer Position distanzieren: \textit{abgrenzen}}
	\end{itemize}
	\subsection*{\textbf{a} Ergänzen Sie zu den Verben \textit{sehen, blicken und betrachten} die Nomen, gebräuchliche Fügungen mit Präpositionen und andere Wörter derselben Wortfamilie}
	\begin{tabular}{| c | c | c |}
		\toprule
		\textbf{sehen} & \textbf{blicken} & \textbf{betrachten}\\ \midrule
		die Sicht & der Blick & die Betrachtung \\
		aus (der / seiner) Sicht von & aus dem Blickwinkel von & in der Betrachtung von \\
		die Sichtweise & der Blickwinkel & die Betrachtungsweise \\
		die Rücksicht & Augenblick & in Betracht ziehen \\
		ersichtlich & zurückblickend & in Anbetracht \\
		in dieser Hinsicht & rückblickend & beträchtlich \\
		einsichtig / uneinsichi & einen Einblick geben & - \\
		- & einen Überblick geben & - \\
		- & im Hinblick auf & - \\
		\bottomrule
	\end{tabular}
	\subsection*{\textbf{b} Welches Wort passt: \textit{Sicht, Blick} oder \textit{Betrachtung}?}
	\begin{enumerate}
		\item{Die \textit{Sicht} des Autors auf diese Entwicklung ist kritisch zu sehen.}
		\item{Bei genauerer \textit{Betrachtung} der Ergebnisse ergibt sich Folgendes: ...}
		\item{Mit \textit{Blick} auf die dargestellte Situation muss gesagt werden, dass}
		\item{Als nächstes sollen die Ursachen für diese Entwicklung in den \textit{Blick} genommen werden.}
		\item{Eine gründlichere \textit{Betrachtung} der Thematik erfolgt in Kapitel 4.}
	\end{enumerate}
	\subsection*{\textbf{c} Inwiefern unterscheiden sich die Nomen aus Aufgabe 6b voneinander? Ergänzen Sie die Erklärungen mit \textit{Sicht, Blick} oder \textit{Betrachtung}.}
	\begin{enumerate}
		\item{\textit{Sicht} trägt in der Wissenschaftssprache u.a. die Bedeutung 'Meinung', 'Position' in sich.}
		\item{\textit{Blick} bedeutet u.a. etwas unter einem bestimmten Gesichtspunkt anzusehen, etwas zu berücksichtigen, um etwas anders einzuschätzen.}
		\item{\textit{Betrachtung} bedeutet, etwas gründlicher zu analysieren, genauer zu prüfen oder auch 'Abhandlung'. Hier steht der Prozess im Mittelpunkt.}
	\end{enumerate}
	\subsection*{\textbf{d} Setzen Sie die Nomen \textit{Sicht, Blick, Betrachtung} auf jeweils einer der Vokabelkarten ein}
	\begin{itemize}
		\item{aus (der) \textit{Sicht} von Meier (2009:13) ... \\ in seiner / ihrer \textit{Sicht}}
		\item{bei (genauerer / näherer / eingehender) \textit{Betrachtung}}
		\item{mit \textit{Blick} auf + A auf den ersten / zweiten ... \\ im (Hin)\textit{Blick} auf + A}	
	\end{itemize}
	\subsection*{\textbf{d} Testen Sie ihre Kollokationskompetenz: Welches Wort passt? Kreuzen Sie an.}
	\begin{tabular}{ l  l  l  l }
		\textbf{1} die \dots $\;$Arbeit & \textbf{a} vorlegen & \textbf{b} vorgelegte & \textbf{c} \textbf{vorliegende} \\
		\textbf{2} die \dots $\;$Ergebnisse & \textbf{a} \textbf{ausgeführten} & \textbf{b} \textbf{angeführten} & \textbf{c} abgeführten \\
		\textbf{3} der \dots $\;$Nachweis & \textbf{a} besprechende & \textbf{b} sprechende & \textbf{c} \textbf{entsprechende} \\
		\textbf{4} die sich daraus \dots $\;$Schlussfolgerung & \textbf{a} ergebene & \textbf{b} \textbf{ergebende} & \textbf{c} gebende \\
		\textbf{5} die später noch \dots $\;$Gründe & \textbf{a} \textbf{aufzuzeigenden}(modales Partizip) & \textbf{b} zu erkennenden & \textbf{c} gezeigten \\
		\textbf{6} das \dots $\;$Muster & \textbf{a} beschreibende & \textbf{b} \textbf{beschriebene} & \textbf{c} geschriebene \\
		\textbf{7} eine \dots $\;$Frage & \textbf{a} nächstliegende & \textbf{b} nahliegende & \textbf{c} \textbf{naheliegende} \\
	\end{tabular}
	\subsection*{\textbf{e} Setzen Sie die folgenden Wörter in die Lücken im Text:}
	\begin{enumerate}
		\item{ergeben}
		\item{betrachten}
		\item{belegen}
		\item{überraschend}
		\item{vertraut}
		\item{umzusetzen}
		\item{Richten}
	\end{enumerate}
	\subsection*{\textbf{a} Ableitungen des Verbs \textit{gehen}: Welche Bedeutung ist äquivalent? Kreuzen Sie an. Notieren Sie außerdem zu jedem Verb den Infinitiv mit passender Präposition (und ggf. dazugehörige Nomen) und Kasus. Geben Sie außerdem an, ob das Verb trennbar ist oder nicht.}
	\begin{enumerate}
		\item{Der Autor \underline{geht} erst zum Schluss \underline{auf} die Frage \underline{ein}, ob \dots - eingehen auf + A (trennbar)}
		\begin{itemize}
			\item[a]{\textbf{behandeln}}
			\item[b]{zurückkommen}
		\end{itemize}
				\item{Diese Ereignisse \underline{gingen in} die Geschichtsschreibung der Stadt \underline{ein}. - eingehen in + A (trennbar)}
		\begin{itemize}
			\item[a]{\textbf{Aufnahme finden}}
			\item[b]{gelöscht werden}
		\end{itemize}
				\item{Das \underline{geht aus} den zusammengestellten Angaben \underline{hervor}. - hervorgehen aus + D (trennbar)}
		\begin{itemize}
			\item[a]{betonen}
			\item[b]{\textbf{deutlicher werden}}
		\end{itemize}
				\item{Die Verfasserin \underline{geht} dabei folgendermaßen \underline{vor}. - vorgehen (trennbar)}
		\begin{itemize}
			\item[a]{\textbf{agieren}}
			\item[b]{sich beeilen}
		\end{itemize}
				\item{Wie \underline{mit} mit den Quellen \underline{umgeht}, ist für die Qualität der Arbeit entscheidend. - umgehen mit + D (trennbar)}
		\begin{itemize}
			\item[a]{\textbf{arbeiten mit}}
			\item[b]{auslassen}
		\end{itemize}
				\item{Im Folgenden \underline{gehe} ich der Frage \underline{nach}, inwieweit \dots - nachgehen + D (trennbar)}
		\begin{itemize}
			\item[a]{ablehnen}
			\item[b]{\textbf{betrachten}}
		\end{itemize}
				\item{Der Autor \underline{geht} dabei den Ursachen \underline{auf den Grund}. - etw. auf den Grund gehen + D (nicht trennbar)}
		\begin{itemize}
			\item[a]{\textbf{herausfinden}}
			\item[b]{genau analysieren}
		\end{itemize}
	\end{enumerate}

	\subsection*{\textbf{b} Oberflächliche Ähnlichkeit: Unterstreichen Sie in den Beispielsätzen das vollständige Verb und die dazugehörige Präposition. Ordnen Sie anschließend jedem Verb eine Bedeutung a, b oder c zu. Notieren Sie den Infinitiv mit passender Präposition und Kasus. Arbeiten Sie gegebenfalls mit einem Wörterbuch.}
	\begin{enumerate}
		\item{Diese Entwicklung \underline{führt} Schiech \underline{auf} die Revolution von 1848 \underline{zurück} - \textbf{c} - Man nennt eine Ursache für etwas}
		\item{Die Bemühungen zur Einrichtung eines solchen Rates \underline{gehen auf} die Vereinbarungen zwischen den verschiedenen EInwanderergruppen im Jahre 1976 \underline{zurück}. - \textbf{a} - etwas hat seinen Ursprung in etwas}
		\item{Abschließend \underline{komme} ich \underline{auf} die zu Beginn aufgestellte These \underline{zurück}. - \textbf{b} - man bezieht sich auf einen früher genannten Punkt / eine bereits gemachte Aussage}
	\end{enumerate}
	\subsection*{\textbf{e} Welches Verb aus dem Schüttelkasten passt in die Lücke?}
	\begin{enumerate}
		\item{Das Forschungsgebiet ist nicht zu \textit{überblicken}}
		\item{Diese Veröffentlichung kann als Grundsteinlegung dieser neuen Wissenschaft \textit{angesehen} werden.}
		\item{Von einer genaueren Darstellung wird aus Platzgründen \textit{abgesehen}.}
		\item{Nach Abschluss der Studie ist eine ausführliche Auswertung \textit{vorgesehen}}
		\item{Die Dokumente wurden mit handschriftlichen Notizen}
	\end{enumerate}
























\end{document}