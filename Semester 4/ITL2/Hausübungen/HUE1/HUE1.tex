\documentclass{article}

\usepackage{geometry}
\usepackage{makecell}
\usepackage{array}
\usepackage{multicol}
\usepackage{setspace}
\usepackage{changepage}
\usepackage{booktabs}
\usepackage{hyperref}
\usepackage[ngerman]{babel}
\usepackage{color}
\usepackage{fancyhdr}
\usepackage{datetime}
\newcolumntype{?}{!{\vrule width 1pt}}
\renewcommand\theadalign{tl}
\setstretch{1.10}
\setlength{\parindent}{0pt}

\hypersetup{
    colorlinks=true,
    urlcolor=blue,
}
\pagestyle{fancy}
\fancyhead[L]{\today}
\fancyhead[C]{Intralinguale Textarbeit II}
\fancyhead[R]{Andreas Hofer}

\geometry{top=3.5cm}
\title{21S 520.230 Deutsch: Mutter-/Bildungssprache: Textanalyse und Textproduktion Gruppe 3}
\author{Andreas Hofer}

\begin{document}
\section*{Hausübung 1}
\subsection*{Originaltext - Kinder zwischen 5 und 12 Jahre}
\href{https://nextliberty.buehnen-graz.com/stuecke/robin-hood/}{Link - Next Liberty} \\
Wer kennt ihn nicht? Robin von Locksley, den edlen Räuber, Rächer der Enterbten und sympathischen Gesetzlosen, der es wagte, sich gegen den ausbeuterischen (Möchte-Gern-)König Prinz John aufzulehnen und nun vom Sherwood Forest aus eine Bande „Vogelfreier“ an- und seine Verfolger ordentlich an der Nase herumführt. Weil Robin Gerechtigkeit über Recht stellt, den Reichen nimmt, was seiner Meinung nach den Armen zusteht, und die Hoffnung nicht aufgibt, dass der rechtmäßige König Richard Löwenherz doch irgendwann wieder von seinen Kreuzzügen zurückkehrt, wird er vom Sheriff von Nottingham und ihren (!) Soldaten steckbrieflich gesucht – und vom Volk als Held gefeiert. Als Prinz John jedoch feststellen muss, dass auch seine (Hätte-Er-Gern-)Verlobte Marian die Überzeugungen dieses Gesetzlosen teilt und kurz davor ist, die Schlossmauern und Rollenklischees gegen Prinzipien, Pfeil und Bogen zu tauschen, versucht der gekränkte Monarch mit allem, was Unrecht ist, dem „König der Diebe“ das Handwerk zu legen.\\

Mit dieser Neubearbeitung seines großen Musical-Hits spannt Robert Persché (u. a. „Aladdin und die Wunderlampe“ und „Der Zauberlehrling“) einmal mehr den Bogen vom großen Stoff der Weltliteratur hin zur Familienunterhaltung mit Ohrwurmgarantie und zeigt durch treffsichere Pointen und Held\_innen, was man mit großem Herzen, klarem Blick und mutigen Träumen im Wald vor lauter Bäumen tatsächlich alles so entdecken und erreichen kann.\\

Für einen sicheren Besuch unseres Familienmusicals bitten wir Sie, vorab die Empfehlung der Oper Graz zum Vorstellungsbesuch durchzulesen.

\subsection*{Neue Zielgruppe - Erwachsene}
Tauchen Sie ein in die wahrscheinlich bekannteste Geschichte des mittelalterlichen Großbritannien: Robin Hood. Doch Robin von Locksley, wie er offiziell bekannt ist, ist mehr als nur sein Name. So lehnt er sich gegen die tyrannische Herrschaft von Prinz John, des Bruders des rechtmäßigen Königs auf und führt einen erbitterten Partisanenkrieg aus dem Sherwood Forest. Obgleich seines gefährlichen Statuses als gesuchter Verbrecher, setzt er sich trotzdem für die Bevölkerung ein und bestiehlt die Reichen um es den Armen zu geben, während er stets die Hoffnung im Herzen trägt, dass Richard Löwenherz eines Tages von seinem Kreuzzug zurückkehrt. Doch erst als Prinz Johns Verlobte Marian flieht und sich den Widerstandskämpfern anschließt, beginnt dieser konkret mit dem Versuch sich des "Königs der Diebe" zu entledigen. \\ \\
Nach "Aladdin und die Wunderlampe" und "Der Zauberlehrling" eröffnet Robert Persché erneut die große Bühne für Weltliteratur verpackt in einem familienfreundlichen Musical, in welchem im Angesicht eines übermächtigen Feindes nie der Drang nach Gerechtigkeit aus den Augen verloren wird. \\ \\
Zur Sicherheit unserer Gäste bitten wie Sie die Empfehlung der Oper Graz zum Vorstellungsbesuch durchzulesen. \\
	

























\end{document}