\documentclass{article}

\usepackage{geometry}
\usepackage{makecell}
\usepackage{array}
\usepackage{multicol}
\usepackage{setspace}
\usepackage{changepage}
\usepackage{booktabs}
\usepackage{graphicx}
\usepackage{float}
\newcolumntype{?}{!{\vrule width 1pt}}
\renewcommand\theadalign{tl}
\setstretch{1.10}
\setlength{\parindent}{0pt}

\geometry{top=12mm, left=1cm, right=2cm}
\title{\vspace{-3cm}22S 520.230 Deutsch: Mutter-/Bildungssprache: Textanalyse und Textproduktion Gruppe 3}
\author{Andreas Hofer}

\begin{document}
	\section{Schreibzentrum Graz}
	Wissenschaftliche Texte sind umständlich formuliert, schwer verständlich und nur bestimmten Personen vorbehalten. Das besagt zumindest eine oft vertretene Meinung. Doch dem muss nicht sein. So versucht das Schreibzentrum der Universität Graz dem Mythos des wissenschaftlichen Schreibens entgegenzuwirken: \\

	\textit{"Das Schreibzentrum arbeitet unter der Annahme, dass alle Studierenden mithilfe geeigneter Strategien, in der Lage sind, das Verfassen akademischer Texte zu erlernen. Studierende können sich mit Workshops, Beratungen oder in Schreibgruppen mit den Arbeitstechniken vertraut machen und sich untereinander austauschen."}  \\

	meint \textbf{Andreas Hofer}, Mitarbeiter des Schreibzentrums und Student der Universität. So werden nicht nur theoretische Prozesse besprochen, sondern auch praktische Schreiberfahrung gesammelt. Denn wissenschaftliches Schreiben erfordert unter anderem viel Übung.  \\

	\textit{"Als eine der komplexesten Schreibpraktiken, kann man wissenschaftliches Schreiben nicht von heute auf morgen erlernen. Die Fähigkeit des Verfassens muss ausgebildet werden. Der Prozess selbst ähnelt sehr dem anderer sprachlicher Erwerbsfelder, wie dem Erzählen oder dem Schreiben im Allgemeinen."}  \\

	Es ist diese Orientierung an der Praxis, welche die Grundlage des Erfolgs des Schreibzentrums bildet. Der Kontakt zwischen Lernenden spielt hierbei auch eine große Rolle.  \\

	\textit{"Wir stellen systematisch Situationen her, in welchem zusammen mit anderen Personen, befreit von Beurteilungsdruck, diese zu erwerbenden Fähigkeiten eingeübt werden können. Besonders effektiv sind Lernerfolge, wenn sie zusammen errungen wurden, weshalb Gruppen sich in etwa auf dem gleichen Lernweg befinden sollten."} \\

	Mehr Informationen zum Angebot des Schreibzentrums sowie Termine für Workshops unter \textit{schreibzentrum.uni-graz.at}.
	Für den AirCampus der Grazer Universitäten: Andreas Hofer
	
\end{document}