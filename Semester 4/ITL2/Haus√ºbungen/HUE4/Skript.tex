\documentclass{article}

\usepackage{geometry}
\usepackage{makecell}
\usepackage{array}
\usepackage{multicol}
\usepackage{setspace}
\usepackage{changepage}
\usepackage{booktabs}
\usepackage{graphicx}
\usepackage{float}
\newcolumntype{?}{!{\vrule width 1pt}}
\renewcommand\theadalign{tl}
\setstretch{1.7}
\setlength{\parindent}{0pt}

\geometry{top=12mm, left=1cm, right=2cm}
\title{\vspace{-3cm}21S 520.230 Deutsch: Mutter-/Bildungssprache: Textanalyse und Textproduktion Gruppe 3}
\author{Andreas Hofer}

\begin{document}
	\section*{Lachen - Eine Ausstellung}
	\large
	Sehr geehrte Damen und Herren. Liebe Kinder. Es ist mir eine Freude Sie alle an diesem Abend hier begrüßen zu können. Mein Name ist Ilona Papousek und ich forsche zu Humor, Lachen und der Regulation von Emotionen an der Universität Graz. Und genau das ist auch das Thema der neuen Ausstellung hier am \textit{Center of Science Activities}. Passiert es ihnen manchmal, dass Sie das Gelächter einer Gruppe an Personen mitbekommen und obwohl es logisch gesehen keinen Sinn macht, denn Sie kennen diese Personen nicht, drängt sich das Gefühl auf, dass sie über Sie lachen. Wenn ja, dann leiden Sie vielleicht an Gelotophobie, was die Angst ausgelacht zu werden ist. Unsere Forschung hat ergeben, dass Personen die von so einer Angst betroffen sind, oft nicht in der Lage sind positives von negativem Lachen zu trennen. Positives Lachen ist, wenn man aus Heiterkeit oder Spaß lacht, während negatives Lachen auf Kosten einer anderen Person passiert. Wenn man diese nicht unterscheiden kann, dann hört es sich immer so an, als ob man ausgelacht wird, auch wenn das in keiner Weise so ist. Falls Sie jetzt Bedenken haben, dass Sie davon betroffen sind, dann empfehle ich ihnen die neue Ausstellung auszuprobieren. In 15 Stationen werden Sie aufgefordert einzuschätzen, ob mit oder über jemanden gelacht wird und am Ende erhalten Sie eine Zusammenfassung der Ergebnisse. Natürlich sollte man das nicht als endgültigen Beweis sehen, sondern mehr als Selbstversuch, ob man so gut in der Lage ist Situationen einzuschätzen wie man glaubt. Vielleicht helfen ihnen diese Aufnahmen auch Lachen in der Zukunft besser einschätzen zu können, da Sie definitiv wissen, ob eine Aufnahme positives oder negatives Lachen enthält. Reflektieren Sie wie Sie Situationen wahrnehmen und wie objektiv diese Wahrnehmung ist. Versuchen Sie sich der Linse bewusst zu werden, durch welche Sie die Welt sehen, denn nur so können Sie gezielt und überlegt handeln, was Fähigkeiten sind, die einem immer nützlich kommen können. Doch ich will Sie jetzt nicht zu lange auf die Folter spannen. Ich hoffe ihr Besuch heute ist sowohl interessant, als auch erkenntnisreich. Vielen Dank für Ihre Aufmerksamkeit.







	
\end{document}