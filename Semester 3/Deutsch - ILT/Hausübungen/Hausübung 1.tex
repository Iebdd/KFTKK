\documentclass{article}

\usepackage{geometry}
\usepackage{makecell}
\usepackage{array}
\usepackage{multicol}
\usepackage{setspace}
\usepackage{changepage}
\usepackage{booktabs}
\usepackage{graphicx}
\newcolumntype{?}{!{\vrule width 1pt}}
\renewcommand\theadalign{tl}
\linespread{1.5}
\setlength{\parindent}{0pt}

%\geometry{top=12mm, left=1cm, right=2cm}
\title{\vspace{-3.5cm}21W 520.205 Deutsch: Intralinguale Textarbeit I: Gruppe 3}
\author{Andreas Hofer}

\begin{document}
	\maketitle
	Laut der Genfer Flüchtlingskonvention ist jede Person ein Flüchtling, "die sich aus wohl begründeter Furcht vor Verfolgung wegen ihrer Rasse, Religion, Nationalität, Zugehörigkeit zu einer bestimmten Gruppe oder wegen ihrer politischen Überzeugung außerhalb ihres Herkunftsstaates befinden und den Schutz des Herkunftsstaates nicht in Anspruch nehmen können oder wegen dieser Befürchtungen nicht in Anspruch nehmen wollen.". Zusätzlich zu dieser grundlegenden Definition erweitern Staaten Gesetze, welche das Asylrecht betreffen, regelmäßig. Auch in Österreich wurden Gesetze im Zusammenhang mit dem Asylrecht bereits mehrmals abgeändert. \\

	Die letzten großen Änderungen erfuhr es in den Jahren 2003, 2005 und 2016. In der Regel brachten solche Novellen Verschärfungen des Asylrechts mit sich, weshalb deren Einführung oftmals heftig von Nichtregierungsorganisationen kritisiert wurde. Mit der Asylgesetznovelle 2003 sah die "Drittstaatsicherheit" Anwendung, durch welche Anträge von Asylwerbenden, welche aus sicheren Drittstaaten wie, unter anderem, Slowenien, der Tschechischen Republik sowie Ungarn, nach Österreich kommen, als unzulässig erachtet wurden. \\

	Mit dem Asylgesetz 2005 wurde die sogenannte "Mitwirkungspflicht" eingeführt, durch welche Asylwerbende die Asylbehörde bei der Abwicklung des Asylverfahrens nach bestem Wissen unterstützen müssen. Dies beinhaltet die Pflicht "ohne unnötigen Aufschub seinen Antrag zu begründen und alle zur Begründung des Antrags auf internationalen Schutz erforderlichen Anhaltspunkte über Nachfrage wahrheitsgemäß darzulegen". Zusätzlich muss die Behörde über jedwede Veränderung des Sachverhalts informiert werden. Eine Missachtung dieser Pflicht kann mit einer Geldstrafe von bis zu 15,000€ geahndet werden. 2016 sah das Asylgesetz mit der "Asyl auf Zeit"-Novelle die letzte große Anpassung. Die namensgebende Änderung sieht vor, dass nach Anerkennung des Asylstatuses zuerst nur eine befristetes Aufenthaltsrecht. Nach drei Jahren wird die Schutzwürdigkeit jedes Asylwerbers erneut geprüft und falls sich an der Situation im Ursprungsland nichts geändert hat, wird ein unbefristetes Aufenthaltsrecht gewährt. Kommt es im Herkunftsland jedoch zu einer "wesentlichen, dauerhaften Veränderung der spezifischen, insbesondere politischen Verhältnisse" oder liegt ein anderer Aberkennungsgrund vor, wird ein Abschiebeverfahren eingeleitet
\end{document}