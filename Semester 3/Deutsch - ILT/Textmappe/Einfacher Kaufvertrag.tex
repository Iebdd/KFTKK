\documentclass{article}

\usepackage{geometry}
\usepackage{makecell}
\usepackage{array}
\usepackage{multicol}
\usepackage{setspace}
\usepackage{changepage}
\usepackage{booktabs}
\usepackage{graphicx}
\usepackage{fancyhdr}
\usepackage{float}
\newcolumntype{?}{!{\vrule width 1pt}}
\renewcommand\theadalign{tl}
\setstretch{1.10}
\setlength{\parindent}{0pt}

\pagestyle{fancy}
\fancyhf{}
\rhead{Andreas Hofer}
\lhead{Intralinguale Textarbeit}

%\geometry{top=12mm, left=1cm, right=2cm}
\title{\vspace{-3cm}21S 520.230 Deutsch: Mutter-/Bildungssprache: Textanalyse und Textproduktion Gruppe 3}
\author{Andreas Hofer}

\begin{document}
	\section*{Kaufvertrag für eine Wohnung: Einfache Sprache}
	Max Mustermann und Mara Mustermann haben sich mit Otto Normalverbraucher auf den Kaufpreis von XX.XXX€ geeinigt. Sie werden bis spätestens 31. Februar 2021 die Summe auf sein Konto überweisen. Falls das nicht geschieht, kann Otto N. aus diesem Vertrag wieder aussteigen. Wenn er das tun will, muss er Max und Mara M. einen eingeschriebenen Brief an ihre Adresse schicken, in dem er sein Vorhaben beschreibt. \\ \\
	Bei so einem Vertragsausstieg müssen Max und Mara M. die Kosten zum Erstellen des Vertrags zu jeweils der Hälfte bezahlen. \\
	
\end{document}