\documentclass{article}

\usepackage{geometry}
\usepackage{makecell}
\usepackage{array}
\usepackage{multicol}
\usepackage{setspace}
\usepackage{changepage}
\usepackage{booktabs}
\usepackage{graphicx}
\usepackage{xcolor}
\usepackage{hyperref}
\usepackage{fancyhdr}
\usepackage{float}
\pagenumbering{arabic}
\newcolumntype{?}{!{\vrule width 1pt}}
\renewcommand\theadalign{tl}
\setstretch{1.10}
\setlength{\parindent}{0pt}

\pagestyle{fancy}
\fancyhf{}
\rhead{Andreas Hofer}
\lhead{Intralinguale Textarbeit}

%\geometry{top=12mm, left=1cm, right=2cm}
\title{\vspace{-3cm}21S 520.230 Deutsch: Mutter-/Bildungssprache: Textanalyse und Textproduktion Gruppe 3}
\author{Andreas Hofer}

\begin{document}
\section*{Umgeschriebener Rechtstext}
	Allgemeine Bestimmungen
	\begin{itemize}
		\item[§ 1.]{}
		\item[{}]{
			\begin{itemize}
				\item[(1)]{Als Maßnahme im Sinne dieser Verordnung gilt eine Pflicht des regelmäßigen Reinigens der Hände. Diese angemessene Reinigung muss die Verwendung von Seife oder einem alkoholhaltigen Desinfektionsmittel beinhalten. Diese Verordnung tritt in Kraft, wenn die Schnupfeninzidenzrate (gemäß §3 der 3. Schnupfen Maßnahmenverordnung) bei über 300 liegt.}
				\item[(2)]{Als Nachweis der regelmäßigen Reinigung gilt:
					\begin{itemize}
						\item[1.]{“Seifenpass”: Nachweis über eine unmittelbare Reinigung der Hände, wobei diese nicht länger als 60 Minuten zurückliegen darf. Der Seifenpass ist nicht gültig, wenn:
							\begin{itemize}
								\item[a)]{Seit der letzten Reinigung Handkontakt etabliert worden ist.}
							\end{itemize}}
					\end{itemize}}
			\end{itemize}}
	\end{itemize}
\end{document}