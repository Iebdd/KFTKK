\documentclass{article}

\usepackage{geometry}
\usepackage{makecell}
\usepackage{array}
\usepackage{multicol}
\usepackage{setspace}
\usepackage{changepage}
\usepackage{booktabs}
\usepackage{fancyhdr}
\usepackage{float}
\usepackage{hyperref}
\usepackage{xcolor}
\newcolumntype{?}{!{\vrule width 1pt}}
\renewcommand\theadalign{tl}
\setstretch{1.10}
\setlength{\parindent}{0pt}

\pagestyle{fancy}
\fancyhf{}
\rhead{Andreas Hofer}
\lhead{Intralinguale Textarbeit}

%\geometry{top=12mm, left=1cm, right=2cm}
\title{21W 520.205 Deutsch: Intralinguale Textarbeit I Gruppe 3}
\author{Andreas Hofer}

\begin{document}
	\section*{Verwaltung}
	Liebe Bürgerinnen und Bürger! \\ \\
	Mülltrennung wird für viele Einwohner unserer Stadt immer wichtiger und wir vom Gemeindeamt begrüßen diese Entwicklung, denn ein sauberes Graz, ist auch ein schönes Graz. Während die sachgerechte Entsorgung von Plastik, Glas sowie Bioabfällen jedoch schon außerordentlich reibungslos funktioniert, kommt es beim Restmüll immer wieder zu Komplikationen mit Sondermüll, der eigentlich separat entsorgt werden müsste. Und das mit gutem Grund! Sondermüll ist meist hochgiftig oder in einer anderen Form durch bloßen Kontakt mit Menschen schädlich. \\
	Es ist mir deshalb eine große Freude ihnen den Giftmüllexpress, die neueste Initiative der Stadt Graz, vorzustellen. Der Giftmüllexpress ist eine mobile Sammelstelle, an der Sie einige schwer zu entsorgende Materialien abgeben können. Dies beinhaltet unter anderem:
	\begin{itemize}
		\item{Laugen, Säuren oder sonstige Problemstoffe}
		\item{Elektrokleingeräte mit einer Kantenlänge von weniger als 50 cm}
		\item{Gasentladungslampen}
		\item{Batterien}
		\item{Altspeisefette und -öle}
	\end{itemize}
	Falls ich Ihr Interesse wecken konnte, zögern Sie nicht ihre Altstoffe abzugeben. Den aktuellen Fahrplan finden Sie in unserer Broschüre, welche Sie stets \href{https://www.umweltservice.graz.at/infos/abfall/Giftmuellexpress.pdf}{\color{blue}{\underline{hier}}} herunterladen können. \\

	Gemeinsam können wir unser Graz zu einer sichereren und nachhaltigeren Stadt machen! \\

	Euer Bürgermeister, \\

	Siegfried Nagl
\end{document}