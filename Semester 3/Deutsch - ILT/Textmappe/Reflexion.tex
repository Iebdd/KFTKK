\documentclass{article}

\usepackage{geometry}
\usepackage{makecell}
\usepackage{array}
\usepackage{multicol}
\usepackage{setspace}
\usepackage{changepage}
\usepackage{booktabs}
\usepackage{graphicx}
\usepackage{fancyhdr}
\usepackage{float}
\newcolumntype{?}{!{\vrule width 1pt}}
\renewcommand\theadalign{tl}
\setstretch{1.10}
\setlength{\parindent}{0pt}

\pagestyle{fancy}
\fancyhf{}
\rhead{Andreas Hofer}
\lhead{Intralinguale Textarbeit}

%\geometry{top=12mm, left=1cm, right=2cm}
\title{\vspace{-3cm}21S 520.230 Deutsch: Mutter-/Bildungssprache: Textanalyse und Textproduktion Gruppe 3}
\author{Andreas Hofer}

\begin{document}
\section*{ILT Textmappe Reflexion}
Meine Reflexion ist über den Asylrecht Text, welcher die erste Hausaufgabe darstellte. Bevor ich begann den Text zu schreiben wusste ich nicht wirklich über Asylrecht Bescheid, oder zumindest nur das, was man vom Hören-Sagen sowie in den Medien mitbekommt. Zum Beispiel wusste ich nicht, dass das Asylrecht seit 2000 bereits dreimal größere Änderungen erlebt hatte. Aus diesem Grund ist das ein Aspekt, in welchem ich während des Schreibens dieses Textes definitiv etwas gelernt habe. Weiters wollte ich die spezifischen Informationen über die Rechtstexte kennen, wobei es jedoch keine wirklich gute, und zumal objektive, Zusammenfassung gab. Es war sehr schwer objektive Informationen über die Veränderung des Asylrechts zu finden, da jegliche Texte im Internet Zeitungsartikel oder sonstiges waren, welche die, zumeist, Verschärfungen entweder lobten und die Teile die ihnen nicht so gefielen einfach wegließen oder heftig kritisierten und behaupteten, dass jegliche Änderung sowieso gegen das Menschenrecht verstößt. In dieser Hinsicht war es sehr interessant den Text den Sie uns zur Verfügung gestellt hatten nach Verfassen meines Textes erneut zu lesen, da ich davor annahm, dass es sich um einen überspitzten Artikel handelt. Als ich jedoch so viele weitere, meist gleichlautende Artikel fand, in welchen stets die Behauptung, dass es gegen die Menschenrechte verstößt, inflationär Anwendung fand, gelesen hatte, sah ich ihn in einem anderen Licht. Um das ganze Zusammenzufassen, ich habe durch diesen Text gelernt, Rechtstexte aus dem Rechtsinformationssystem des Bundes zu finden, was eine praktische Fähigkeit ist. \\
Gelungen finde ich an meinem Text die, hoffentlich, objektive und differenzierte Herangehensweise an die Änderungen an dem Asylrecht, da diese, meiner Meinung nach, nicht wertend sind, sondern einfach darstellen was verändert wurde, sowie ein paar, zumeist negative, Reaktionen auf diese.  \\
Ich hatte die Aufgabe Frau Röcklingers Text zu korrigieren und interessanterweise hat sie die Aufgabe in einer sehr anderen Weise gelöst als ich. Ich habe mich mehr auf die Rechtslage selbst konzentriert, während sie mehr Fokus darauf legte, wie man mit dieser Rechtslage umgeht. Ich glaube, dass ich meinen Text verbessern könnte, wenn man Inhalte ihres Textes in meinem ebenfalls behandelt, so das ein theoretisches sowie praktisches Bild des Asylrechts entsteht. \\
	
\end{document}