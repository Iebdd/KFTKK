\documentclass{article}

\usepackage{geometry}
\usepackage{makecell}
\usepackage{array}
\usepackage{multicol}
\usepackage{setspace}
\usepackage{changepage}
\usepackage{booktabs}
\usepackage{graphicx}
\usepackage{xcolor}
\usepackage{hyperref}
\usepackage{fancyhdr}
\usepackage{float}
\pagenumbering{arabic}
\newcolumntype{?}{!{\vrule width 1pt}}
\renewcommand\theadalign{tl}
\setstretch{1.10}
\setlength{\parindent}{0pt}

\pagestyle{fancy}
\fancyhf{}
\rhead{Andreas Hofer}
\lhead{Intralinguale Textarbeit}
\cfoot{\thepage}

%\geometry{top=12mm, left=1cm, right=2cm}
\title{\vspace{-3cm}21S 520.230 Deutsch: Mutter-/Bildungssprache: Textanalyse und Textproduktion Gruppe 3}
\author{Andreas Hofer}

\begin{document}
\section*{Umgeschriebener Rechtstext}
	Allgemeine Bestimmungen
	\begin{itemize}
		\item[§ 1.]{}
		\item[{}]{
			\begin{itemize}
				\item[(1)]{Als Maßnahme im Sinne dieser Verordnung gilt eine Pflicht des regelmäßigen Reinigens der Hände. Diese angemessene Reinigung muss die Verwendung von Seife oder einem alkoholhaltigen Desinfektionsmittel beinhalten. Diese Verordnung tritt in Kraft, wenn die Schnupfeninzidenzrate (gemäß §3 der 3. Schnupfen Maßnahmenverordnung) bei über 300 liegt.}
				\item[(2)]{Als Nachweis der regelmäßigen Reinigung gilt:
					\begin{itemize}
						\item[1.]{“Seifenpass”: Nachweis über eine unmittelbare Reinigung der Hände, wobei diese nicht länger als 60 Minuten zurückliegen darf. Der Seifenpass ist nicht gültig, wenn:
							\begin{itemize}
								\item[a)]{Seit der letzten Reinigung Handkontakt etabliert worden ist.}
							\end{itemize}}
					\end{itemize}}
			\end{itemize}}
	\end{itemize}
	\newpage

	\section*{Kaufvertrag für eine Wohnung: Einfache Sprache}
	Max Mustermann und Mara Mustermann haben sich mit Otto Normalverbraucher auf den Kaufpreis von XX.XXX€ geeinigt. Sie werden bis spätestens 31. Februar 2021 die Summe auf sein Konto überweisen. Falls das nicht geschieht, kann Otto N. aus diesem Vertrag wieder aussteigen. Wenn er das tun will, muss er Max und Mara M. einen eingeschriebenen Brief an ihre Adresse schicken, in dem er sein Vorhaben beschreibt. \\ \\
	Bei so einem Vertragsausstieg müssen Max und Mara M. die Kosten zum Erstellen des Vertrags zu jeweils der Hälfte bezahlen. \\
	\newpage

	\section*{Verwaltung}
	Liebe Bürgerinnen und Bürger! \\ \\
	Mülltrennung wird für viele Einwohner unserer Stadt immer wichtiger und wir vom Gemeindeamt begrüßen diese Entwicklung, denn ein sauberes Graz, ist auch ein schönes Graz. Während die sachgerechte Entsorgung von Plastik, Glas sowie Bioabfällen jedoch schon außerordentlich reibungslos funktioniert, kommt es beim Restmüll immer wieder zu Komplikationen mit Sondermüll, der eigentlich separat entsorgt werden müsste. Und das aus gutem Grund, denn Sondermüll ist meist hochgiftig oder in einer anderen Form durch bloßen Kontakt mit Menschen schädlich. \\
	Es ist mir deshalb eine große Freude ihnen den Giftmüllexpress, die neueste Initiative der Stadt Graz, vorzustellen. Der Giftmüllexpress ist eine mobile Sammelstelle, an der Sie einige schwer zu entsorgende Materialien abgeben können. Dies beinhaltet unter anderem:
	\begin{itemize}
		\item{Laugen, Säuren oder sonstige Problemstoffe}
		\item{Elektrokleingeräte mit einer Kantenlänge von weniger als 50 cm}
		\item{Gasentladungslampen}
		\item{Batterien}
		\item{Altspeisefette und -öle}
	\end{itemize}
	Falls ich Ihr Interesse wecken konnte, zögern Sie nicht ihre Altstoffe abzugeben. Den aktuellen Fahrplan finden Sie in der Informationsbroschüre, welche Sie stets \href{https://www.umweltservice.graz.at/infos/abfall/Giftmuellexpress.pdf}{\color{blue}{\underline{hier}}} herunterladen können. \\

	Gemeinsam machen wir unser Graz zu einer sichereren und nachhaltigeren Stadt! \\

	Ihr Bürgermeister, \\

	Siegfried Nagl
	\newpage

	\section*{Eidesstattliche Erklärung}
	Ich, Endesgefertigter Andreas Hofer geboren am 05.10.1996, in Oberndorf b. Salzburg erkläre hiermit an Eides Statt, dass ich
	\begin{itemize}
		\item{Meine Fahrprüfung in Leibnitz positiv abgeschlossen habe}
		\item{Bis zum 14.07.2021 im Besitz meines Führerscheines war, bis mir dieser entwendet wurde}
	\end{itemize}
	\vspace{10pt}
	Rosenheim am 15.07.2021 \hspace{8.5cm} \texttt{\small{Andreas Hofer}} \\
	\newpage

	\section*{Vollmacht}
	Hiermit bevollmächtige ich, Andreas Hofer geboren am 05.10.1996, in Oberndorf b. Graz, Herrn Dipl. Ing. Dr. Peter Hofer und Frau Adelheid Hofer über mein Konto mit der Nummer XXXXXXXX zu verfügen und in meinem Name Beträge abzuheben und/oder auf dem Konto einzusetzen, sowie von dem obigen Konto meine Rechnungen auszugleichen. \\ \\
	Die Vollmacht gilt ab untenstehendem Datum, bis die Vollmacht meinerseits wieder eingezogen wird. \\ \\
	Bei Kontendispositionen ist die Vollmacht im Original und der Personalausweis vorzulegen.

	\vspace*{\fill}
	Leibnitz am 12.01.2022 \hspace{8cm} \texttt{\small{Andreas Hofer}}
	\newpage

	\section*{ILT Reflexion}
	Meine Reflexion ist über den Asylrecht Text, welcher die erste Hausaufgabe darstellte. Bevor ich begann den Text zu schreiben wusste ich nicht wirklich über Asylrecht Bescheid, oder zumindest nur das, was man vom Hören-Sagen sowie in den Medien mitbekommt. Zum Beispiel wusste ich nicht, dass das Asylrecht seit 2000 bereits dreimal größere Änderungen erlebt hatte. Aus diesem Grund ist das ein Aspekt, in welchem ich während des Schreibens dieses Textes definitiv etwas gelernt habe. Weiters wollte ich die spezifischen Informationen über die Rechtstexte kennen, wobei es jedoch keine wirklich gute, und zumal objektive, Zusammenfassung gab. Es war sehr schwer objektive Informationen über die Veränderung des Asylrechts zu finden, da jegliche Texte im Internet Zeitungsartikel oder sonstiges waren, welche die, zumeist, Verschärfungen entweder lobten und die Teile die ihnen nicht so gefielen einfach wegließen oder heftig kritisierten und behaupteten, dass jegliche Änderung sowieso gegen das Menschenrecht verstößt. In dieser Hinsicht war es sehr interessant den Text den Sie uns zur Verfügung gestellt hatten nach Verfassen meines Textes erneut zu lesen, da ich davor annahm, dass es sich um einen überspitzten Artikel handelt. Als ich jedoch so viele weitere, meist gleichlautende Artikel fand, in welchen stets die Behauptung, dass es gegen die Menschenrechte verstößt, inflationär Anwendung fand, gelesen hatte, sah ich ihn in einem anderen Licht. Um das ganze Zusammenzufassen, ich habe durch diesen Text gelernt, Rechtstexte aus dem Rechtsinformationssystem des Bundes zu finden, was eine praktische Fähigkeit ist. \\
	Gelungen finde ich an meinem Text die, hoffentlich, objektive und differenzierte Herangehensweise an die Änderungen an dem Asylrecht, da diese, meiner Meinung nach, nicht wertend sind, sondern einfach darstellen was verändert wurde, sowie ein paar, zumeist negative, Reaktionen auf diese.  \\
	Ich hatte die Aufgabe Frau Röcklingers Text zu korrigieren und interessanterweise hat sie die Aufgabe in einer sehr anderen Weise gelöst als ich. Ich habe mich mehr auf die Rechtslage selbst konzentriert, während sie mehr Fokus darauf legte, wie man mit dieser Rechtslage umgeht. Ich glaube, dass ich meinen Text verbessern könnte, wenn man Inhalte ihres Textes in meinem ebenfalls behandelt, so das ein theoretisches sowie praktisches Bild des Asylrechts entsteht. \\
\end{document}