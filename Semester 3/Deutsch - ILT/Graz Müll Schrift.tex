\documentclass{article}

\usepackage{geometry}
\usepackage{makecell}
\usepackage{array}
\usepackage{multicol}
\usepackage{setspace}
\usepackage{changepage}
\usepackage{booktabs}
\usepackage{lipsum}
\newcolumntype{?}{!{\vrule width 1pt}}
\renewcommand\theadalign{tl}
\setstretch{1.5}
\setlength{\parindent}{0pt}

\geometry{top=12mm, left=1cm, right=2cm}
\title{21W 520.205 Deutsch: Intralinguale Textarbeit I Gruppe 3}
\author{Andreas Hofer}

\begin{document}
	Liebe Bürgerinnen und Bürger! \\ \\
	Mülltrennung wird für viele Einwohner unserer Stadt immer wichtiger und wir vom Gemeindeamt begrüßen diese Entwicklung, denn ein sauberes Graz, ist auch ein schönes Graz. Während die sachgerechte Entsorgung von Plastik, Glas sowie Bioabfällen jedoch schon außerordentlich reibungslos funktioniert, kommt es beim Restmüll immer wieder zu Komplikationen mit Sondermüll, der eigentlich separat entsorgt werden müsste. Und das mit gutem Grund! Sondermüll ist meist hochgiftig oder in einer anderen Form durch bloßen Kontakt mit Menschen schädlich. 
\end{document}