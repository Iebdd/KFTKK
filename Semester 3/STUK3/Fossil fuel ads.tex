\documentclass{article}

\usepackage{geometry}
\usepackage{makecell}
\usepackage{array}
\usepackage{multicol}
\usepackage{setspace}
\usepackage{changepage}
\usepackage{booktabs}
\usepackage{graphicx}
\newcolumntype{?}{!{\vrule width 1pt}}
\renewcommand\theadalign{tl}
\setstretch{1.25}
\setlength{\parindent}{0pt}

\geometry{top=12mm, left=1cm, right=2cm}
\title{English: Presentation}
\author{Andreas Hofer}


\begin{document}
	\maketitle
	\large
I am writing my CAJ about renewable energies but my presentation is about the opposite. I will be talking about fossil fuel companies and how advertisement campaigns paid by them have been downplaying climate change and when denial was no longer possible ridding themselves of responsibility for it despite having known about its effects for years. To be very specific since 1982, when Royal Dutch Shell carried out an assessment which led to predictions eerily close to what is actually happening today. One analyst said that "Global changes in air temperature would also drastically change the way people live and work" and concluded that "the changes may be the greatest in recorded history." Still, the closing remark was that the problem is still "not as significant to mankind as nuclear holocaust or world famine". Despite this knowledge this newspaper ad was published in 1991 that says "Who told you that earth is warming?...Chicken Little". Walt Disney published an animated short in 1944, with that title where a hungry wolf tricks the, in his eyes most gullibe and least intelligent chicken in the coop, to believe that the sky is falling down and as panic ensues, the wolf is able to pick the chickens off one by one. The obvious connection here is that if you believe this nonsense, you are being misled by a gullible and unintelligent person. Following this, countless opinion pieces by ostensible experts with ties to the fossil fuel industry were published until it was no longer possible to downright deny the changes CO2 had, which is why in 2008 British Petroleum sought to offload the responsibility of climate change onto ... you. To all of us. Their marketing campaigns introduced the, and I am certain most of you have heard of it, the Carbon Footprint as well as means to calculate it. By making people aware of their own contribution to climate change less scrutiny falls on the companies enabling all of it in the first place. Even today the carbon footprint is a very widely known way of finding out for how much CO2 you are personally responsible for. But the possible reduction of a singular person through informed purchases is severely limited. In 2014 students in the US calculated the average American's carbon footprint and it came down to 20.7 tons of CO2. Then they subtracted everything you can personally change, and were left with a homeless person who lives under the bridge and eats at a soup kitchen. Even this homeless person still has a carbon footprint of 8.2 tons, due to public services and construction not really being ours to change. In comparison BP in the same year, was directly responsible for 460 million tons of CO2, which is equivalent to 40 million people or about 10\% of the US-Population becoming homeless. These are the actions of one singular company and there are dozens more just like them. Real change in CO2 emissions cannot be achieved through personal change but by holding the 100 companies, which cause 71\% of CO2 emissions responsible.
\end{document}
