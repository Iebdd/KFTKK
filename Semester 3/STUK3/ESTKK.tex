\documentclass{article}

\usepackage{geometry}
\usepackage{makecell}
\usepackage{array}
\usepackage{multicol}
\usepackage{setspace}
\usepackage{changepage}
\usepackage{booktabs}
\newcolumntype{?}{!{\vrule width 1pt}}
\renewcommand\theadalign{tl}
\setstretch{1.10}
\setlength{\parindent}{0pt}

\geometry{top=12mm, left=1cm, right=2cm}
\title{Englisch: Sprach- Text- und Kulturkompetenz - 10.11.2021}
\author{Andreas Hofer}

\begin{document}
	\maketitle
	\section{Process Descriptions}
	\subsection{What does a process description do?}
	\normalsize
	It is a short report which tells the reader:
	\begin{itemize}
		\item{How something works}
		\item{How changes take place}
	\end{itemize}
	Through a series of stages. \\
	\large{Those are events that take place \textbf{regardless of the reader's actions}} \\
	\normalsize
	Do not use you or we in a process description as neither the author nor the reader or listener need to do anything for the process to take place. \\
	\section{Introduction}
	An introduction to a process description supplies a good sentence definition of the process to be analysed i.e. topic, purpose, scope. \\
	\large
	\textit{"In this presentation, I will explian how..."} \\
	\textit{One of the greatest environmental threats to our nation's agriculture is the growing acid rain problem.} \\
	\normalsize
	Each step requires a miniature process description:
	\begin{itemize}
		\item{define the step}
		\item{state its purpose (or function within the process)}
		\item{providin the necessary content}
	\end{itemize}
	
	
\end{document}