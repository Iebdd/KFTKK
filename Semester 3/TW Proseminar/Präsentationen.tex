\documentclass{article}

\usepackage{geometry}
\usepackage{makecell}
\usepackage{array}
\usepackage{multicol}
\usepackage{setspace}
\usepackage{changepage}
\usepackage{booktabs}
\usepackage{titlesec}
\newcolumntype{?}{!{\vrule width 1pt}}
\renewcommand\theadalign{tl}
\setstretch{1.10}
\setlength{\parindent}{0pt}

\titleformat{\section}
  {\normalfont\Large\bfseries}{\thesection}{1em}{}[{\titlerule[0.8pt]}]

\geometry{top=12mm, left=1cm, right=2cm}
\title{Proseminar Referate}
\author{Andreas Hofer}

\begin{document}
	\maketitle
	\section{Textanalysemodell nach Nord}
	Die Bedeutung textinterner und textexterner Faktoren. \\
	Nord: Funktionale Übersetzungstheorie. Funktion als Handlung. Skopostheorie als Ausgangsbasis für Nord, sowie funktionsgerichtetheit und Loyalität. \\
	Der Übersetzer denkt schon bei der Textanalyse an den Ausgangstext. \\
	Ein Funktionsgerechter Zieltext erfüllt die Vorgaben eines Translats. \\
	\section{Descriptive Translation Studies}
	In den 80er Jahren ging Translation mehr in Richtung des Zieltextes. \\
	3 Postulate: 
	Ausgangstextpostulat: Wenn ein Text eine Übersetzung ist, muss es den Text auch in anderen Sprachen existiert. \\
	Beziehungspostulat: 
	Normen sind gewisse Erwartungen einer Kultur, jede Kultur hat gewisse Normen. Normen werden als deskriptiv gesehen und haben Einfluss auf die Übersetzungspraxis. Trotzdem können sich Normen mit der Zeit ändern. Toury fasst Normen als Verhaltensmodelle aus.
\end{document}