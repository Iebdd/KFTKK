\documentclass{article}

\usepackage{geometry}
\usepackage{makecell}
\usepackage{array}
\usepackage{multicol}
\usepackage{setspace}
\usepackage{changepage}
\usepackage{booktabs}
\usepackage{graphicx}
\newcolumntype{?}{!{\vrule width 1pt}}
\renewcommand\theadalign{tl}
\setstretch{1.10}
\setlength{\parindent}{0pt}

\geometry{top=10mm, left=3cm, right=4cm}
\title{21W 520.003 Allgemein: Translationswissenschaftliches Proseminar I Gruppe 2 PS \\ 
					Vorkommnisse von translation shifts in der Deutschen Übersetzung Englischer Prosa}
\author{Andreas Hofer}

\begin{document}
	\maketitle
	Diese Proseminararbeit befasst sich mit der Frage, welche Instanzen von "translation shifts" bei der Übersetzung von Prosa aus dem Englischen in das Deutsche auftreten. Translation shifts sind gewisse Änderungen am Zieltext, welche, aufgrund gewisser Eigenschaften in der Zielsprache, während der Übersetzung vorgenommen werden müssen. Zunächst muss man die Ursachen für translation shifts definieren und diese in Kategorien teilen. Dies passiert nach dem Modell Eugene Nidas, beschrieben in seinem Buch \textit{Toward a Science of Translating (1964)}, welcher diese in drei grobe Kategorien teilt. Danach wird darauf eingegangen, in welchem Ausmaß sich die Übersetzung von Prosa mit der anderer Texte unterscheidet, sowie den unterschiedlichen Einfluss, welche translation shifts auf diese haben. Schließlich werden spezifische Vorkommnisse von translation shifts in Übersetzungen veranschaulicht, sowie auf deren Ursachen eingegangen. \\ \\ \\
	Gliederung:
	\begin{enumerate}
		\item[]{Einleitung}
		\item{Definition}
		\begin{enumerate}
			\item[1.2]{Welche Ursachen haben translation shifts? (Nida:1964)}
			\item[1.3]{Wie kann man diese Unterteilen? (Nida:1964)}
		\end{enumerate}
		\item{Spezielle Probleme bei der Übersetzung von Prosatexten}
		\item{Veranschaulichungen von translation shifts}
		\begin{enumerate}
			\item[3.1]{Ungewissheit der 2. Person Singular oder Plural}
			\item[3.2]{Übersetzung von Reimen sowie Stabreimen}
		\end{enumerate}
		\item{Schlussfolgerung}
		\item[]{Bibliographie}
	\end{enumerate}
\end{document}