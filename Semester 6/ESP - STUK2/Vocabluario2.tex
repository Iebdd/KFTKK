\documentclass{article}

\usepackage{geometry}
\usepackage{makecell}
\usepackage{array}
\usepackage{multicol}
\usepackage[ngerman=ngerman-x-latest]{hyphsubst}
\usepackage{setspace}
\usepackage{changepage}
\usepackage{booktabs}
\usepackage{graphicx}
\usepackage{float}
\newcolumntype{?}{!{\vrule width 1pt}}
\renewcommand\theadalign{tl}
\setstretch{1.10}
\setlength{\parindent}{0pt}
\geometry{top=12mm, left=1cm, right=2cm}
\title{\vspace{-3cm}22W 520.800 Spanisch: Sprach-, Text- und Kulturkompetenz 2 KS}
\author{Andreas Hofer}

\begin{document}
	\begin{multicols}{2}
	\begin{itemize}
		\item{Geografía de España}
		\begin{itemize}
			\item{\textbf{Spanische Hochebene} - \textbf{la meseta}}
			\item{\textbf{das Tal} - \textbf{el valle}}
			\item{\textbf{blühen} - \textbf{florecer}}
			\item{\textbf{die Meerenge} - \textbf{el estrecho}}
			\item{\textbf{der Affe} - \textbf{el mono}}
			\item{\textbf{das Archipel} - \textbf{el archipielago}}
			\item{\textbf{weich/mild} - \textbf{suave}}
			\item{\textbf{das Verbrechen} - \textbf{la delincuencia}}
			\item{\textbf{der Ureinwohner} - \textbf{el indigena}}
			\item{\textbf{der Berg} - \textbf{la montaña}}
			\item{\textbf{der Hügel} - \textbf{la colina}}
			\item{\textbf{das Gebirge} - \textbf{la sierra/la cordillera/la cadena montañosa}}
			\item{\textbf{die Gebirgskette} - \textbf{el sistema montañoso}}
			\item{\textbf{die Bergspitze} - \textbf{el pico}}
			\item{\textbf{der Gipfel} - \textbf{la cima/la cumbre}}
			\item{\textbf{die Wüste} - \textbf{el desierto}}
			\item{\textbf{die Ebene} - \textbf{la llanura}}
			\item{\textbf{eben} - \textbf{llano}}
			\item{\textbf{die Nordhalbkugel} - \textbf{el hemispherio norte}}
			\item{\textbf{die Fläche} - \textbf{el superficie}}
			\item{\textbf{münden} - \textbf{desembocar}}
			\item{\textbf{entspringen (Fluss)} - \textbf{nacer}}
			\item{\textbf{das Mittelmeer} - \textbf{el mar mediterráneo}}
			\item{\textbf{die Trockenheit} - \textbf{la sequía}}
			\item{\textbf{trocken} - \textbf{seco}}
			\item{\textbf{begnadigen} - \textbf{indultar}}
			\item{\textbf{der Schneesturm} - \textbf{el temporal de nieve}}
			\item{\textbf{der Nieselregen} - \textbf{la llovinza}}
			\item{\textbf{sehr kleines Dorf} - \textbf{la aldea}}
			\item{\textbf{(Wert) schätzen} - \textbf{tassar}}
		\end{itemize}
		\item{Gramatico}
		\begin{itemize}
			\item{\textbf{die Gewohnheit} - \textbf{la costumbre}}
			\item{\textbf{die Angst zu Scheitern} - \textbf{el miedo al fracaso}}
			\item{\textbf{die Resignation} - \textbf{la resignación}}
			\item{\textbf{die Trägheit} - \textbf{la inercia}}
			\item{\textbf{die Veränderung} - \textbf{el cambio}}
			\item{\textbf{das Risiko} - \textbf{el riesgo}}
			\item{\textbf{detwas hat meine Aufmerksamkeit erregt} - \textbf{algo me llama la atención}}
			\item{\textbf{jemanden rügen} - \textbf{llamar la atención}}
			\item{\textbf{zuschnüren/-binden/festbinden} - \textbf{atar}}
			\item{\textbf{die Fesseln} - \textbf{las ataduras}}
			\item{\textbf{ausreißen} - \textbf{arrancar}}
			\item{\textbf{freilassen} - \textbf{soltar}}
			\item{\textbf{sich befreien} - \textbf{soltarse}}
			\item{\textbf{weise sein} - \textbf{ser sabio}}
			\item{\textbf{die Weisheit} - \textbf{la sabiduría}}
			\item{\textbf{die Hoffnung} - \textbf{la esperanza}}
			\item{\textbf{gefangen sein} - \textbf{estar atrapado}}
			\item{\textbf{etwas vereinbaren/ausmachen} - \textbf{acordar algo}}
			\item{\textbf{das Abkommen} - \textbf{el acuerdo}}
			\item{\textbf{etwas verschlimmern} - \textbf{agravar algo}}
			\item{\textbf{sich verschlimmern} - \textbf{agravarse}}
			\item{\textbf{jemanden verständigen} - \textbf{avisar a alguien}}
			\item{\textbf{die Mitteilung} - \textbf{el aviso}}
			\item{\textbf{die Rettung} - \textbf{el rescate}}
			\item{\textbf{Immer in Subjuntivo}}
			\item{\textbf{Bevor} - \textbf{antes de que}}
			\item{\textbf{Damit} - \textbf{para que}}
			\item{\textbf{ohne dass} - \textbf{sin que}}
		\end{itemize}
	\end{itemize}
	\end{multicols}
	\begin{itemize}
		\item{El estrecho de Gibraltar:}
		\begin{itemize}
			\item{La franja de mar que se sitúa entre la Península Ibérica y Marruecos (y separa el sur de España y Marruecos). En su punto más estrecha mide 14km.}
		\end{itemize}
		\item{Las islas Baleares:}
		\begin{itemize}
			\item{Son un archipiélago español que se ubica en el mar Mediterráneo, frente la Comunidad Valenciana. Está formado por las islas de Mallorca, Menorca, Ibiza, Formentera y algunas islas más pequeñas.}
		\end{itemize}
		\item{El Ebro:}
		\begin{itemize}
			\item{El rio más largo de España. Recorre el noreste del país y desemboca en el mar.}
		\end{itemize}
	\end{itemize}






























\end{document}